\begin{wrapfigure}[22]{l}[\sidebarwidth]{\halfbarwidth}
\begin{shadebox}[SPOTTED]{\halfbarinnerwidth}
% \begin{sidebox}{SPOTTED}
You can't kill it if you don't know where it is. We assume that everyone starts out hidden, so rather than model stealth we model being spotted and stealth takes care of itself.

Hence the \SPOTTED\ marker. You can't shoot anything that doesn't have a \SPOTTED\ marker, though you can attack the zone (with the \stunt{Zone Effect} stunt) or attempt to spot.

Attack actions are restricted to units that have been spotted. All other actions can proceed without having a spotted target.

\SPOTTED\ markers are eroded by successful attempts to hide (\skill{Camouflage}) or by moving while out of the line of sight.

\SPOTTED\ markers are gained by enemy activity (\skill{Spot} observation) or by firing your weapons.

\SPOTTED\ markers might be a stack of chips under the unit (one per spotted value) or glass beads touching the model or a paper chit with the number written on it. As the value will go up and down, an easy way to record this change is the only driving requirement beyond identifying the value with a particular unit.
% \end{sidebox}
\end{shadebox}
\end{wrapfigure}
