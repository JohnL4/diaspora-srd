\newpage
\subsection{Some special Skills}\label{sec:Some special Skills} % \href{sec:id77}

\begin{description}
\item[Profession]
Profession is the only Skill that can be taken more than once. It represents the character's familiarity with the expectations of a given profession. Anyone with two levels in a given profession may confidently present themselves as a member of that profession, whether it be cobbler or diplomat. Note that a knowledge of professional standards can at times be separate from a practical expertise in the necessary subject areas. Profession: Astronomer (which involves the necessary Skills associated with holding the profession) is different from both Science (which might include astronomical knowledge as part of a general appreciation of science) and Navigation (which is the applied knowledge, using astronomy in the field, as it were). A player wishing to play an astronomer may wish to invest in all three Skills, or only some of them; each combination could yield slightly different stories. Profession allows the player to choose a particular career for his character not otherwise covered in the Skill set.

\item[Culture/Tech]
Culture/Tech represents the ability to get by on other planets within the cluster, and covers anything that would be part of regular civilian day-to-day life. It won't make you an atomic plant engineer, but it will let you wire up a VCR. The intent is more exclusive than inclusive --- it's more useful in how it keeps characters from easily performing basic tasks way outside their area of familiarity. Culture/Tech works differently than other Skills, in that ranks indicate the total number of systems in which the character is comfortable: it is not something that is subject to rolls, though lacking the Skill in an appropriate context might create a penalty to rolls on other Skills. For every rank in this skill, you get by in one additional world in the cluster. So someone with Culture/Tech 3 would note all the worlds on which the character is comfortable (i.e. the home world and three others). On the character sheet, they could use the system letter code to indicate their comfort zones: ``C/T 3 (A, B, F, D).'' This means that someone with an apex investment in the Skill is comfortable on most worlds in the cluster, but it's unlikely that anyone is happy everywhere. Players may select familiarity with cultures that do not actually exist in order to represent historical knowledge or re-enactment hobbies. If you are playing with the Weapon Familiarity optional rules below, this can be a cool way to build a high technology person with an interest (and facility) in ancient forms of warfare.

\item[Languages (optional)]
Some games may wish to add the issue of language comprehension. If so, this Skill can be included. It works like Culture/Tech with each rank corresponding to one language you can speak fluently other than your native tongue. (All characters are assumed to speak one language fluently).
\end{description}

