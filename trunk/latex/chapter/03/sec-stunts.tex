\section{Stunts}\label{sec:Stunts} % \href{sec:id82}

Each character also selects three Stunts, from a list of general Stunt types. Players are expected to define what exactly their Stunt does based on the general rule for the type of Stunt they have selected.

\rulebox{Characters have three Stunts.}

There are four well-defined categories of stunts: military-grade, have a thing, skill substitution, and alter a track. There is also room to build your own stunts without reference to these categories, which we call ``free-form'' stunts.

% \begin{description}
\subsection{military-grade}

Apply to one Skill. Military-grade (MG for short) provides a qualitative difference to its Skill: it allows you to make rolls in circumstances where you would not be able to otherwise. For weapons and armour, the character can now use and have access to non-civilian weapons. For space Skills, the effect varies by system; see Space Combat (\autoref{cha:space-combat}).

For other Skills the player may have to invent the effect. The precise effects of a military-grade Stunt can vary from game to game and from character to character.

\subsection{have a thing}

The character has an important thing at his disposal. This might be a spaceship, property, or something else. Note that this does not help with monthly maintenance costs, and so a character with this Stunt who has taken a slipship may be assuming obligations he otherwise wouldn't face.

\begin{description}
\item[Increased technology]
with the referee's approval and a good narrative, the thing may be at one tech level above the cluster maximum: e.g. ``Grand\-ma's blaster'' an ancient T3 laser pistol that survived the last collapse. This works better for personal weapons or armour than for a spaceship, though conceivably two Stunts could justify owning a spaceship a tech level higher than the system maximum (not to exceed +4, naturally).

\item[Integral equipment]
when a character's design demands that some piece of standard equipment be intrinsic to his body, this Stunt provides it. Choose a piece of equipment of T0 or lower and it will always be present. Use Aspects to provide limitations if that feels necessary. Applying this stunt twice could grant functionality of a T2 or T1 piece of gear.
\end{description}

\subsection{Skill substitution}

Apply to one Skill, to benefit either yourself or an ally.

\begin{description}
\item[Swap a Skill]
Some Skill you have can be used in place of some other Skill you also have, to a maximum value of 3. If you want to use a high-ranking Skill at a higher level, this Stunt costs you a fate point each time it is used (high-ranking Skills can be used at level 3 without paying a point).

\item[Use my Skill]
allies can use the Skill you specify (or they may use it as another Skill) instead of their own, to a maximum value of three. You need to explain how this works and it should be conditional so that is not universally applicable.

\item[Take a bonus]
allies can use a Skill of at least level 3 to receive a +1 bonus to a roll, as specified. When there are restrictions, the effects may operate at the scale of space or platoon combat, subject to the approval of the referee.

Note: Using a ``Take a bonus'' Stunt in space combat counts as the character's action for that phase, and risks incurring Skill penalties for further actions later in the turn.
\end{description}

\subsection{Alter a Track}
Improve the length or functionality of one of your stress tracks and the way hits on them are mitigated.

\subsection{Free-form stunts}
Stunts like Military-grade that can have a player-defined effect are approved under the authority of the table --- that is, they are acceptable when there is consensus from all players.

Other Stunts might be created at the discretion of the table.

% \end{description}

