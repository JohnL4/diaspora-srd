\chapter{Platoon Combat}
\label{cha:platoon-combat}
% \vfil

\begin{wrapfigure}{r}[\sidebarwidth]{\halfbarwidth}
\begin{shadebox}[Military Organization]{\halfbarinnerwidth}
% \begin{sidebox}{Military Organization}

Organizational units vary from army to army, and from world to world, but it might help to think in terms of the following:

\begin{description}
\item [A team] is composed of 2-5 soldiers of a single vehicle. This is the basic unit of platoon-scale combat, and is represented on the map by a single figure.

\item [A squad] is composed of 1-3 teams and is commanded by a Sergeant or a Corporal.

\item [A platoon] is composed of 2-6 teams, commanded by a Lieutenant.

\item [A company] is composed of 3-5 platoons, under the command of a Captain. This is the largest unit to be controlled by a single player.

\item [A battalion] is composed of 2-6 companies, under the command of a Lieutenant Co\-lo\-nel. An engagement of two battalions is at the extreme upper end of the size of what can be modeled in Platoon-scale combat in Diaspora.
\end{description}
% \end{sidebox}
\end{shadebox}
\end{wrapfigure}
% \vfil


The rules for platoon-scale wargaming are provided in Diaspora for campaigns that want to focus on military combat. Perhaps the players are old army buddies, offering their services to their planet's defense. Perhaps a heavily balkanized world is enlisting off-worlders to fight their battles for them. Perhaps the characters are a mercenary strike team, performing short-term contracts for anyone willing to pay. Whatever the story, it is easy to imagine combats where the scale of personal combat is simply inadequate: when there are too many individuals active in battle, platoon scale provides opportunities for incorporating infantry, artillery, armour and aircraft units in a ground war.

% \vfil

Some campaigns may choose to build around this sort of encounter, with the story effectively moving the characters from one mercenary ticket to the next with each game session. Others might never need these rules, but will be able to occasionally use them to model robbing an armoured vehicle.
%
% \vfil
\iflandscape{\newpage}{}

%
Whatever the application, the rules are provided as a means of representing something that could also be done within the existing Diaspora mechanisms. Operating at the platoon scale becomes an option that is available for those tables that want it.

% \vfil

The organizational unit of interest is the platoon. This is a number of single units, one of which is a leader, all in communication. The only stress track used is the \Morale{} stress track of individual units.

For infantry units, platoon scale represents combat between teams of 2-5 soldiers organized into platoons of 2-6 teams; an armour platoon might be 4 or 5 tanks. Individual characters can improve a platoon's performance, but, as in space combat, an individual can only modify the performance of the larger unit. The principles of FATE still apply, of course, and (as is common in Diaspora), they have been stripped down.

At the platoon scale, there is only a single stress track: \Morale. If a platoon loses morale, it can no longer function. Whether it has lost morale because of its ongoing frustration with lack of supply lines, because of the constant pressure of close enemy fire, or simply because most of the platoon has been killed, Diaspora models the only crucial variable of loss at the platoon scale along the axis of morale.

In most games, players will control either a single platoon, or (as is more usual when used as a stand-alone game) a company of 3-5 platoons.

As with the other mini-games, platoon-scale war-gaming can be played independent of the larger Diaspora RPG.

% \vfil
% \newpage

\section{The Map}
\label{sec:personal-combat-map}

A combat session should take place on a map laid out in zones. Transition between zones may have some action cost associated with it (doors, etc.) or not, using a mechanism referred to as a border. Range is measured in numbers of zones, and is pretty loose; but generally:

\begin{itemize}
\item Characters in the same zone are in hand-to-hand combat range. They can punch, grapple, and stab with ease.
\item Characters in adjacent zones can be poked with sticks with some effort --- a couple of meters distant or so.
\item Characters five zones apart are at the limit of effective rifle range --- hundreds of meters.
\end{itemize}

This is deliberately abstract, and involves some deliberate bending of space. Maps for a good Diaspora fight should be kept simple. We like to lay a piece of paper over the playing area and then sketch the map. When a few terrain elements have been laid down, it should become obvious how to divide it into zones and apply zone Aspects and pass values.

Avoid laying out a grid. The zone system rewards non-orthogonal layout. Zones should not only represent strict distances but also represent the relationships between space and ease of travel and view. Wide open spaces can be big, for example, while rooms in a spacecraft or building can be much smaller, becoming zones with their walls as boundaries. A long straight corridor can reasonably be a single zone if it is narrow enough that you couldn't swing a broadsword in it.

\begin{sidebox}{Map Design: Zones}

Some heuristics for zones inside structures include:

\begin{itemize}
\item Rooms with doors that close are a zone, no matter how small.
\item Split big zones up simply because the range is long (if the space is big enough to swing a broadsword).
\item Overall, try to keep the basic rules for zone ranges: same zone is punching, adjacent zone is poking, two zones away is throwing, three or more is shooting. Four zones is enough to credibly claim you can escape.
\end{itemize}

\end{sidebox}


Write the Aspects right on the map. If a zone has an Aspect (and this is a great way to model terrain effects), just write the Aspect right on the zone.

Borders can have pass values. Any borders between zones that is especially difficult to cross will have a pass value --- the number of shifts (from a successful move action) needed to pass through the border. Basic doors might have a pass value of 1 or 2. Dogged hatches might have a much higher pass value: perhaps 4 or higher. A pass value may be zero, as in an open room or an automatically opening door.

Borders that have a state change state when the pass value is paid and remain in that state until the pass value is paid again. So a door that someone has already paid to pass through is now in the open state and costs nothing to pass through until someone pays 2 movement successes (shifts) to close it. Some borders may have a state that is not reversible --- for example an obstacle that must be dismantled somehow and cannot easily be put back together --- in which case the border reverts permanently to the new state's pass value (probably zero, but a referee could get creative here). Note then that the pass value is also the cost to change state, even when it is in a state where the effective pass value is zero.

A simple notation for borders is to use a digit representing the pass value. For borders with two states, separate the three (cost to open, cost to pass while open, cost to close) pass values with a slash. A low stone wall might be represented simply with a 2/2/2 or just 2. A dogged hatch might be 4/0/4, and would cost 4 shifts to open at which time its pass value is zero. It would take 4 shifts to then close it again. Punching a hole in a thin bulkhead might be represented 20/4/X --- costs plenty to get through and is never all that easy to crawl through the hole and isn't reversible.

Some pass values:

\begin{description}
\item[A dogged hatch] (hard to open, stays open, hard to close): 4/0/4
\item[An automatic door/hatch] (opens at the push of a button and closes automatically): 1/1/1
\item[A barrier of burning tires] (hard to clear, stays cleared): 8/0/X
\end{description}

The situation is slightly different for outdoor locations, where too many zones clutter the map.

\rulebox{Outdoors all brawling and close combat weapons have a range of 0} (combatants must be in the same zone).

\subsection{Overhead Map}
\label{sec:personal-combat-overhead-map}

An overhead map may have several zones. A region that is hard to pass may be split into more zones. Visual cues on a map can be used exactly as a worded Aspect. There may be zones without Aspects. These are areas that don't offer tactical options. The personal combat system presented here is well suited to simple maps.

\subsection{Cross Sectional Map}
\label{sec:personal-combat-cross-sectional-map}

Spacecraft will typically be organized with small decks stacked along the axis of thrust so that the ship's acceleration provides ``gravity'' for the occupants. This presents a minor problem for running personal combat: the decks are not going to be very big or very interesting, so a familiar overhead deckplan view might not be the best way to proceed.

Another possibility is to display the ship in lateral cross section instead of the usual overhead view and increase the abstraction. In this case (or any case where you want to use a cross section instead of a floorplan --- a fight in an office building, for example) it will be handy to invent a Stunt that makes a whole set of zones (a deck or a storey) behave accordingly. We'll call that set of zones a ``level.''

\begin{description}
\item[Cluttered]
a cluttered level is full of things that block line of sight and make movement difficult. It can still be huge (two, three, four, even five zones), but the clutter means that weapons cannot be used beyond range zero.
\item[Complicated]
a complicated level is such that it is impossible to acquire line of sight past an adjacent zone. around the shaft, or a floor in a hotel with many rooms. The maximum range characters can engage in is one zones regardless of the number of zones in the level.
\item[Open]
an open deck has no interesting obstructions and characters can engage at any range.
\end{description}

When using a cross sectional map, it is not necessary to represent literally the features of the interior.

As with the overhead map, borders are given numeric values for the number of shifts needed to cross.


\section{Command}\label{sec:Command}

\index{command range}
A platoon is a grouping of units that can be ordered to act by a leadership unit. By default that unit must be in the same zone as the leader, but this may be modified by Stunts. The maximum distance allowed between a unit and its leader in order to maintain membership in the platoon is the unit's command range.
%TODO glossary: command range

\rulebox{Command range indicates how may zones a unit can be away from its leader and still be in communication with it.}

\index{OOC|see{Out Of Communication}}\index{Out Of Communication (OOC)}
Command range is zero (same zone only) unless modified by a stunt on the leader, the unit, or both. Non-leader units that do not belong to a platoon (are out of command range) may move away from enemy units, may attack enemy units that have fired upon them, and may attempt to unjam and remove Out Of Communication (\OOC) counters on themselves, but may take no other actions. They have no fate points and do not share a platoon's Consequences. Hits on these units may not be mitigated by a platoon taking a \Consequence. Units that become disassociated from their platoon do not change the platoon's fate point total.

Platoon membership is checked at the beginning of the platoon's action. Units with \OOC{} counters are only part of a platoon membership when in the same zone as the leader unit.

A leader unit with \OOC{} counters disconnects all its platoon members but suffers none of the other restrictions described here.


\section{Units}\label{sec:Units}
\iflandscape{\vfil}{}


A unit is the minimum element represented on the map for each type: a single miniature or counter. For infantry, that's a team of a few soldiers. For armour that's one vehicle. For artillery that's a battery.

There are four types of team units: infantry, artillery, armour, and aircraft. For each platoon, one unit (which may not be aircraft) is also designated the leader. There is no maximum number of units in a platoon.

Each unit in a platoon grants the platoon one fate point. All fate points are kept on the platoon and spent from the platoon. All Consequences are on the platoon.

Similarly, spin counters are associated with platoons and not with units. They may be spent by any unit in the platoon. Spin expires after having had one complete turn in which to use it (thus if spin is acquired during a defensive roll, it lasts until the end of the platoon's next opportunity to act whereas if it is acquired during movement, say, it lasts until the end of the platoons next opportunity to act and not the opportunity in which it moved).

Units that are not normally part of a platoon (typically aircraft) are associated with a particular platoon and donate their fate point to that platoon. They draw fate points from that platoon when invoking, tagging, or compelling.

All units have Skills, Aspects, Stunts, and a \Morale\ stress track. Skills are an n-cap pyramid (i.e. one Skill at level three, two Skills at level two, and three Skills at level one) or a column (i.e. one Skill at level four, one Skill at level three, one Skill at level two, and one Skill at level one).

All units have one Aspect and contribute one fate point to their platoon. Infantry units have a baseline of zero Stunts, plus one Stunt for each technology level. All other units have one Stunt, plus one additional Stunt for each technology level; consequently, units at T-1 or lower do not have Stunts. As described below, the platoon leader always has one additional Stunt, regardless of technology level. No unit may have negative Stunts.

Units have only one stress track: \Morale. When a unit takes a hit past the end of its \Morale\ track that cannot (or will not) be mitigated by a platoon \Consequence, it is eliminated. The narrative associated with this elimination can be determined by the table: it might represent panic and dispersal or surrender; a complete lack of morale is also adequately explained by everyone being killed. Some combination of the three is most likely.  The mechanical effect at this scale is the same. As with other stress tracks, a hit on a marked box rolls up to the next unmarked box.

\subsection{Skills}
\label{sec:platoon-combat-skills}

All Skills are eligible to be chosen by any unit type. See \autoref{tab:platoon-unit-skills} for a list of the unit skills available.

\begin{table}[ht]\centering
\begin{tabular}{lp{0.6\columnwidth}}
\toprule
Skill		& Roll to: \\
\midrule
Anti-air	& inflict harm on aircraft units \\
Armour		& defend against fire \\
Camouflage	& avoid detection \\
Command		& improve (repair) morale \\
Signals		& jam or unjam a unit's communications \\
Direct Fire	& inflict harm in line of sight \\
Hand-to-Hand	& inflict harm in the same zone \\
Indirect Fire	& inflict harm beyond line of sight (including off map) \\
Movement	& move. Without \skill{Movement}, a unit is only capable of advancing a single zone (the free move) per turn. \\
Observation	& detect and locate enemy artillery fire \\
\midrule
Skill		& Description \\
\midrule
Specialist	& sink \Skill\ that has no mechanical effect. The apex \Skill\ of a unit not designed for the represented form of combat (eg, artillery crews as infantry or a staff convoy as armour). \\
Veteran		& modifier to \Morale\ track \\
\bottomrule
\end{tabular}
\caption{Platoon unit skills}
\label{tab:platoon-unit-skills}
\end{table}


Units may only roll Skills for which they are trained. The only
exception is when defending against an opposed roll, in which case the
untrained Skill is presumed to be zero. The only case presented here
is Armour, though with an appropriate Stunt  Camouflage may also
qualify.


% \subsection{Unit Stunts} % should be?
\subsection{Stunts}\label{sec:unit-stunts}
\begin{description}
\item[Cavalry]
this unit is undersized and overpowered, so its maximum move is increased by one (infantry, armour, or artillery only). Infantry units may take this \Stunt\ multiple times: the unit may be thought to have an intrinsic vehicle for mobility. When taken by an armour unit, the \Stunt\ is designated ``\stunt{Light}.''

\item[Special forces]
this unit is not automatically spotted when it shares a zone with an enemy unit (infantry only).

\item[Wireless]
unit is attached to an integrated communications net, increasing command range by 1. May be taken multiple times.

\item[Engineer]
may use a successful maneuver roll to use up the free-tag on an enemy-applied \Aspect\ or make two maneuver rolls on the zone it is in instead of the usual one (infantry and armour only).

\item[Guerrilla tactics]
attacks from this unit never generate spin for the defender (infantry only).

\item[Highly trained]
this unit has one additional morale box.

\item[Infantry carrier]
this unit can carry infantry (armour or aircraft only). One infantry unit in the zone can move with this carrying unit (including traversing the Re-arm track for aircraft). The infantry unit cannot act this turn (before or after the move). The unit can begin the game carrying its infantry load. For aircraft, when the aircraft re-enters the map, the infantry is deployed and may act normally; the aircraft may not otherwise act while deploying infantry. Carried infantry do not have to be in the same platoon as the carrier.

\item[Interceptor]
if this unit is on the LAUNCH! box, it may enter the map any time an enemy aircraft enters the map and act immediately before the target aircraft can act. It may act only against this target aircraft (aircraft only).

\item[Irregulars]
this unit is an irregular non-professional unit (a sink \Stunt, chosen only to model a unit that is less effective than other units of the same technology level). Other sink \Stunts\ can be invented to fit the scenario: \stunt{Slow}, to represent a low rate of fire, etc.

\item[Long range]
ignores one zone for attack roll range modification. May be taken multiple times.

\item[Orbital]
this unit can only be attacked by fire from other orbital units (artillery only). Orbital units that are attacked with the jam action, however, take damage as though attacked with weapons (in addition to the effects of jamming).

\item[Prepared positions]
this unit was set up long before the battle (artillery only). Before combat begins, it may add a the \Aspect\ of \aspect{Locked in} to any two zones on the map. This \Aspect\ can be free-tagged by any allied artillery unit, and remains an \Aspect\ on the zone which may be tagged normally thereafter.

\item[Scatterable mines payload]
this unit can deliver area-denial ordnance (ie, mines). Pass values placed by the unit from an interdiction strike are permanent (artillery and aircraft only).

\item[Scout]
this unit can continue movement after entering a zone containing enemy units (infantry and armour only).

\item[Skill substitution]
With an appropriate narrative, additional \Stunts\ may be designed to allow \Skill\ substitutions. Each unit may only ever have one \Skill\ substitution \Stunt. The following are offered as representative examples.

\begin{description}
\item[Agile]
can use \skill{Movement} in place of \skill{Armour} (armour only).

\item[Graphite payload]
this unit can deliver payloads designed to interrupt electrical and electromagnetic function (artillery and aircraft only).  It may use its \skill{Indirect Fire} Skill to effect Jam attacks (which would normally use the \skill{Signals}). Note that this can be combined with \stunt{Zone Effects} to jam all units in a zone (regardless of owner).

\item[Shoot and scoot]
this weapon system is designed to be fired while on the move or to move very soon after firing a mission. It may use its \skill{Movement} Skill instead of \skill{Camouflage} (artillery only).

\item[Technology enhancement]
increase any \Skill\ by one. This \Stunt\ may be taken at most twice per \Skill, for a total bonus of +2.

\item[Stealth technology]
designed to hide, this unit can use \skill{Camouflage} in place of \skill{Armour} (armour only).
\end{description}

\item[VTOL]
this unit is designed to stay on target --- once on the map it may remain, moving a maximum of 1 zone (its free move) per turn (aircraft only).

\item[Zone effects]
this unit may attack all units in the target zone with one roll at -2 (armour, artillery, or aircraft only). Units do not need to be spotted to be attacked in this fashion.
\end{description}

\subsubsection{Leadership stunts}

Each platoon leader additionally chooses one of the following four Stunts.

\begin{description}
\item [Battlefield genius]
units can be one zone further from the Leader than otherwise allowed.

\item[Logistics genius]
units in platoon do not have the \aspect{Out of ammo} Aspect.

\item[Tactical genius]
units in platoon ignore one extra zone of range when attacking.

\item[Not a genius]
sink Stunt for crap commanders.
\end{description}


\subsection{Stress Tracks}\label{sec:platoon-unit-stress-tracks}

All units have a single stress track, \Morale. A platoon may expend \Consequences to mitigate hits past the end of the track on a unit. A platoon has three \Consequences\ to allocate and each can mitigate two shifts. As is standard in FATE, a platoon \Consequence\ becomes an Aspect and may be free-tagged once or compelled or tagged normally to affect any unit in the platoon.

\begin{itemize}
\item Infantry units have a base \Morale\ stress track of two boxes.
\item Armour units have a base \Morale\ stress track of one box.
\item Artillery units have a base \Morale\ stress track of two boxes.
\item Aircraft units have a base \Morale\ stress track of one box.
\end{itemize}

Unit \Morale\ tracks are modified by the unit's \Veteran\ skill. Leader units also gain a bonus \Morale.

\begin{itemize}
\item Leader units increase the base \Morale\ stress track by two.
\item Units with Veteran 1 or Veteran 2 increase their \Morale\ stress track by one.
\item Units with Veteran 3 or Veteran 4 increase their Morale stress track by two.
\item Units with Veteran 5 or Veteran 6 increase their Morale stress track by three.
\item Some stunts may further alter the length of the Morale stress track.
\end{itemize}


\subsection{Aspects}
\label{sec:platoon-unit-aspects}

All units have one descriptive Aspect chosen by the owner and add one fate point to their platoon. All units also have the Aspect \aspect{Out of ammo}. A unit, when spending fate points, expends platoon fate points. When a unit gains a fate point through a compel, that fate point belongs to the platoon.


\subsection{Infantry}
\label{sec:Infantry}

% \input{C08/TB-infantry}

Infantry units represented a small number of individuals of similar or concerted equipment: a unit typically represents 2-5 individuals, though it could be as many as 12. Specific weapons and armour per individual are not modelled except as they are represented in the Skill and Stunt list.

Infantry have a 3-cap Skill pyramid. Infantry units choose one Skill at rank 3, two Skills at rank 2, and three Skills at rank 1. They have a Morale track two boxes long. If the unit has Veteran at rank 1 or 2, the Morale track is three boxes long. If the unit has Veteran at rank 3, the Morale track is four boxes long.

The maximum movement for infantry is two zones. Infantry units may move a maximum of two zones regardless of their movement roll.


\subsection{Armour}
\label{sec:armour}

Armour units are individual tanks, cars, or other mobile armoured platform. They represent all of the equipment present on precisely that model of vehicle.

Armour has a 4-cap Skill column. Armour units have one Skill at rank 4, one at rank 3, one at rank 2, and one at rank 1. They have a Morale track one box long. If the unit has Veteran Skill 1 or 2, the Morale track is two boxes long. If the unit has Veteran Skill of 3 or 4, the Morale track is three boxes long.

The maximum movement for armour is four zones. Armour units may move a maximum of four zones regardless of their movement roll result.


\subsection{Artillery}
\label{sec:Artillery}

Artillery units are equipment capable of Indirect Fire which are kept off map. They move only in a notional sense insofar as they can roll Movement as a defensive roll against counter-battery detection and Indirect Fire. Infantry-based artillery (such as mortar or grenade launcher crews) should be represented by including an Indirect Fire Skill on an infantry unit. Artillery batteries that need to be represented on the map for purposes of the scenario should be represented by their attending personnel as lightly armed infantry units.

It can be handy to create an off-map artillery card for artillery platoons, especially if they have a command range greater than one. This will greatly simplify aircraft attacks on the artillery platoon. Artillery has a 3-cap Skill column. Artillery units have one Skill at rank 3, one at rank 2, and one at rank 1. Their Morale track is two boxes long. If the unit has Veteran Skill 1 or 2, the Morale track is three boxes long. If the unit has veteran Skill 3, then the Morale track is four boxes long.

Artillery may make Movement rolls to change position on their battery
card if there is more than one zone on the card. Moving artillery
units do not remove \SPOTTED\ markers, however.

Artillery can only fire on targets that are in line-of-sight to a
friendly unit that is currently attached to a platoon (or does not
need to be) and has no Out Of Communications (\OOC) counters.

All artillery units in the same platoon are considered to be in the
same zone as their leader for purposes of command and communication,
and for purposes of any attacks that affect all targets in a single
zone when attacked by aircraft, unless they have a Stunt that allows a
greater command range. All members of an artillery platoon not
situated on the map must be on the same artillery card (they cannot be
spread over multiple cards). Not all units in an artillery platoon
need to actually be artillery units (there might be an infantry leader
unit supplying comms and other coverage and an armour unit supplying
AA for example).
Non-artillery units in an artillery platoon must be in the off-map
artillery card in order to be associated with the platoon. This means
that, although the leader unit might be represented by something other
than artillery (an armour or infantry unit might be more advantageous)
it will gain no advantages from its mobility on the map.


\iflandscape{}{\vfil}
\subsection{Aircraft}
\label{sec:Aircraft}

Aircraft are independent units and therefore require no leader unit. They also move differently from other units: an aircraft unit may place itself on any zone on the map when its turn to act comes up.

Aircraft are automatically spotted when they are on the map.

Aircraft movement is different from other units:

Aircraft begin on the \LAUNCH\ box of the Re-arm track.

While on the Re-arm track, an aircraft unit may make Movement rolls to pro\-gress along the track.

An aircraft on the \LAUNCH\ box at the beginning of its turn may be placed on any zone on the map. Its turn is now over.

Once an aircraft acts while on the map (usually in the turn after it has moved there), it is returned to the RE-ARM box of the Re-arm track.

An aircraft unit on the map may act as any other unit except that it may not make a Movement roll.

Aircraft have a 4-cap Skill column. Aircraft units have one Skill at rank 4, one at rank 3, one at rank 2, and one at rank 1. They have a \Morale\ track one box long. If the unit has \Veteran\ Skill 1 or 2, the \Morale\ track is two boxes long. If the unit has \Veteran\ Skill 3 or 4, then the Morale track is three boxes long.

The maximum movement for aircraft is zero zones as they are not represented on the map. Aircraft units do not move on the map in the same fashion as ground units and so do not make Movement rolls except to decrease their re-arm time.

Aircraft increase range by 1 for all distance calculations (both against them and against other targets).

Aircraft can only be attacked by the Anti-air Skill.


\subsection{Leadership}
\label{sec:Leadership}

Each platoon has one, and only one, leader unit. An infantry, armour, or artillery unit can be designated a leader.

A leader unit may perform a  action in addition to its normal action. It may, therefore, make two  actions in a turn.

Leaders have one Stunt chosen from the leadership Stunts list in addition to the Stunts for their base unit type.

Leaders add two morale boxes to the unit to which they are attached. The maximum movement for leader units is their base unit's maximum movement. Leader units contribute one extra fate point to the platoon (one for the leader and one for the base unit).

Leader units have two Aspects --- one for the leader and one for the base unit --- in addition to \aspect{Out of ammo}.


\iflandscape{}{\vfil}
\subsection{Typical Units}
\label{sec:typical-units}

\input{units/lib-unit-blocks}
\begin{sidebox}{Typical T-1 units}
\centering
\begin{tikzpicture}[scale=0.75]
\tikzstyle{every node}=[transform shape]
\begin{scope}
\infantryblock%
  {Marines}% name
  {B--1}% number
  {they'll never see us comin'}% aspect
  {1/+3/Camouflage,2/+2/Direct Fire,3/+2/Observation,4/+1/Armour,5/+1/Hand-to-hand,6/+1/Command}% skills
  {1/Special Forces/T-1,2/{}/T-2,3/{}/T-3,4/{}/T-4}% stunts
  {}% move
  {4}% morale
\end{scope}
\begin{scope}[yshift=-6cm]
\armourblock%
  {Light Tanks}% name
  {B--2}% number
  {Outfitted for rough terrain}% aspect
  {1/+4/Armour,2/+3/Direct Fire,3/+2/Movement,4/+1/Anti-Air}% skills
  {1/Light/free,2/{}/T-1,3/{}/T-2,4/{}/T-3,5/{}/T-4}% stunts
  {2}% move
  {4}% morale
\end{scope}
\begin{scope}[xshift=6cm,yshift=0cm]
\artilleryblock%
  {Mortar Team}% name
  {B--3}% number
  {Pin-point accuracy}% aspect
  {1/+3/Indirect Fire,2/+2/Camouflage,3/+1/Movement}% skills
  {1/Zone Effects/free,2/{}/T-1,3/{}/T-2,4/{}/T-3,5/{}/T-4}% stunts
  {2}% move
  {4}% morale
\end{scope}
\begin{scope}[xshift=6cm,yshift=-6cm]
\aircraftblock%
  {Bomber Squad}% name
  {B--4}% number
  {Smart-bomb payload}% aspect
  {1/+4/Direct Fire,2/+3/Observation,3/+2/Anti-air,4/+1/Movement}% skills
  {1/Long Range/free,2/{}/T-1,3/{}/T-2,4/{}/T-3,5/{}/T-4}% stunts
  {2}% move
  {4}% morale
\end{scope}
\end{tikzpicture}

% \forlist{\EmptySkill}{\SkillLineNode}{one,two,three}
%   \SkillList(one,two,three)

\end{sidebox}


\vfil

\subsubsection{Typical Infantry}

Skill tree: Camouflage 3, Direct Fire 2, Observation 2, Armour 1, Hand to-Hand 1, Command 1.

Infantry are used to capture and hold territory as well as to provide spotting for heavier units.  They have NCOs capable of regrouping broken units and are adept at close combat as well as ranged. They excel at not being seen.

\subsubsection{Typical Armour}

Skill tree: Armour 4, Direct Fire 3, Movement 2, Anti-air 1.

There are two core types of armour: assault tanks designed to move into heavy fire and attack units spotted by associated infantry, and tank hunters, which would swap Armour and Direct Fire.

\vfil

\subsubsection{Typical Artillery}

Skill tree: Indirect Fire 3, Camouflage 2, Movement 1.

Artillery's immediate objective is to destroy spotted enemy equipment. It does so by projecting a huge volume of fire, which makes it suddenly very vulnerable. It offsets this vulnerability by immediately moving and re-hiding.

\subsubsection{Typical Aircraft}

Skill tree: Direct Fire 4, Observation 3, Anti-air 2, Movement 1.

Aircraft Skill trees are capped by their primary design goal --- Direct Fire for ground attack vehicles, Observation for reconnaissance craft, and Anti-air for interceptors. Most aircraft will be capable in all of these. Movement for aircraft indicates their re-arm time --- high Movement rates indicate rapid re-arming cycles, trading off for specialty effectiveness such as ground attack or anti-air capability.




\section{The Sequence}\label{sec:social-combat-sequence}

Combat occurs according to a strict sequence of events, just as with the other combat systems. In order to run the Sequence, one player should be named the caller (usually the referee, but this is not essential). The duty of the caller is to run the Sequence: he ensures that each phase is given sufficient time and that there is a smooth pace as phases proceed. The caller should have the Sequence sheet in front of him during the game.


\begin{sidebox}{The Sequence}
The Caller establishes the order characters will act in, then iterates over the following sequence, one character at a time:

\begin{enumerate}
\item Declare an action (and potentially a target).
\item \Compels{} from the table.
\item Free move (optional).
\item Roll \dplusskill{}.
\item Apply \Aspects{} and spin (player and table).
\item Action resolution and narration.
\end{enumerate}

When all players have had a turn, check timer box and determine whether victory conditions are met.
\end{sidebox}


To begin with, the caller will establish the order in which players will be polled for their actions. The best possible way to do this is the simplest way the table can all agree on: a random order proceeding clockwise, starting with the highest \skill{Charm} and then clockwise, or descending order \skill{Charm} (or whatever social Skill is most relevant). Then, for each player, the caller will ask for an action. Actions can be one of the following:

\begin{itemize}
\item Move
\item Composure attack
\item Obstruct
\item Maneuver
\item Move another
\end{itemize}

If the player is running multiple characters (as might well be the case if he is the referee), each of these characters should declare and resolve their actions separately as though run by different players.

Once the player declares his character's action and target, the caller will ask the table for compels. A compel can involve any of the acting character's \Aspect{}s, any \Aspect{} on his equipment, any \Aspect{} on the zone he is in, or any \Aspect{} on the scene. Anyone wanting to compel should hold up a fate point token and name the \Aspect{} being compelled. The caller will verify that it is a legitimate \Aspect{} for a compel and the acting player can either accept the fate point (and thus the compel) or pay the compelling player's character a fate point and deny the compel.

If a compel is accepted by the player, go to the next character (possibly one run by the same player).

Next the caller will ask the player to make his free move. The player may then move his character a single zone if he wants to.

The caller will then ask the player what Skill will be used for his action.

\rulebox{Characters in social combat may not use the same Skill twice in a row.}

Each action requires a 4dF + Skill roll to resolve. Once the dice are on the table, Aspects may be invoked or tagged by all participating players as appropriate. The usual rules for tagging Aspects apply: you may tag only one of each category of Aspect except for free-taggable Aspects, of which you may tag as many as are available. A tagged or invoked Aspect adds 2 to the roll or allows a re-roll.

During the Aspect tagging, the caller will offer all players any spin that's on the table in order to improve their rolls. It can be spent to add one to a roll.

Once all negotiable dice modifications are complete, the caller announces the resolution of the roll (who won) and directs the appropriate player to narrate the result. The authority to narrate depends upon the action declared --- see below for details.

When all players have had a turn, the caller then checks a box on the timer and determines whether the victory conditions have been met. If there is a victory, he announces it and hands control to the referee. If there is no victory, he begins the next turn.

\subsection{Move}\label{sec:Move}

For a move action, the player rolls 4dF + Skill, then modify by any Aspects tagged or invoked. He may then move his character this many zones, expending movement points as needed to erode any pass values that might be on borders between his character and his goal.

The move action represents the character aligning himself with his interests (moving towards a target zone) or feigning alignment with another in order to be more effective (moving closer to another in order to reduce range modifiers).

\subsection{Composure attack}\label{sec:Composure attack}

A Composure attack is an effort to remove a character from play altogether by attacking his Composure stress track until he is Taken Out. To begin, the acting player names the target of the attack. The attacker names his attacking Skill and the target names the Skill he will use to defend. Both will narrate their efforts or otherwise justify the Skill selection.

Both players then roll 4dF + Skill and modify through Aspect tags, invokes, and spin. Count the attacker's shifts and then reduce the shifts by the range between characters. The defender may reduce these shifts using Consequences. He may reduce the shifts by one by taking a mild Consequence, reduce by two by taking a moderate Consequence, or reduce by four by taking a severe Consequence. He may apply more than one Consequence if necessary. Each Consequence becomes a free-taggable Aspect on the character.

The remaining shifts are then used to mark the defender's Composure stress track: one box on the track is marked at the rank according to the shifts and all open boxes below it (one shift marks the first box, three shifts marks the first, second, and third box, and so on). If the highest box to be marked has already been filled, then the next highest available box is filled. If the box to be filled is past the end of the character's Composure stress track, then the character is Taken Out. The two players should negotiate what this means, mediated by the referee.

If the attacker fails his roll by three or more (gets three or more negative shifts), the defender gets spin.

The Composure attack represents an attempt to remove a character from play by making her ineffective. It might be an embarrassing anecdote designed to shame the character into removing herself from the scene, or a stinging insult that makes her too angry to act with the social subtlety necessary to participate in this kind of combat. Or something else.

\subsection{Obstruct}\label{sec:Obstruct}

The player obstructing begins by identifying the zone that will be obstructed. He then rolls 4dF + Skill - Range, modified by any Aspects tagged or invoked. If he obtains a positive result, he may apply the number of shifts as pass values on any borders in the zone. The total of all pass values added cannot exceed the number of shifts. So, if a player generated four shifts he could create a single pass value of four on one border, or a pass value of three on one border and one on another, or any other combination of pass values adding up to no more than four.

The obstruct action represents efforts to pin a character into his current mind-set or deflect him from ideas that would be contrary to the acting character's interests. This might be attempts at levity in order to block off a more sober zone, awkward geek behaviour in order to make it harder to get into an intimate zone, or similar.

\vfil

\subsection{Maneuver}\label{sec:Maneuver}

The player maneuvering begins by identifying the target of the maneuver. This target is typically a zone, but may be a character or the entire scene. He then announces the Aspect he intends to put on  the target and narrates the effort. He then rolls 4dF + Skill, modified by any Aspects tagged or invoked. If he obtains a positive result, the target acquires an Aspect described by the acting player. This Aspect is free-taggable once by any ally. Putting an Aspect of \aspect{Long-winded anecdote} on a zone will give other players a reason to avoid that zone, lest they be mired in a boring conversation, and unable to escape.

Permanent Aspects are Aspects that affect the person or zone directly. This includes things like \aspect{Liar}, \aspect{Out of crudit\'ees}, and so on. Transient Aspects are Aspects that derive from the continuous action of an individual. \aspect{Making socially unacceptable small talk}, \aspect{Spitting}, and so on. Transient Aspects last only until the placing character acts again, though he may use the Aspect in this last turn of its existence.

The caller determines whether a given \Aspect{} is permanent or transient.

% badbox here
% \newpage

\subsection{Move Another}\label{sec:Move Another}

The move another action is an attempt to force another character to move along the social map in a direction desired by the attacker. To begin, the acting player names the target of the attack. The attacker names his attacking Skill and the target names the Skill he will use to defend. Both will narrate their efforts to justify the Skill selection.

Both players then roll 4dF + Skill and modify through Aspect tags, invokes, and spin. Count the attacker's shifts and then reduce the shifts by the range between characters. These shifts are then used to move the defending player: one zone or pass value per shift, exactly as a move action.

If the attacker fails by three or more shifts, the defender is awarded a spin token.

The move another action is a careful effort to persuade. It represents effective rhetoric, brilliant argument, seduction, and like forms of persuasion. The acting character is trying to manipulate the target character directly.


\section{Damage}\label{sec:personal-combat-damage}

\index{consequences}
When a character has been hit by an attack that generates shifts, she may take damage, called \emph{stress}. Before marking the stress, she may reduce the shifts by applying one or more \Consequence{}s: a mild \Consequence{} reduces the number of shifts by one, a moderate \Consequence{} by two, and taking a severe \Consequence{} reduces the number of shifts by four.

After mitigation by \Consequence{}s, the remaining number of shifts indicate the box to be marked on the appropriate stress track. Mark this box and all boxes below it. If the highest box to be marked has already been marked, the damage ``rolls up'': mark the next higher open box and all below it.

A player may only ever have a maximum of three \Consequence{}s and may only have a maximum of one of each type regardless of the track the \Consequence{} was scored against. This means that a character suffering economic hardship (see Chapter 4) is easier to take out.

The defender determines the precise wording of the \Consequence{} (subject to reasonableness, as determined by table authority).

\subsection{Taken out}
\label{sec:personal-combat-taken-out}

A character is out of play when he sustains a hit past the end of any stress track. This means a person can be \TakenOut{} without ever taking a \Consequence{}, and therefore without ever taking any serious damage! A person that takes eight shifts of stress past the end of his \Health{} track cannot be saved. That's a one-shot kill\ldots or maybe there's a better way to narrate it?

The attacker narrates taking out his opponent (subject to reasonableness, as determined by table authority). Anything that suits the method (gunfire, punching, whatever) and that genuinely removes the character from play is suitable.

When narrating how an opponent is \TakenOut{}, it is essential to articulate how and if the opponent can return to the game. A ship that has been \TakenOut{} is no longer able to participate in space combat, but could, in theory, be boarded. In this case, the game could shift to the personal combat minigame. Or it could be destroyed, in which case it could not re-enter the game.

This gives a lot of power to the victor, and should be an incentive to players to offer concessions when things aren't going their way. A major opponent \TakenOut{} in personal combat can no longer fight, but the long-term repercussions are determined by the narrative. Being \TakenOut{} might also change features of a character sheet, though this requires some negotiation.

\subsection{Healing}
\label{sec:personal-combat-healing}

Characters cannot begin removing \Consequence{}s until the associated stress track has been cleared (and this is not instantaneous but rather dependent on the number of boxes and the associated \Skill{}).

\subsubsection{Recovering Stress Box Hits}

\rulebox{All \Health{} and \Composure{} stress hits are erased after a few days' relaxing downtime.}
Stress box hits are not real damage. They are the sweats, panic, scratches, ``only a flesh wound,'' and so on: nothing that can't be fixed with a tiny amount of downtime and nothing that actually affects performance. Consequently all \Health{} and \Composure{} stress track hits are cleared after a few days' relaxing downtime, whether that's a fancy hotel room with no one fighting in it or just the three days' travel time to the slipknot. The table should rule when enough time has passed or whether the downtime was sufficiently relaxing.

\subsubsection{Recovering Consequences}

Healing \Consequence{}s is governed, in the first instance, by an external time frame, which forces players to endure the effects of combat through the rest of the session.
\begin{itemize}
\item A mild \Consequence{} is cleared as soon as combat is over.
\item A moderate \Consequence{} remains until the end of the session.
\item A severe \Consequence{} must be carried through one complete session in which the associated stress track is never marked. If it is incurred during session one, it is gone no sooner than the end of session two, and if the associated stress track takes hit in a fight during that session, you'll need to hold the \Consequence{} through yet another one.
\end{itemize}

\subsubsection{Medics}

In addition to the purely mechanical process of recovery described above, there may be narrative reasons to introduce the need for actual medical help. The following guidelines are suggested, when pertinent. A mild \Consequence{} can be treated by a medic without a roll after the combat in which the wound was sustained is over. It requires a first-aid kit.

A moderate \Consequence{} remains until a medic can make a successful check against difficulty zero. Base time to heal is a week with (positive or negative) shifts modifying time to solve by one per shift. It requires a medical clinic (such as would be found on an ambulance or in a ship's sick bay), and the technology rating of the facility is applied as a modifier to the roll.

A severe \Consequence{} can be healed by a medic rolling against difficulty 4. It requires an advanced medical facility such as would be found in a hospital, and the technology rating of the facility is applied as a modifier to the roll. The referee may decide that the facility is, despite technology, better or worse equipped and apply this as a modifier to the difficulty. This takes one month, modified by the number of shifts achieved. In no case is the impact of the severe \Consequence{} removed before the end of the session following the one in which it was received. Example: getting a finger shot off


\iflandscape{}{\vfil}
\section{Characters in Platoon Combat}\label{sec:characters-in-platoon-combat}

A character may be associated with any unit. While more than one character can be associated with a single unit, for playability it helps to assign characters to different units, to allow players something to control during the game. A single character stand can have his base touching the associated unit base for representation, but it is easier just to note which character is associated with which unit separately. The player character moves with the unit. If the unit is destroyed, the player character is no longer involved in the combat (he's gone to ground, run off, or dead --- let the player narrate his escape).

Each character associated with a unit may, however, amplify one Skill of the unit. The player may choose which Skill gets amplified based on \autoref{tab:amplifying-platoon-skills}.

\begin{table}[ht]
\centering
\begin{tabular}{ll}
\toprule
Unit Skill	& Amplified by \\
\midrule
Anti-air	& MG Slug Thrower \\
{}		& MG Energy Weapons \\
{}		& Gunnery \\
Armour		& Tactics \\
Camouflage	& Stealth \\
{}		& Survival \\
Command		& Tactics \\
{}		& Intimidation \\
{}		& Oratory \\
Signals		& Communications \\
Direct Fire	& Slug Thrower \\
{}		& Energy Weapons \\
Hand-to-Hand	& Brawling \\
{}		& Close Combat \\
Indirect Fire	& Gunnery \\
{}		& Demolitions \\
Movement	& Tactics \\
{}		& Vehicle \\
Observation	& Alertness \\
Veteran		& Resolve \\
\bottomrule
\end{tabular}
\caption{Amplifying Platoon Skills}
\label{tab:amplifying-platoon-skills}
\end{table}


\section[Wargaming]{Wargaming}
\label{sec:personal-combat-wargaming}

Sometimes it's fun just to make one-off characters and have them shoot at each other. To play independently as a tactical war game, you need three things: a map, a story, and characters.

\subsection{The Map}
\label{sec:personal-combat-wargaming-map}

Someone is chosen as caller. Either the caller or the table draws a map. Is it a shoot out in an airport? A race to secure a bunker at the top of a hill? A boarding action in a submarine or a spaceship? Whatever the case, you need a map to play on.

You can start with a blank piece of paper, and take turns drawing features, until it looks good enough. Feel free to write words on the map too – these can become Aspects and help clarify what's what.

Once that is done, divide the map into zones. You don't want too many, but enough to allow opportunities for getting outside of range, and to allow movement. When drawing zones, it is often helpful to go from corner to corner: that means it is always clear when a character enters an area (from a door, or otherwise along a side) what zone he is in.

\subsection{The Story}
\label{sec:personal-combat-wargaming-story}

The process of drawing a map has already begun to determine what the story is: is this a fight to the death? Are there teams? Is most of the table maneuvering against a small cadre controlled by the caller (or by someone else)? Is there a difference in tech level between two sides? Whatever the case, articulating the story that is being told might mean that you go back and change the map slightly, add an Aspect to a zone or two, or whatever.

Most important is that the story articulates victory conditions, which need not be the same for all players. Is this a fight to the death? An attempt to capture someone alive? Someone working to escape detection and get out of a building, or sabotage a spacecraft's drives? Whatever the case, the victory condition might be defined in terms of time: get off the ship in eight turns; spend two turns alone in the engine room setting explosives.

\subsection{Characters in Wargaming}
\label{sec:characters-in-wargaming}

\iflandscape
{\begin{wraptable}[14]{r}[0.9\sidebarwidth]{5cm}}
{\begin{wraptable}[14]{r}[0.9\sidebarwidth]{5.5cm}}
% \begin{table}[ht]
\centering
\begin{tabular}{ll}\toprule
Skill & type \\
\midrule
Agility		\\
Alertness	\\
Brawling	& combat \\
Close Combat	& combat \\
Energy Weapons	& combat \\
EVA		\\
MicroG		\\
Resolve		& track \\
Slug Throwers	& combat \\
Stamina		& track \\
Stealth		\\
Tactics		\\
\bottomrule
\end{tabular}
\caption{Useful skills for personal combat wargaming}
\label{tab:personal-wargaming-skills}
% \end{table}
\end{wraptable}


Once the map and the story are determined, everyone should spend five minutes (no more) making one or two characters to push around the map.

\subsubsection{Skills}

Given the limited focus of this tactical game, 3-cap characters should be sufficient: pick one Skill at level 3, two at level 2, and three at level 1. Everything else is considered untrained. While any Skill might be taken, \autoref{tab:personal-wargaming-skills} presents Skills particularly relevant to this mini-game.

\subsubsection{Stress Tracks}

Characters should only concern themselves with the \Health{} and \Composure{} stress tracks. Each is three boxes long. If the character has \skill{Resolve} at level 1 or 2, the \Composure{} track has four boxes; if he has \skill{Resolve} 3, the \Composure{} track has five boxes. If the character has \skill{Stamina} at level 1 or 2, the \Health{} track has four boxes; if he has \skill{Stamina} 3, the \Health{} track has five boxes.

\subsubsection{Stunts}

Every character selects a Stunt. Making something Military-grade or altering how a stress track works are both obvious choices. (For some stories, it may be desirable to allow two Stunts per character; that's fine, as long as it's the same across the board).

\subsubsection{Aspects}

Each character should have three Aspects, revealed to all at the table. Each character also begins with three fate points.

Making a note card for each character, placed in front of the player with all the relevant information and a small pile of fate points stacked on top keeps all the information clear at all times. This is obviously scaled back from the RPG, and introduces a slightly different calculus for what constitutes a success. With reduced characters, teamwork, particularly in laying down maneuvers to be free-tagged, is rewarded.

