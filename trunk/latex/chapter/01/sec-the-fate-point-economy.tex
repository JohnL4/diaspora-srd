\section{The Fate Point Economy}\label{sec:the-fate-point-economy} %\label{sec:id58}

Fueling almost all interactions in \emph{Diaspora} is the fate point economy. Characters have fate points, as do ships, and the referee has an unending supply. Even if a given interaction doesn't actually lead to an exchange of fate points, the possibility that it might do so inevitably affects player choices. Fate points are limited, and as a scarce resource, players will be looking to spend them carefully, and collect them zealously. If a player wants something to happen and the dice have said no, then fate points provide a mechanism for the player to create success.

Fate points use other qualities of a character to create in-game effects; that is why the precise wording of an aspect can be so important. The natural instinct for players is to hoard fate points, and save them for a big flourish at the end. But there are rewards to be had in keeping the flow of fate points relatively constant. Maybe not for every roll, but regularly, fate points should be spent by players, or should be offered by the referee, to create a sense of them as units of trade, as a genuine economy, that creates an ebb and flow throughout the session.

