\section{Players get the Power}\label{sec:players-get-the-power}
\vfil

\emph{Diaspora} is not a game in which the players drive the action without the input from the referee, who only establishes the setting and mediates the rules.  But many of the ways players can use their characters' skills give the player some power over narration.

The guiding principle is \emph{say yes or roll the dice.} When a player has an idea about what he wants to happen, it can often be the case that what he wants doesn't mesh with what the referee wanted. Look at the idea, ignore your plans, and either say \emph{yes} or set a difficulty and make them roll to see what happens.

\rulebox{Say yes or roll the dice.}
\vfil

Alternatives to \emph{yes} exist that are not \emph{no.} One popular one is, \emph{yes, but...}: In this case the referee agrees but adds a complication. If everyone is grinning and nodding, the referee has succeeded. Another is \emph{yes, and...}: Here the referee agrees and escalates the player's idea even further.

Players sometimes get to say something that's true without much mediation from the referee.

We also talk frequently about \emph{the table}. The table is the sum of the players, the referee included, with all opinions weighed equally. The table is the consensus, and it is more important than any single player's authority, including the referee's.

\rulebox{The table is the consensus.}

We grant equal weight (though your table may choose otherwise) to all players throughout the first session. Cluster, world, and character creation are all egalitarian pursuits.

\subsection{Abstraction}
\label{sec:abstraction}
\vfil

% 
\begin{wraptable}{r}[\sidebarwidth]{3\sidebarwidth}
\centering
\begin{tabular}{r@{:\quad}l}
\toprule
Score	& Adjective \\
\midrule
+8	& Legendary \\
+7	& Epic \\
+6	& Fantastic \\
+5	& Superb \\
+4	& Great \\
+3	& Good \\
+2	& Decent \\
+1	& Average \\
+0	& Mediocre \\
-1	& Poor \\
-2	& Terrible \\
\bottomrule
\end{tabular}
\caption[The Adjective Ladder]{The Adjective Ladder}
\label{tab:the-ladder}
\end{wraptable}


In place of hard rules, what Diaspora rewards is narration: narration from the players, and from the referee. Giving details of what you want to happen within the game is as important as working out what roll on the dice is needed for success. Both, of course, will happen. The talk around the table will always be a mix of in-game-character-based narration and out-of-character rules discussion. There are no necessary mechanical consequences of narration in the game, but it may still prove to be the most memorable of the session.

Player authority and character integrity are both important. Because of the fate point economy, it will often be the case that the player wants something to happen that the character would not. Two things follow from this.

First, the referee keeps very little mechanical information secret. Mechanical details, such as aspects and skills, are not hidden from the players (unless there is a game-based reason why that might be). Players are always maintaining a double awareness at all times, and the tension between player and character is something that the FATE system exploits powerfully.

Second, there is a continual back and forth between these two levels, and narration, from the players and from the referee, becomes essential. A player narrates what he wants to happen, which may lead to an out-of-character tabulation of whether a roll is needed and what the target number might be. Dice are rolled, and the result leads to more narration (from the successful player, from the referee, or from the table generally) giving an interpretation of the roll within the game.

Abstraction facilitates narration, because it allows the players to define constraints or accomplishments for themselves. Narration feeds into the rules, which then in turn provide opportunities for the interpretation of a given roll, in the form of more narration. It's all about the stories.


\begin{wraptable}{r}[\sidebarwidth]{3\sidebarwidth}
\centering
\begin{tabular}{r@{:\quad}l}
\toprule
Score	& Adjective \\
\midrule
+8	& Legendary \\
+7	& Epic \\
+6	& Fantastic \\
+5	& Superb \\
+4	& Great \\
+3	& Good \\
+2	& Decent \\
+1	& Average \\
+0	& Mediocre \\
-1	& Poor \\
-2	& Terrible \\
\bottomrule
\end{tabular}
\caption[The Adjective Ladder]{The Adjective Ladder}
\label{tab:the-ladder}
\end{wraptable}


In FATE, successes and difficulties are rated by numbers or by the terms on the ladder (\autoref{tab:the-ladder}). Our Ladder here is slightly different from the \emph{Spirit of the Century} Ladder, in that the term \emph{Fair} is replaced by \emph{Decent}.

The words only applicable directly when a single character acts. Since an apex \Skill\ is at level 5 (as we will see in character generation) and since the best result from a roll of the dice is +4, a result of +9 represents an exceptionally successful attempt at something by a dedicated professional.

While higher numbers are possible (through the invocation of Aspects, described below), most numbers in the game, when all things are considered, are single digits. If one is looking for appropriate adjectives to describe an action, it is often the difference between two rolls that might determine the quality of success. So, in an opposed roll (described below, in which a player roll is compared against a referee roll) results of 7 against 5 represent a decent success.
