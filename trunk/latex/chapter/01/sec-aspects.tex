\section{Aspects}\label{sec:aspects}  %\href{sec:id59}

All characters and some things will have Aspects. These short phrases indicate what is important about the character. Scenes might have Aspects, maps might have Aspects, systems, worlds, and cities might all have Aspects. Give a thing an Aspect when you want it to have a feature but don't need a specific rule mechanism to govern how that feature operates. Instead you are declaring that something is important and leave it to players to determine how to make it important.

% \vfil
\subsection{What do Aspects Do?}\label{sec:what-do-aspects-do} %\href{sec:id60}

There are a number of ways that aspects come into being, and a number of ways they can be used during conflict (whether that's just a skill check during regular role-play or a specific roll in a combat sequence).

\rulebox{
Any time you roll the dice, you could bring Aspects into play.
}

\begin{itemize}
\item you can \emph{invoke}\index{aspects!invoking} one of your own Aspects \emph{after} you roll the dice. You narrate how the Aspect affects the roll and, assuming everyone at the table nods assent and says \emph{that's cool,} you can add +2 to your roll or re-roll. You pay a fate point immediately.

\item you can \emph{tag}\index{aspects!tagging} an Aspect on something else that's relevant to the roll \emph{after} you roll the dice. That could be an Aspect on your opponent, an ally, on a map  or any other Aspect that's relevant and not on your character. Take +2 on your roll or re-roll. You pay a fate point immediately.

\item You may only tag one Aspect on each related scope per roll:
\end{itemize}

\begin{itemize}
\item one Opponent Aspect
\item one System Aspect
\item one Scene Aspect (if one exists)
\item one Zone Aspect (if one exists)
\item one Ship Aspect (if a ship is relevant)
\item one Campaign Aspect (if one exists)
\item one Ally's Aspect
\end{itemize}

In addition, any number of free-taggable\index{aspects!free tags} Aspects from any scope may be tagged and don't count against your tagging limit (that is, you can tag two free-taggables at zone scope and still tag a third if there is one for the usual fate point cost).

\begin{itemize}
\item You can \emph{compel}\index{aspects!compelling} an Aspect on your opponent \emph{before} you roll the dice. In this case you offer your opponent a deal related to his Aspect: he can take the deal and one of your fate points or deny the deal and give you a fate point. Outside of a combat sequence the deal can be quite free-form and it is a negotiation between \emph{players} and not between characters. You might offer the Referee a deal relating to an NPC, a deal relating to an ally or most commonly a deal offered by the Referee to a character's player. During a combat sequence the effects of a compel are far more constrained (and dealt with in detail in the appropriate section).

\item You can \emph{compel} an Aspect on a scene or zone (or anything for that matter). You offer the referee a deal related to the Aspect: he can take the deal and one of your fate points or deny the deal and give you a fate point. In or out of combat, the deal is free-form and it is a negotiation between \emph{players} and not characters. You might offer the Referee a deal relating to any scope as mentioned above.
\end{itemize}

Aspects come into being in several ways:
\begin{itemize}
\item player characters start with 10 Aspects derived from the character generation stories. They get a \emph{fate refresh} of 5 fate points at the beginning of a session.

\item spaceships start with 5 Aspects\index{spaceships!aspects!starting amount} created by the designer (some forced by the design process). They start each session with five fate points.\index{aspects!on spaceships}\index{spaceships!fate points!starting amount}

\item scenes, maps, campaigns, and things get Aspects at the discretion of the referee.\index{aspects!on scenes}\index{aspects!on maps}\index{aspects!on campaigns} The referee has an unlimited supply of fate points.

\item players can put an Aspect on a character or scene with a \textbf{Maneuver}.
\end{itemize}

\begin{sidebox}{Aspects}
The selection of a character's aspects an essential part of character generation.

Aspects are the catalysts for the economies of fate points. They need to be worded in a way that you can invoke them on yourself (for when a bonus to a roll is needed), but --- more importantly --- they need to invite compels from the referee. Otherwise you lose your fate points too quickly and there is no obvious source for replenishment.

A well-worded Aspect can be both revealing of the character's nature, and be obviously invokable for both benefit and detriment.

Not all aspects can work that way, and it may emerge in play that some aspects do not enter into the fate point economy at all. They are the ones which can be traded out through the experience process.

Aspects reveal something about the character that the character may not even be aware of. Similarly, an Aspect might be a physical object (an heirloom weapon, or a spaceship). In making that choice, the player is telling the referee that this object is part of the character identity. It won't be taken away, but it will also confer obligations and responsibilities, so that it too is an active part of the economy.
\end{sidebox}

\subsection{Maneuvers}\label{sec:maneuvers} % \href{sec:id61}

A maneuver\index{maneuver} is an action your character takes that will change the status of something and this status change will be represented by the addition of an Aspect. The referee will decide what to roll (either a \nameref{sec:fixed-difficulty-roll} or an \nameref{sec:opposed-roll} --- see the \nameref{sec:resolution} section) and on success the target acquires an appropriate Aspect.\index{aspects!from maneuvers}

Having an Aspect of your choosing placed on an enemy is pretty powerful all by itself, but there is an additional power: an Aspect placed as a result of a maneuver can be tagged \emph{without paying a fate point} once by the maneuverer or an ally (it is \emph{free-taggable}).\index{maneuver!and free tags} It can be tagged normally subsequently as long as the Aspect lasts, but the first time (and only the first time) is free.

\rulebox{
Place an Aspect on an opponent or a scene with a Skill check
}
(static or opposed, as determined by the referee). If successful, the target now has the Aspect.\index{aspects!from skill checks} This Aspect can be tagged once for free and thereafter for a fate point.

