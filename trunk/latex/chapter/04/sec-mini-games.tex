\section{Mini-Games}
\label{sec:mini-games}

In Diaspora we are interested in dealing with various forms of combat as more detailed structures than a simple roll of the dice, because combat games are fun. As an adjunct to this, we want the combat games to stand on their own --- you should be able to make up guys and run a big fight with no role-playing context and no referee.  So the primary combat mini-games of Diaspora --- personal combat, social combat, platoon combat, and space combat --- can be played with or without the context of the role-playing game.

You can skip any or all of these subsystems. If combat mini-games don't interest you, the core mechanisms of FATE are certainly sufficient to resolve issues, whether gunfire is involved or not.

% \newpage
\iflandscape{}{\vfil}
\subsection{Scale}
\label{sec:scale}

There are four ``scales'' to combat in Diaspora: personal combat, space combat, social combat, and platoon combat. This demands that we address the issue of the interface between them: what happens when a guy shoots a spaceship? Or the reverse?

Nothing.

To make that a little clearer, individuals do not affect space combat directly. Spacecraft do not affect personal combat directly.

There is no strict mechanical interaction between stats in one mini game and stats in the other. Any interaction is part of the table process of negotiation for effect common to all the combat systems. So, shooting your laser (a personal combat weapon) at the hull of a spaceship does not cause hull damage (a space combat statistic) to the spaceship.

It might, however, be done to add the Aspect \emph{Weakened Hull} to the zone in the context of personal combat. A spacecraft firing a shipboard weapon at people on the planet's surface does not do Health damage, but it might place the Aspect \emph{Under bombardment} on the zone in the context of personal combat, or it might be the equivalent of an area-effect grenade going off (or more).

But the point is, the resolution must come into the context of one of the mini-games: there is no interface. Bombarding a planet might add the Aspect \emph{Ruthless killer} on a ship in the context of space combat, but the planetside effects, if pertinent, would need to be determined in the context of personal combat. If during social combat someone starts firing an assault rifle, the social combat is deemed unsuccessful, and a new map is drawn for the personal combat. Any strict mechanical linkage between Skills and stats of one system and Skills and stats of the other system will be intrinsically broken, though, so don't do it.

% \newpage

\subsection{Structure}\label{sec:Structure} % \href{sec:id106}

The rules for each of the sub-systems have at least the following parts: the map, the Sequence, and detail of the Sequence. The map describes the terrain in which combat is fought. In all sub-systems, it is abstracted, like so much of Diaspora, to allow a rough-and-ready feel without a great deal of preparation.

The Sequence outlines the order of combat, and is presented both in outline and with detailed explanations. Wherever relevant, a list of equipment is provided: guns and armour in the chapter on personal combat, typical units for platoon combat, and ships in the chapter on space combat.

% \newpage

\subsection{Mini-Games}\label{sec:Mini-Games} % \href{sec:id107}

There are four combat mini-games in Diaspora.

\begin{description}
\item[Personal combat] ~

whenever things go south in a scene and the result is violence. Personal combat assumes that there's an interesting map to be drawn and that there is something at stake. Before starting establish what the objectives are and what's at risk --- are the characters trying to escape? Trying to capture something? Beat a clock? Draw the map, set a timer if appropriate, and go. Player character Skills are highlighted.

\item[Space combat] ~

when, in space, some vessel wants another vessel to behave other than the crew desires, go to combat. As with any other combat, set the risk and set the objective. Space combat rewards escape, evade, and incapacitation over simple destruction: equal ships beating on each other is less interesting. Ship capabilities are highlighted with characters having minor influence on results.

\item[Social combat] ~

when a role-playing scene is stal\-ling with players over-thinking, planning, or otherwise not getting down to the nuts and bolts of a problem, take them to social combat. This turns the problem into an immediate tactical one where they have to solve specific problems in easily managed pieces. Use this to break up any session that's nursing a problem but not dealing with it. This is going to handle seductions,  debates, murder mysteries, and year-long political battles. Character Skills are in the spotlight.

\item[Platoon combat] ~

in military campaigns, you will sometimes find that there are scenes needing resolution that involve dozens or even hundreds of people, vehicles, and other weaponry. This is the tactical warfare mini-game, letting you get down to traditional wargame objectives with FATE mechanisms. Tank assaults, commando raids, or desperate defensive hold outs are all well modeled. Player character Skills take a background role, influencing results but dominated by the effects of technology and tactics.
\end{description}

