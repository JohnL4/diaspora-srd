\subsection{Moving within Systems}
\label{sec:moving-within-systems}


% two-column multicolumn, centred.
\newcommand{\mcc}[1]{\multicolumn{2}{c}{#1}}
\newcommand{\mcl}[1]{\multicolumn{2}{l}{#1}}
% three-column multicolumn, centred.
\newcommand{\mccc}[1]{\multicolumn{3}{c}{#1}}

\begin{table*}[ht]
\begin{center}
\begin{tabular}{
c	r@{.}l 		r@{~/~}l		*5{r@{~}l}
}
\toprule
{}	& \mcc{}	& \mcc{}		&
\multicolumn{2}{b{1.5cm}}{\multirow{2}{1.5cm}{\centering Typical slipknot}} &
\multicolumn{8}{c}{Earth to:} \\
\cmidrule{8-15}
{} & \mcc{}	& \mcc{Range, days}	&
\mcc{} &
\mcc{Moon}	& \mcc{Mars}	& \mcc{Pluto}	& \mcc{Oort Cloud} \\
V-shift & \mcc{Acc (G)}	& normal&extended	& \mcc{5AU} &
.4&Gm		& .5&AU		& 30&AU		&   1&ly \\
\midrule
0	& 0&01		& 130	& 520		&  65&d. &
34&h.	&  17&d.	& 180&d.	&   40&y. \\
1	& 0&1		& 40	& 160		&  20&d. &
11&h.	& 5.5&d.	&  56&d.	& 12.5&y. \\
2	& 0&5		& 18	& 72		&   9&d. &
5&h.	&  60&h.	&  25&d.	&  5.5&y. \\
3	& 1&0		& 13	& 52		& 6.5&d. &
3.5&h.	&  48&h.	&  18&d.	&    4&y. \\
4	& 1&5		& 10	& 40		&   5&d. &
3&h.	&  34&h.	&  14&d.	&    3&y. \\
5	& 2&0		& 9	& 36		& 4.5&d. &
2.5&h.	&  29&h.	&  12&d.	&  2.8&y. \\
6	& 3&0		& 8	& 32		&   4&d. &
2&h.	&  24&h.	&  10&d.	&  2.3&y. \\
\bottomrule \\
\end{tabular}

\begin{tabular}{*2{r@{~=~~}l}}
\mcl{Units of time}	& \mcl{Units of distance} \\
h.	& Hours		& AU	& Astronomical Unit \\
d.	& Days		& Gm	& Gigametre, 1 million kilometres \\
y.	& Years		& ly	& Light-year
\end{tabular}
\end{center}
\caption{Intersystem Travel Times}
\label{tab:intersystem-travel-times}
\vfil
\end{table*}

% \vfil

We are postulating realistic reaction motors whose efficiency changes as technology increases, but which are still fundamentally operated by sending some reaction mass (\RMass) out the back at high velocity, usually by heating it. The performance of a spacecraft is measured by its \skill{V-shift} Skill for game mechanical purposes.

Advances in technology only change little in the fact that reaction mass makes up for a majority of system level ships, with the best mass/payload ratio approaching but not going under 5:1 for 1G of thrust. Radical changes should only be expected at the far end of T4 development.

In general, travel times will be determined by the referee or the table, but you may want more detail.  \autoref{tab:intersystem-travel-times} gives some guidelines, highlighting the common case: travel between the slipknot and the habitable orbit zone of most stars.

Interplanetary distances are closest approach. Safe to say that the Oort cloud is forever out of reach to human travel: no one has enough \RMass\ to run a motor for 2 years and no one would be able to live under 3G acceleration for that time anyway.

Moving around inside a system can be extrapolated from these numbers. The typical distance to a slipstream entrance from a world on the ecliptic is around 5AU, which is a little less than the distance from the sun to Jupiter. Typical destination inside a system will be around that number, and much less inside a planetary system, traveling from moon to moon.
% \vfil

% \vfil
% {~}
\subsubsection{Resources}
% \vfil

Ships generally carry enough \RMass\ to reach a slipknot and back, with some to spare. You can go faster, using all your \RMass, to a closer destination.

Civilian ships never travel at more than 1G (\Vshift\ 3) for longer than a full day, and military ships never travel at more than 2G (\Vshift\ 5) for longer than a full day.

Any ship thrusting at its full \Vshift\ for at least the rated time to slipknot has the free-taggable Aspect \aspect{Low on R-Mass}.

Any time someone attempts to free-tag a \aspect{Low on R-Mass} Aspect, the ship's navigator may make a Navigation check against target 3 to deny it, indicating that he has plotted an extremely efficient course.

A ship that travels at speeds two \Vshift{}s lower is conserving \RMass\ to maximize effective travel range. It may use the extended range duration for the (adjusted) \Vshift\ rating.

% ~
% \newpage

% ~

\subsubsection{Overburn}

A ship may travel at one \Vshift\ higher than its stat value for no longer than the time it takes to reach a slipknot (overburn). On arrival the \Heat\ track is filled and the ship acquires the free-taggable Aspect \aspect{Low on r-mass}. If you can count on a refueling point right outside your slipstream, you can use the next better category: a \Vshift\ 2 can reach the entrance point in 6.5 days, arriving empty and helpless.

Following an overburn, a ship has an effective \Vshift\ 0 until it can re-supply. For purposes of combat, the Aspect and \Heat\ track problems should be enough to deal with.

\subsubsection{Extended Range}

Ships with the \stunt{Extended range} Stunt cannot conduct an overburn, since they are not outside the design efficiency envelope for mass versus drive design. They can, however, travel for 4 times the normal period. So, whereas a \Vshift\ 2 vessel normally has a duration of 18 days (twice the slipknot distance), one with the \stunt{Extended range} Stunt has a duration of 72 days but cannot run faster than \Vshift\ 2. Note that this does not speak to life support duration.

Any ship can be assumed to carry a year's worth of life support material. \stunt{Extended range} vessels carry twice that.

If you arrive in the middle of a firefight with a free-taggable Aspect and a full \Heat\ track, you are probably screwed and soon to be \TakenOut\ anyway. If you arrive in that state at a way-station, you are totally not screwed --- in fact you're fine. If there's no fight and no way-station, then that's a story in itself, so tell it! You don't need a mechanism to tell you what to do.

