\subsection{Slipping between Systems}\label{sec:slipping-between-systems} % \href{sec:id97}

Activating a slipdrive is relatively simple: flick a switch and you're there. A robot could do it. Problem is, you can't predict within a hundred thousand kilometers where that will be. Nor how fast you'll be going. Nor in what direction. Since a ship emerging from the slipknot has, essentially, a random vector, in almost every case one loses by going through with momentum. Except in emergencies, pilots tend to decelerate as they approach the knot.

A person with sufficient training and a good computer with up to date information, however, can significantly reduce the unknowns: momentum can usually be preserved, even if the entry vector cannot be governed.

The process requires sophisticated problem solving and pattern matching that is just not available directly to computers until the dream of artificial intelligence is realized (T4). This makes automation practically useless for military operations.

\rulebox{When exiting a slipstream, make a Navigation roll}
against the Time Table, measuring the positive or negative shifts against a target of ``a day''. The result is the time to orient the vessel and begin normal travel. If a ship had entered the slipknot without control (e.g. if there had been no deceleration), this roll is made at -2.

Automated navigation systems always score a -4 on a Navigation roll (that is, they don't roll: they always generate -4 shifts) unless it has T4 equipment. T4 equipment has arbitrary behaviour under the narrative control of the table or the referee.

All spacecraft have a Heat stress track that keeps a record of how hot the vessel is compared to how fast it can dissipate the heat. This stress track is used in combat. It also absorbs the heat that is generated while traversing the slipstream. On arrival after a slip, a ship with a T2 slipdrive has its Heat track filled. A ship with a T3 slipdrive has its highest box marked, but none below. T4 slipdrives do not generate Heat stress in the slipstream.

If a ship enters combat as it enters the system, the initial detection phase of the ship combat system replaces the orientation check. Ships leave combat oriented.

Note that a much worse roll, -3, would mean a disastrous overshot, and require ``a few weeks'' to re-orient. The ship might not have enough supplies for that, and suddenly the story becomes one of deep-space rescue.

