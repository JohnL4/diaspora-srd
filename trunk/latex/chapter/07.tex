\chapter{Social Combat}
\label{cha:social-combat}
\vfil

Social combat can be used to handle complicated social and personal situations. It adds a clear objective, so you can avoid spending a lot of energy talking fruitlessly in character when no real strategies for resolution present themselves. It gives the same opportunities to make interesting narration as the regular combat system, wrapped up in a tactical challenge.

To begin, set the stakes: establish clearly what happens if the characters win and what happens if they lose. Stakes might be ``We get the location of the secret base'' or ``We get to make a \skill{Science} roll to determine how much about the base we get to narrate'' or ``I get the girl'' or something entirely different. Losing could simply indicate failure to achieve these things, but the referee should be creative in establishing real (but interesting) losses --- failure perhaps earns the enmity of the girl's family or gets your license to practice medicine revoked.

Once the stakes are established, establish victory conditions, which depend on the map (\autoref{sec:social-combat-map}).

% badbox here
% \newpage
% \vfil

Usually the only stress track that gets action in social combat (and it doesn't need to) is the \Composure{} track on individual characters. In some cases it might make sense to place the \Wealth{} stress track at risk instead or as well, but this is at the discretion of the referee when designing the conflict.

% \vspace{\fill}

\section{The Map}
\label{sec:personal-combat-map}

A combat session should take place on a map laid out in zones. Transition between zones may have some action cost associated with it (doors, etc.) or not, using a mechanism referred to as a border. Range is measured in numbers of zones, and is pretty loose; but generally:

\begin{itemize}
\item Characters in the same zone are in hand-to-hand combat range. They can punch, grapple, and stab with ease.
\item Characters in adjacent zones can be poked with sticks with some effort --- a couple of meters distant or so.
\item Characters five zones apart are at the limit of effective rifle range --- hundreds of meters.
\end{itemize}

This is deliberately abstract, and involves some deliberate bending of space. Maps for a good Diaspora fight should be kept simple. We like to lay a piece of paper over the playing area and then sketch the map. When a few terrain elements have been laid down, it should become obvious how to divide it into zones and apply zone Aspects and pass values.

Avoid laying out a grid. The zone system rewards non-orthogonal layout. Zones should not only represent strict distances but also represent the relationships between space and ease of travel and view. Wide open spaces can be big, for example, while rooms in a spacecraft or building can be much smaller, becoming zones with their walls as boundaries. A long straight corridor can reasonably be a single zone if it is narrow enough that you couldn't swing a broadsword in it.

\begin{sidebox}{Map Design: Zones}

Some heuristics for zones inside structures include:

\begin{itemize}
\item Rooms with doors that close are a zone, no matter how small.
\item Split big zones up simply because the range is long (if the space is big enough to swing a broadsword).
\item Overall, try to keep the basic rules for zone ranges: same zone is punching, adjacent zone is poking, two zones away is throwing, three or more is shooting. Four zones is enough to credibly claim you can escape.
\end{itemize}

\end{sidebox}


Write the Aspects right on the map. If a zone has an Aspect (and this is a great way to model terrain effects), just write the Aspect right on the zone.

Borders can have pass values. Any borders between zones that is especially difficult to cross will have a pass value --- the number of shifts (from a successful move action) needed to pass through the border. Basic doors might have a pass value of 1 or 2. Dogged hatches might have a much higher pass value: perhaps 4 or higher. A pass value may be zero, as in an open room or an automatically opening door.

Borders that have a state change state when the pass value is paid and remain in that state until the pass value is paid again. So a door that someone has already paid to pass through is now in the open state and costs nothing to pass through until someone pays 2 movement successes (shifts) to close it. Some borders may have a state that is not reversible --- for example an obstacle that must be dismantled somehow and cannot easily be put back together --- in which case the border reverts permanently to the new state's pass value (probably zero, but a referee could get creative here). Note then that the pass value is also the cost to change state, even when it is in a state where the effective pass value is zero.

A simple notation for borders is to use a digit representing the pass value. For borders with two states, separate the three (cost to open, cost to pass while open, cost to close) pass values with a slash. A low stone wall might be represented simply with a 2/2/2 or just 2. A dogged hatch might be 4/0/4, and would cost 4 shifts to open at which time its pass value is zero. It would take 4 shifts to then close it again. Punching a hole in a thin bulkhead might be represented 20/4/X --- costs plenty to get through and is never all that easy to crawl through the hole and isn't reversible.

Some pass values:

\begin{description}
\item[A dogged hatch] (hard to open, stays open, hard to close): 4/0/4
\item[An automatic door/hatch] (opens at the push of a button and closes automatically): 1/1/1
\item[A barrier of burning tires] (hard to clear, stays cleared): 8/0/X
\end{description}

The situation is slightly different for outdoor locations, where too many zones clutter the map.

\rulebox{Outdoors all brawling and close combat weapons have a range of 0} (combatants must be in the same zone).

\subsection{Overhead Map}
\label{sec:personal-combat-overhead-map}

An overhead map may have several zones. A region that is hard to pass may be split into more zones. Visual cues on a map can be used exactly as a worded Aspect. There may be zones without Aspects. These are areas that don't offer tactical options. The personal combat system presented here is well suited to simple maps.

\subsection{Cross Sectional Map}
\label{sec:personal-combat-cross-sectional-map}

Spacecraft will typically be organized with small decks stacked along the axis of thrust so that the ship's acceleration provides ``gravity'' for the occupants. This presents a minor problem for running personal combat: the decks are not going to be very big or very interesting, so a familiar overhead deckplan view might not be the best way to proceed.

Another possibility is to display the ship in lateral cross section instead of the usual overhead view and increase the abstraction. In this case (or any case where you want to use a cross section instead of a floorplan --- a fight in an office building, for example) it will be handy to invent a Stunt that makes a whole set of zones (a deck or a storey) behave accordingly. We'll call that set of zones a ``level.''

\begin{description}
\item[Cluttered]
a cluttered level is full of things that block line of sight and make movement difficult. It can still be huge (two, three, four, even five zones), but the clutter means that weapons cannot be used beyond range zero.
\item[Complicated]
a complicated level is such that it is impossible to acquire line of sight past an adjacent zone. around the shaft, or a floor in a hotel with many rooms. The maximum range characters can engage in is one zones regardless of the number of zones in the level.
\item[Open]
an open deck has no interesting obstructions and characters can engage at any range.
\end{description}

When using a cross sectional map, it is not necessary to represent literally the features of the interior.

As with the overhead map, borders are given numeric values for the number of shifts needed to cross.


\section{The Sequence}\label{sec:social-combat-sequence}

Combat occurs according to a strict sequence of events, just as with the other combat systems. In order to run the Sequence, one player should be named the caller (usually the referee, but this is not essential). The duty of the caller is to run the Sequence: he ensures that each phase is given sufficient time and that there is a smooth pace as phases proceed. The caller should have the Sequence sheet in front of him during the game.


\begin{sidebox}{The Sequence}
The Caller establishes the order characters will act in, then iterates over the following sequence, one character at a time:

\begin{enumerate}
\item Declare an action (and potentially a target).
\item \Compels{} from the table.
\item Free move (optional).
\item Roll \dplusskill{}.
\item Apply \Aspects{} and spin (player and table).
\item Action resolution and narration.
\end{enumerate}

When all players have had a turn, check timer box and determine whether victory conditions are met.
\end{sidebox}


To begin with, the caller will establish the order in which players will be polled for their actions. The best possible way to do this is the simplest way the table can all agree on: a random order proceeding clockwise, starting with the highest \skill{Charm} and then clockwise, or descending order \skill{Charm} (or whatever social Skill is most relevant). Then, for each player, the caller will ask for an action. Actions can be one of the following:

\begin{itemize}
\item Move
\item Composure attack
\item Obstruct
\item Maneuver
\item Move another
\end{itemize}

If the player is running multiple characters (as might well be the case if he is the referee), each of these characters should declare and resolve their actions separately as though run by different players.

Once the player declares his character's action and target, the caller will ask the table for compels. A compel can involve any of the acting character's \Aspect{}s, any \Aspect{} on his equipment, any \Aspect{} on the zone he is in, or any \Aspect{} on the scene. Anyone wanting to compel should hold up a fate point token and name the \Aspect{} being compelled. The caller will verify that it is a legitimate \Aspect{} for a compel and the acting player can either accept the fate point (and thus the compel) or pay the compelling player's character a fate point and deny the compel.

If a compel is accepted by the player, go to the next character (possibly one run by the same player).

Next the caller will ask the player to make his free move. The player may then move his character a single zone if he wants to.

The caller will then ask the player what Skill will be used for his action.

\rulebox{Characters in social combat may not use the same Skill twice in a row.}

Each action requires a 4dF + Skill roll to resolve. Once the dice are on the table, Aspects may be invoked or tagged by all participating players as appropriate. The usual rules for tagging Aspects apply: you may tag only one of each category of Aspect except for free-taggable Aspects, of which you may tag as many as are available. A tagged or invoked Aspect adds 2 to the roll or allows a re-roll.

During the Aspect tagging, the caller will offer all players any spin that's on the table in order to improve their rolls. It can be spent to add one to a roll.

Once all negotiable dice modifications are complete, the caller announces the resolution of the roll (who won) and directs the appropriate player to narrate the result. The authority to narrate depends upon the action declared --- see below for details.

When all players have had a turn, the caller then checks a box on the timer and determines whether the victory conditions have been met. If there is a victory, he announces it and hands control to the referee. If there is no victory, he begins the next turn.

\subsection{Move}\label{sec:Move}

For a move action, the player rolls 4dF + Skill, then modify by any Aspects tagged or invoked. He may then move his character this many zones, expending movement points as needed to erode any pass values that might be on borders between his character and his goal.

The move action represents the character aligning himself with his interests (moving towards a target zone) or feigning alignment with another in order to be more effective (moving closer to another in order to reduce range modifiers).

\subsection{Composure attack}\label{sec:Composure attack}

A Composure attack is an effort to remove a character from play altogether by attacking his Composure stress track until he is Taken Out. To begin, the acting player names the target of the attack. The attacker names his attacking Skill and the target names the Skill he will use to defend. Both will narrate their efforts or otherwise justify the Skill selection.

Both players then roll 4dF + Skill and modify through Aspect tags, invokes, and spin. Count the attacker's shifts and then reduce the shifts by the range between characters. The defender may reduce these shifts using Consequences. He may reduce the shifts by one by taking a mild Consequence, reduce by two by taking a moderate Consequence, or reduce by four by taking a severe Consequence. He may apply more than one Consequence if necessary. Each Consequence becomes a free-taggable Aspect on the character.

The remaining shifts are then used to mark the defender's Composure stress track: one box on the track is marked at the rank according to the shifts and all open boxes below it (one shift marks the first box, three shifts marks the first, second, and third box, and so on). If the highest box to be marked has already been filled, then the next highest available box is filled. If the box to be filled is past the end of the character's Composure stress track, then the character is Taken Out. The two players should negotiate what this means, mediated by the referee.

If the attacker fails his roll by three or more (gets three or more negative shifts), the defender gets spin.

The Composure attack represents an attempt to remove a character from play by making her ineffective. It might be an embarrassing anecdote designed to shame the character into removing herself from the scene, or a stinging insult that makes her too angry to act with the social subtlety necessary to participate in this kind of combat. Or something else.

\subsection{Obstruct}\label{sec:Obstruct}

The player obstructing begins by identifying the zone that will be obstructed. He then rolls 4dF + Skill - Range, modified by any Aspects tagged or invoked. If he obtains a positive result, he may apply the number of shifts as pass values on any borders in the zone. The total of all pass values added cannot exceed the number of shifts. So, if a player generated four shifts he could create a single pass value of four on one border, or a pass value of three on one border and one on another, or any other combination of pass values adding up to no more than four.

The obstruct action represents efforts to pin a character into his current mind-set or deflect him from ideas that would be contrary to the acting character's interests. This might be attempts at levity in order to block off a more sober zone, awkward geek behaviour in order to make it harder to get into an intimate zone, or similar.

\vfil

\subsection{Maneuver}\label{sec:Maneuver}

The player maneuvering begins by identifying the target of the maneuver. This target is typically a zone, but may be a character or the entire scene. He then announces the Aspect he intends to put on  the target and narrates the effort. He then rolls 4dF + Skill, modified by any Aspects tagged or invoked. If he obtains a positive result, the target acquires an Aspect described by the acting player. This Aspect is free-taggable once by any ally. Putting an Aspect of \aspect{Long-winded anecdote} on a zone will give other players a reason to avoid that zone, lest they be mired in a boring conversation, and unable to escape.

Permanent Aspects are Aspects that affect the person or zone directly. This includes things like \aspect{Liar}, \aspect{Out of crudit\'ees}, and so on. Transient Aspects are Aspects that derive from the continuous action of an individual. \aspect{Making socially unacceptable small talk}, \aspect{Spitting}, and so on. Transient Aspects last only until the placing character acts again, though he may use the Aspect in this last turn of its existence.

The caller determines whether a given \Aspect{} is permanent or transient.

% badbox here
% \newpage

\subsection{Move Another}\label{sec:Move Another}

The move another action is an attempt to force another character to move along the social map in a direction desired by the attacker. To begin, the acting player names the target of the attack. The attacker names his attacking Skill and the target names the Skill he will use to defend. Both will narrate their efforts to justify the Skill selection.

Both players then roll 4dF + Skill and modify through Aspect tags, invokes, and spin. Count the attacker's shifts and then reduce the shifts by the range between characters. These shifts are then used to move the defending player: one zone or pass value per shift, exactly as a move action.

If the attacker fails by three or more shifts, the defender is awarded a spin token.

The move another action is a careful effort to persuade. It represents effective rhetoric, brilliant argument, seduction, and like forms of persuasion. The acting character is trying to manipulate the target character directly.


\section{Damage}\label{sec:personal-combat-damage}

\index{consequences}
When a character has been hit by an attack that generates shifts, she may take damage, called \emph{stress}. Before marking the stress, she may reduce the shifts by applying one or more \Consequence{}s: a mild \Consequence{} reduces the number of shifts by one, a moderate \Consequence{} by two, and taking a severe \Consequence{} reduces the number of shifts by four.

After mitigation by \Consequence{}s, the remaining number of shifts indicate the box to be marked on the appropriate stress track. Mark this box and all boxes below it. If the highest box to be marked has already been marked, the damage ``rolls up'': mark the next higher open box and all below it.

A player may only ever have a maximum of three \Consequence{}s and may only have a maximum of one of each type regardless of the track the \Consequence{} was scored against. This means that a character suffering economic hardship (see Chapter 4) is easier to take out.

The defender determines the precise wording of the \Consequence{} (subject to reasonableness, as determined by table authority).

\subsection{Taken out}
\label{sec:personal-combat-taken-out}

A character is out of play when he sustains a hit past the end of any stress track. This means a person can be \TakenOut{} without ever taking a \Consequence{}, and therefore without ever taking any serious damage! A person that takes eight shifts of stress past the end of his \Health{} track cannot be saved. That's a one-shot kill\ldots or maybe there's a better way to narrate it?

The attacker narrates taking out his opponent (subject to reasonableness, as determined by table authority). Anything that suits the method (gunfire, punching, whatever) and that genuinely removes the character from play is suitable.

When narrating how an opponent is \TakenOut{}, it is essential to articulate how and if the opponent can return to the game. A ship that has been \TakenOut{} is no longer able to participate in space combat, but could, in theory, be boarded. In this case, the game could shift to the personal combat minigame. Or it could be destroyed, in which case it could not re-enter the game.

This gives a lot of power to the victor, and should be an incentive to players to offer concessions when things aren't going their way. A major opponent \TakenOut{} in personal combat can no longer fight, but the long-term repercussions are determined by the narrative. Being \TakenOut{} might also change features of a character sheet, though this requires some negotiation.

\subsection{Healing}
\label{sec:personal-combat-healing}

Characters cannot begin removing \Consequence{}s until the associated stress track has been cleared (and this is not instantaneous but rather dependent on the number of boxes and the associated \Skill{}).

\subsubsection{Recovering Stress Box Hits}

\rulebox{All \Health{} and \Composure{} stress hits are erased after a few days' relaxing downtime.}
Stress box hits are not real damage. They are the sweats, panic, scratches, ``only a flesh wound,'' and so on: nothing that can't be fixed with a tiny amount of downtime and nothing that actually affects performance. Consequently all \Health{} and \Composure{} stress track hits are cleared after a few days' relaxing downtime, whether that's a fancy hotel room with no one fighting in it or just the three days' travel time to the slipknot. The table should rule when enough time has passed or whether the downtime was sufficiently relaxing.

\subsubsection{Recovering Consequences}

Healing \Consequence{}s is governed, in the first instance, by an external time frame, which forces players to endure the effects of combat through the rest of the session.
\begin{itemize}
\item A mild \Consequence{} is cleared as soon as combat is over.
\item A moderate \Consequence{} remains until the end of the session.
\item A severe \Consequence{} must be carried through one complete session in which the associated stress track is never marked. If it is incurred during session one, it is gone no sooner than the end of session two, and if the associated stress track takes hit in a fight during that session, you'll need to hold the \Consequence{} through yet another one.
\end{itemize}

\subsubsection{Medics}

In addition to the purely mechanical process of recovery described above, there may be narrative reasons to introduce the need for actual medical help. The following guidelines are suggested, when pertinent. A mild \Consequence{} can be treated by a medic without a roll after the combat in which the wound was sustained is over. It requires a first-aid kit.

A moderate \Consequence{} remains until a medic can make a successful check against difficulty zero. Base time to heal is a week with (positive or negative) shifts modifying time to solve by one per shift. It requires a medical clinic (such as would be found on an ambulance or in a ship's sick bay), and the technology rating of the facility is applied as a modifier to the roll.

A severe \Consequence{} can be healed by a medic rolling against difficulty 4. It requires an advanced medical facility such as would be found in a hospital, and the technology rating of the facility is applied as a modifier to the roll. The referee may decide that the facility is, despite technology, better or worse equipped and apply this as a modifier to the difficulty. This takes one month, modified by the number of shifts achieved. In no case is the impact of the severe \Consequence{} removed before the end of the session following the one in which it was received. Example: getting a finger shot off



% \vfil