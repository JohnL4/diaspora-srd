\section{Spacecraft Design}\label{sec:Spacecraft Design} % \href{sec:id218}

\subsection{Assumptions}\label{sec:Assumptions} % \href{sec:id219}

These are the rules of the Diaspora universe as they pertain to space flight.

\subsubsection{Ship Size is not Interesting}

The proportion of a ship devoted to reaction mass and heat management
is vast compared to the crew quarters. Further, even basic automation
can do most of the job of running a ship. Finally, the nature of
reaction drives makes most variations in ship size inefficient --- there
is a sweet spot where you want your ship to be and there's not a lot
of point in making it bigger or smaller. So, we will derive ship size
as colour (maybe an Aspect!) from the stats of the desired ship rather
than derive the stats from the desired ship form. Crew sizes, however,
are pretty static: a dozen people or so make a full compliment, and
some ships are designed specially to be run by only three or four.

\subsubsection{Maneuver is a Resource}

The basic efficiency of a reaction motor is fixed by technology level and all
other Aspects of maneuver are based on how much reaction mass you want to blow
out the back. Consequently V-shift ratings are (relatively) fixed by technology
level.

\subsubsection{Slipstream Drives are Little}

If they are big then reaction drives make shipping cargo impossible --- you just
don't have room for cargo, reaction mass, and something else huge. Whatever
device controls slipstream entry, it does not significantly add to the payload
mass.

\subsection{Build Points}\label{sec:spacecraft-build-points} % \href{sec:id220}

\rulebox{bp = 5 + (6T)}

\begin{table}[ht]
\centering
\begin{tabular}{lr}
\toprule
Tech & build points \\
\midrule
T-2 & -7 \\
T-1 & -1 \\
T0 & 5 \\
T1 & 11 \\
T2 & 17 \\
T3 & 23 \\
T4 & 29 \\
\bottomrule
\end{tabular}
\caption{Spacecraft Build Points}
\label{tab:spacecraft-build-points}
\end{table}


\subsection{Statistics}\label{sec:Statistics} % \href{sec:id221}

Stats at or below zero indicate a component that cannot be used offensively.
All stats have a maximum value of technology rating plus two.

All stats start at value zero and can be increase by 1bp/stat point up
to stat point=T, and by 2bp/stat point up to the cap. Thus, rank cost
by technology can be found on the following table:

\begin{table}[ht]
\centering
\begin{tabular}{*{7}c}
\toprule
rank & T-1 & T0 & T1 & T2 & T3 & T4 \\
\midrule
1 & 2 & 2 & 1 & 1 & 1 & 1 \\
2 & ./. & 4 & 3 & 2 & 2 & 2 \\
3 & ./. & ./. & 5 & 4 & 3 & 3 \\
4 & ./. & ./. & ./. & 6 & 5 & 4 \\
5 & ./. & ./. & ./. & ./. & 7 & 6 \\
6 & ./. & ./. & ./. & ./. & ./. & 8 \\
\bottomrule
\end{tabular}
\caption{Technology Rank Cost}
\label{tab:technology-rank-cost}
\end{table}



The statistics are:
\begin{description}
\item[V-shift] influences positioning roll and heat accumulation
\item[Beam] attacks offensively and defensively against Torpedo attacks,
possibly accumulating heat
\item[Torpedo] attacks only offensively, no heat, has the \emph{Out of ammo} Aspect
\item[EW] attacks offensively if the crew is trained for offensive EW and is
reflective in that offensive and defensive rolls are compared and the
loser takes the shifts in damage
\item[Trade] influences monthly maintenance rolls
\end{description}

A ship which doesn't invest points in a statistic possesses a default
value of zero.

\subsection{Stress Tracks}\label{sec:spacecraft-stress-tracks} % \href{sec:id222}

Stress tracks start at three boxes and cost 1bp to increase or give
back 1bp to decrease except the Heat track, which costs 2bp to
increase gives back 1bp per decrease.
\begin{description}
\item[Hull track] a measure of structural integrity
\item[Data track] a measure of network and comms integrity
\item[Heat track] a measure of heat sink capacity
\end{description}

\subsection{Stunts}\label{sec:spacecraft-stunts} % \href{sec:id223}

\subsubsection{Common Stunts}\label{sec:spacecraft-common-stunts}

\begin{description}
\item[Skeleton Crew]
ship can be crewed with a crew the size of the table (2-4 individuals), as
long as Pilot, and Navigation are represented in the available Skill sets. The
ship may not have any crew-related Aspects (``Boarding party,'' say). In combat,
there is no default Skill ability available, and similarly there is no
opportunity for role-playing scenarios with traitorous crew members aboard.
1bp.
\item[Attacks a different track]
applies to a single weapon system. This weapon system now attacks a track
other than its usual target track (a microwave cannon that attacks the heat
track instead of the Frame stress track, an ECM drone missile system that
attacks the Data stress track instead of the Frame stress track, etc.) 4 bp.
\item[Attacks an additional track]
applies to a single weapon system. This weapon system now attacks a track
other than its usual target track in addition to its usual target track (a
fusion cannon that attacks the Data stress track and the Frame stress track, a
thermite missile that attacks the Frame and the Heat track, etc.) 8 bp.  High
capacity magazine: Torpedo does not get the automatic \emph{Out of ammo} Aspect. 1
bp.
\item[Overwatch]
may fire its Beam defensively against missiles targeting any ship it is
tethered with, as many times as opportunity arises in the phase. Each counts
against the sum of beam fire for heat accounting. 2 bp.
\item[Automated Defense]
an automated defense system can be installed for any specific offensive roll.
This gives a rank 2 defense with no offensive capability, even for reflective
rolls like EW. It is never modified by character Skill in any way. 1bp each.
\item[Vector randomizer]
Defense 2 versus Beam
\item[Firewall]
Defense 2 versus EW
\item[Point Defense]
Defense 2 versus Torpedoes
\item[Civilian]
This ship is designed for private ownership. It comes with registration papers
for all systems on board and is perfectly legal to own and operate without
special licensing. More importantly its drives come with all regulated safety
interlockings, restricting its power output and its thrust. Piloting a ship
that is not Civilian is much trickier and requires Military-grade Pilot. 3bp
\item[Cheap]
This ship is constructed from old technology or ill fitted
parts. It might be a factory second, failing quality control and
listed for resale in systems with looser restrictions. Cheap ships
cost 2bp.
\item[Interface Vehicle]
carries its own interface vehicle, capable of landing safely on a planet with
surface gravity equal to ship technology and take off again. 1bp
\item[Extended range]
carries vastly more r-mass than is efficient, removing its ability to
over-burn drives to reduce travel time but increasing its maximum travel
duration by a factor of 4. Extended range ships are huge and cumbersome and
cost 2bp.
\end{description}

\subsubsection{Technology-Restricted Stunts}\label{sec:spacecraft-technology-restricted-stunts}
\begin{description}
\item[T2: Slipstream]
This ship is capable of using slipknots to travel between star systems. 1~bp.

\item[T3: Slipstream]
This ship is capable of more sophisticated slipstream travel. 1~bp.

\item[T4: Dimensional heat dump]
This vessel dumps its heat into another dimension. It does not have a \Heat\ track, and cannot take \Consequences\ from a heat-based attack. 4~bp.
\end{description}

\subsection{Aspects}\label{sec:spacecraft-aspects} % \href{sec:id224}

All ships have five Aspects. Determine the required Aspects and then
fill the remaining slots with Aspects that suit the ship's purpose and
history.
\begin{description}
\item[Huge]
if the Trade value is greater than or equal to twice the technology rating,
the ship is Huge. If the V-shift rating is equal to or greater than twice the
technology rating, then the ship is Huge (implies large amounts of reaction
mass).
\item[Cargo hauler]
if the Trade value is 3 or more, the designer may choose this Aspect or
``Passenger liner.''
\item[Passenger liner]
if the Trade value is 3 or more, the designer may choose this Aspect or ``Cargo
hauler.''
\item[Falling apart]
if a ship has the Cheap Stunt, then it also gets one Aspect that is largely
negative, represented by the ``Falling apart'' Aspect. The table should feel free
to rename the Aspect to reflect the specifics of the ship's cheapness.
\end{description}

\subsection{Fighters (optional)}\label{sec:spacecraft-fighters} % \href{sec:id225}

Given the use of reaction drives in Diaspora, effective single-pilot
fighters are technologically implausible. Still, they can be fun, and
vessels that carry small amounts of reaction mass would be capable of
extremely high acceleration for a short while.

\subsubsection{Decoys}

One thing fighters are extremely powerful for is decoying enemy ships. Because
they generally have bery high maneuver drives and commensurably high pilot
Skills, they represent a great opportunity to drag an injured opponent back to
the middle of the map where he can't escape (harrying!) or push him off the
4-line --- decoying him so far and fast away from his target that he can't return
to the fight. Take advantage of the fact that your opponent probably won't
target your little fighter when he can concentrate on killing your Corvette.

Fighters are small ships to aid in combat, but which may not enter
slipstreams. Any spaceship which buys the Stunt ``Carries Fighters''
(2bp) may reduce its Trade value by 1 for each fighter carried, to a
minimum of zero: this decrease affects all rolls involving the Trade
value, as the small fighter occupies space that could be used for
cargo or other means of economic viability. Fighters may be launched
in any combat phase that the parent ship chooses not to act when it
otherwise could do so. If the parent ship chooses the Beam combat
phase, the Beams may not subsequently be used defensively, however.

All fighters are military vessels, and pilots require Gunnery and
Military-grade Pilot Skills. Fighters are designed by buying Beam,
Torpedo, and V-shift values (as per ship design) and all tracks have
two boxes. They may buy ``High-capacity magazine'' and ``Automated
defense'' Stunts, but no others. Fighters have one aspect, but no fate
points, and any fate points spent come from the parent ship (or the
pilot, if a PC); similarly, any consequences taken are given to the
parent ship.

At T2 fighters are built with 6 bp; at T3 fighters are built with 8 bp (2+2T).

