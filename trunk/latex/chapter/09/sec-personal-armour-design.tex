\section[Personal Armour Design]{Armour Design}\label{sec:personal-armour-design} % \href{sec:id230}

Armour uses the build point equation $bp = 2T + 9$ (see \autoref{tab:armour-build}).

\begin{table}[ht]
\centering
\begin{tabular}{lc}
\toprule
Tech	& BP \\
\midrule
T-4	& 1 \\
T-3	& 3 \\
T-2	& 5 \\
T-1	& 7 \\
T0	& 9 \\
T1	& 11 \\
T2	& 13 \\
T3	& 15 \\
T4	& 17 \\
\bottomrule
\end{tabular}
\caption{Personal armour build points per tech level}
\label{tab:armour-build}
\end{table}

\subsection{Statistics}
\label{sec:personal-armour-statistics}

\begin{table}\centering
\begin{tabular}{cc}
\toprule
BP	& Defense \\
\midrule
0 & 0 \\
1 & 1 \\
2 & 2 \\
3 & 3 \\
5 & 4 \\
7 & 5 \\
9 & 6 \\
13 & 7 \\
17 & 8 \\
\bottomrule
\end{tabular}
\caption{Armour defense build costs}
\label{tab:armour-defense-costs}
\end{table}

\stat{Defense} is subtracted from all attacks on the wearer. The build point cost to increase \stat{Defense} goes up as the stat goes up --- see \autoref{tab:armour-defense-costs}.

\stat{Agility Mod} is added to the wearer's \skill{Agility} Skill rolls. \stat{Agility Mod} starts at the base score of -2, and cannot be increased with build points directly. The \stunt{Flexible} Stunt increases \stat{Agility Mod} one point, and the \stunt{Lightweight} Stunt increases it another. Agility can only be increased above zero with the \stunt{Power Suit} Stunt.

% \subsection[Personal Armour Stunts]{Stunts}
% \label{sec:personal-armour-stunts}
%%% This should be identical to armour-stunts, plus costs.
\definecolor{DTLoddrow}{rgb}{0.9,0.9,0.9}
\begin{table*}[hp]
\iflandscape
{\begin{tabular}{p{0.2\textwidth}ccp{0.65\textwidth}}}
{\begin{tabular}{p{0.15\textwidth}ccp{0.61\textwidth}}}
\toprule
Stunt		& BP	& Tech	& Description %
\DTLforeach{armour-stunts}{%
\DTLname=name,\DTLbp=bp,\DTLtl=tl,\DTLdesc=desc}%
{%
\DTLiffirstrow{\\\midrule}{\\}%
\DTLifoddrow{\rowcolor{DTLoddrow}}{}%
\DTLname	& \DTLbp	& \DTLtl	& \DTLdesc %
}%
\\\bottomrule
\end{tabular}
\caption{Personal armour stunt costs}
\label{tab:personal-armour-stunt-costs}
\end{table*}


\subsection[Personal Armour Aspects]{Aspects}
\label{sec:personal-armour-aspects}

Armour with a \stat{Defense} value that cost more than (4+T) and is not \stunt{Lightweight} gets the \Aspect, \aspect{Very heavy}, which referees should happily compel to ruin roads, damage bridges, and get the authorities mad. It might reasonably be compelled against \skill{Stealth} checks and so forth too.

Armour with the \Stunt\ \stunt{Power Suit} also gets the \Aspect, \aspect{Out of Juice}, which one might compel to restrict actions in order to conserve energy.

Armour with a \stat{Defense} value higher than tech level that has the \Stunt\ \stunt{Civilian} and does not have the Stunt \stunt{Lightweight} also gets the \Aspect\ \aspect{Industrial Equipment}.

\iflandscape{\vfil}{}

\subsection{Cost}
\label{sec:personal-armour-cost}

The base cost for armour is 2 for Civilian and 3 for non-Civilian.
Armour with the \stunt{Power Suit} stunt costs 4 regardless of its Civilian
status.

% Orginal Material
%
% You can download a list of armour as part of the \href{http://www.vsca.ca/Diaspora/Diaspora%20tables.pdf}{essential tables}
% available from the Diaspora Website.

\vfil