\section{Special Maneuvers}
\label{sec:special-maneuvers}

\subsection{Formation Flying}
\label{sec:formation-flying}

Formation flying is a means of keeping two ships in the same range band at all times. Ships in formation may not be separated by the positioning rolls of another ship. This allows a merchant to fly with an escort, for example, or a fleet of fighters to maintain a common range for their attacks.

Ships may begin combat in formation. During set-up in the detection phase, any two ships (or more) in the same band may (but need not) be in formation, if the players controlling the ships so choose. Models are pressed next to each other to represent this.

Ships may enter formation with one another during the positioning phase. In the turn in which any ship is moved to band 0, and there is at least one other ship at band 0, the ship entering the band may enter formation with another ship.

Each pilot makes a roll, but the formation moves based on the lowest roll. If repositioned by another ship, the formation is moved as a unit. Ships in formation may only move as fast as the V-shift of the slowest ship allows.

A ship may leave formation at any time during the positioning phase. Formation flying allows all ships autonomy, but is more challenging to maintain than tethering in combat situations.

\subsection{Tethering}
\label{sec:tethering}

Tethering offers increased performance for ships in formation, at the expense of some autonomy for vessels. Tethering need not be physical, and any viable picture may be used to describe it: slaving the computer, perhaps; tethering could also be useful for slipping (``Convoy''). Two (or more) ships in formation may be said to be tethered, if one of the two following conditions are met: either both ships agree to be tethered and one agrees to lead, or one ship wishes to tether and lead and the other has been \TakenOut\ with a compatible narrated result.

There is always a primary ship when ships are tethered; one leads the other (or others). Multiple ships may be said to be tethered together, but only one can be leading. Only fate points from the lead ship can be spent while ships are tethered. As with formation flying, models are pressed together, but tethered ships gain the temporary Aspect \aspect{Tethered}. Tethered ships may not fire on other ships within their formation. They may be disengaged at any point, but only at the discretion of the leader. In the positioning phase, only the leader makes a piloting roll. Tethered ships may only move as fast as the \Vshift\ of the slowest ship, and may not initiate a burn.

\subsection{Boarding}
\label{sec:boarding}

In any turn, individuals from a lead ship may board any ship to which it is tethered. At this point, the game would normally revert to the individual tactical game: characters fight boarders! See Personal Combat (\autoref{sec:personal-combat}).

However, boarding can be addressed within the space combat game with an opposed roll, where any positive result for the boarders indicates the boarding action to be successful by the end of the following turn. Relevant \Aspects\ include \aspect{Boarding crew}, \aspect{Bunch of thugs}, \aspect{Tight security}, or \aspect{Elite marines}, but not \aspect{Tethered}. Ties favour the defending ship, however, and any ship that withstands a boarding party for three turns has repelled the boarders, and is no longer tethered. If it was tethered as a result of being \TakenOut, it remains \TakenOut, but requires new narration, this time provided by the defender since he succeeded in repelling boarders. These rules could also become applicable if the players stumble onto a boarding situation, or are asked to escort a target ship that is then attacked by pirates: the pirates board the target, while the characters in their ship maneuver about.

\subsection{Coupling}
\label{sec:coupling}

All ships have a nose coupling mechanism which may be attached to the base of the mast of any one other ship and can be used to tow (or, actually, push) the other ship. The coupled ship must be \TakenOut\ or tethered, and need not have a working drive.

Two coupled ships move at the lead ship's \Vshift\ rating -2. The lead ship gains the temporary Aspect \aspect{Slow to respond}, and the coupled ship's counter is removed from the map: the coupled ship may not fire weapons or take any other actions in the space combat game until it is decoupled. Ships couple or decouple during the positioning phase as an action. If ships decouple, the lead ship loses its temporary Aspect and regains full use of its \Vshift, and a counter is placed on the current range band to represent the decoupled ship. Derelicts are not placed on the map.

