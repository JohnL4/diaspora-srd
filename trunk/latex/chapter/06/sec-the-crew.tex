\section{The Crew}
\label{sec:the-crew}

Diaspora assumes that spacecraft have a fully functional crew aboard, who draw a salary and are able to man their stations competently.  There is no need to flesh them out unless there are role-playing reasons to do so, and player characters can work beside an undifferentiated crew happily.

Except as noted below, all combat crew positions on a ship are assumed to be staffed by someone with a \Skill\ level of 2.  PCs serving aboard such ships may use their individual \Skill\ levels, but if they choose not to (e.g. if they ride as passengers), there is always someone who can do the job.

A ship's \Trade\ value is not used in combat, and therefore there is no default broker aboard to assist with maintenance rolls.

The exception to this is a spacecraft that has the \stunt{Skeleton Crew} \Stunt, in which case all the jobs in combat must be taken by a known individual (either a player character or an NPC who has been developed) who is trained in the relevant \Skill. In particular, \skill{Communications} and \skill{Gunnery} stations, if they have a positive value, may not be operated by an untrained individual.

For each crew position, there is only ever one person doing a given job at a given time. One Navigator rolls in the detection phase, and only one Computer expert rolls to repair the \Data\ track.

A PC may occupy more than one position on the ship, but it becomes challenging during combat. Each \Skill\ associated with a combat phase normally requires a single crew member to staff it.

\rulebox{Staffing more than one crew position during combat earns a -1
cumulative penalty to the effective Skill level.}
For example, a gunner may fire beams offensively and defensively without penalty, but would receive a penalty on the torpedo roll if he has fired beams.

\rulebox{Each crew member may only act once per phase in combat.}
For example, a single gunner may not fire beams defensively and launch torpedoes in the same phase.

