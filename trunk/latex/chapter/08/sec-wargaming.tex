\section[Wargaming]{Platoon Combat Wargaming}
\label{sec:platoon-combat-wargaming}

\subsection{Creating Scenarios}
\label{sec:platoon-combat-creating-scenarios}

Units of the same technology level should be roughly balanced unless you deliberately take sink Stunts or Skills.

When constructing scenarios you will want a way to weigh the effectiveness of different units. This system has been designed so that a unit is roughly equal to any other unit, the exception being leader units, which are the equivalent of two normal units. Units of differing technology levels are different in power but not hugely so --- they differ by a small number of Stunts. Depending on Stunts taken, a good rule of thumb is that each difference in technology is about 25\% more unit power.

Within the context of the RPG, scenarios will often be unbalanced, with the players struggling to make do given the odds as determined by the situation in which they have found themselves.

As a stand-alone wargame, though, some approximations of balance are possible. Nevertheless, we recommend that, above all, you run a scenario that's interesting. Create units and organizations with a goal and an engagement with a purpose and it doesn't matter a ton who wins it --- the simulation will be fun either way. It might even be the case that victory conditions can be created or modified as the game progresses, or even at the end. The victor is the side that achieves its goal. Revisit victory frequently during the game --- the goal may change! An ambush might be won and become a pursuit which is lost. A breakthrough that starts going especially well for the runners might turn into a meeting engagement, or even an opportunity to secure objectives.

As a wargame, progress can even be determined by an arbitrary time limit. Set a number of turns after which victory will be evaluated. At this time, the game is over. If you want to continue (perhaps with new objectives!), set a new timer and perhaps allow some reinforcements in. Six to ten turns should be plenty.

\subsection{Purpose}
\label{sec:platoon-combat-purpose}

Engagements come in a limited number of forms. By all means, dream up your own scenario from whole cloth, but you may find these categories useful for setting the scene.

\begin{description}
\item[Meeting Engagement]
Roughly equal forces meet by accident or intent on the field and engage to destroy. Armies start on opposite sides of a map.

\item[Pursuit]
A small force is pursued by an equal or superior force. The small force wishes to disengage. The large force wishes to destroy. Armies start with pursuing force on one edge and the pursued force near the middle-ish. You want a long map for this. The more units the pursued force gets off the map, the better.

\item[Breakthrough]
A dug-in force attempts to keep another force from disengaging. The disengaging force needs to get most of its units through or get special units through. Start with the blockade force setting up in the middle of the map and the runners on one edge. They must exit the opposite edge with their objective units intact.

\item[Ambush]
A strong force has entered the region to patrol and is surprised by an inferior but entrenched force. Set up as Breakthrough but balance units differently (patrol is stronger), and give defensive advantages (Aspects on zones or units). The ambushers should win by inflicting lots of harm and then escaping. The patrol wins by surviving the ambush and destroying the ambushers.

\item[Objectives]
Designate a number of objectives on the map --- perhaps towns or hilltops. Start with equal forces on opposite edges of the map. Win by having units on most objectives at the end. You can mix this with many of the previous scenario designs (especially Meeting Engagement and Ambush).
\end{description}

\subsection{Tactics}\label{sec:platoon-combat-tactics}

One of the chief sources of tactical pressure, aside from the obvious one of move and fire, is the choice of unit action order. The actor must elect to act with specific units in an order he prefers, but counter-activity or action results may have cascading effects on remaining unit action.

\subsection{Simulation through Aspects}\label{sec:simulation-through-aspects}

Smoke? A maneuver to place the Aspect \aspect{Smokey} on a zone. Cratering charges? A maneuver to place the Aspect \aspect{Cratered} on a zone. Time sensitivity is often sufficiently modeled by the free-tag mechanism --- smoke is only especially effective the first time it's tagged; thereafter it's another source of advantage at normal cost. Another simulation effect that Aspects can create is the idea of a forced move. If you want an enemy unit out of a zone, there is no mechanism that forces him to do so. This is because we want to retain as much player autonomy as possible: everything should be a choice. So make him choose: with as many units as you can muster, use maneuvers to pile free-taggable Aspects on his zone. Now he has to choose between leaving the zone and sitting in the same place as a potentially massive free bonus to an attack roll.

\subsection{Balance Decisions}\label{sec:balance-decisions}

These balancing ideas roughly follow the rock-paper-scissors principle, which is basically the idea that every unit type should have some important function and should be defeatable by some other unit type. No single unit type should dominate: if you go all armour, infantry will kick your ass. If you go all artillery, aircraft will cut you up. And so on.

The balancing relationship between unit types is as follows:

\begin{description}
\item[Infantry]
is intended to be flexible, capable of adequately subsuming multiple roles. An observation unit can also be a credible anti-air threat and be adept at re-hiding. An assault unit can also be a credible defensive unit.

\item[Armour]
is intended to be dedicated and reactive. They do a couple of things extremely well and will generally suck up one Skill slot for \skill{Movement} just to take advantage of their intrinsically high speed limit.

\item[Artillery]
is intended to do lots of harm but be vulnerable to reactive attacks from other artillery. It is also intentional that aircraft are the bane of artillery in all cases. A configuration that includes artillery has to consider the possibility of enemy air power and dedicate some counter-measure or they will get their artillery cleared.

\item[Aircraft]
are intended to do periodic but precise and devastating harm. They are highly vulnerable while on the map (always spotted, always in line of sight), again by design, but also extremely precise in their deployment while there (range 1 to any target they choose to kill). Countering aircraft requires deployment of specific equipment: either other aircraft or AA capable units. Putting a little AA capacity in all infantry is a good buy.
\end{description}

