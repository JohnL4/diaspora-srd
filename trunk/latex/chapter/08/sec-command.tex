\section{Command}\label{sec:Command}

\index{command range}
A platoon is a grouping of units that can be ordered to act by a leadership unit. By default that unit must be in the same zone as the leader, but this may be modified by Stunts. The maximum distance allowed between a unit and its leader in order to maintain membership in the platoon is the unit's command range.
%TODO glossary: command range

\rulebox{Command range indicates how may zones a unit can be away from its leader and still be in communication with it.}

\index{OOC|see{Out Of Communication}}\index{Out Of Communication (OOC)}
Command range is zero (same zone only) unless modified by a stunt on the leader, the unit, or both. Non-leader units that do not belong to a platoon (are out of command range) may move away from enemy units, may attack enemy units that have fired upon them, and may attempt to unjam and remove Out Of Communication (\OOC) counters on themselves, but may take no other actions. They have no fate points and do not share a platoon's Consequences. Hits on these units may not be mitigated by a platoon taking a \Consequence. Units that become disassociated from their platoon do not change the platoon's fate point total.

Platoon membership is checked at the beginning of the platoon's action. Units with \OOC{} counters are only part of a platoon membership when in the same zone as the leader unit.

A leader unit with \OOC{} counters disconnects all its platoon members but suffers none of the other restrictions described here.

