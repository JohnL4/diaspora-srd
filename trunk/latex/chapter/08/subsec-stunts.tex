% \subsection{Unit Stunts} % should be?
\subsection{Stunts}\label{sec:unit-stunts}
\begin{description}
\item[Cavalry]
this unit is undersized and overpowered, so its maximum move is increased by one (infantry, armour, or artillery only). Infantry units may take this \Stunt\ multiple times: the unit may be thought to have an intrinsic vehicle for mobility. When taken by an armour unit, the \Stunt\ is designated ``\stunt{Light}.''

\item[Special forces]
this unit is not automatically spotted when it shares a zone with an enemy unit (infantry only).

\item[Wireless]
unit is attached to an integrated communications net, increasing command range by 1. May be taken multiple times.

\item[Engineer]
may use a successful maneuver roll to use up the free-tag on an enemy-applied \Aspect\ or make two maneuver rolls on the zone it is in instead of the usual one (infantry and armour only).

\item[Guerrilla tactics]
attacks from this unit never generate spin for the defender (infantry only).

\item[Highly trained]
this unit has one additional morale box.

\item[Infantry carrier]
this unit can carry infantry (armour or aircraft only). One infantry unit in the zone can move with this carrying unit (including traversing the Re-arm track for aircraft). The infantry unit cannot act this turn (before or after the move). The unit can begin the game carrying its infantry load. For aircraft, when the aircraft re-enters the map, the infantry is deployed and may act normally; the aircraft may not otherwise act while deploying infantry. Carried infantry do not have to be in the same platoon as the carrier.

\item[Interceptor]
if this unit is on the LAUNCH! box, it may enter the map any time an enemy aircraft enters the map and act immediately before the target aircraft can act. It may act only against this target aircraft (aircraft only).

\item[Irregulars]
this unit is an irregular non-professional unit (a sink \Stunt, chosen only to model a unit that is less effective than other units of the same technology level). Other sink \Stunts\ can be invented to fit the scenario: \stunt{Slow}, to represent a low rate of fire, etc.

\item[Long range]
ignores one zone for attack roll range modification. May be taken multiple times.

\item[Orbital]
this unit can only be attacked by fire from other orbital units (artillery only). Orbital units that are attacked with the jam action, however, take damage as though attacked with weapons (in addition to the effects of jamming).

\item[Prepared positions]
this unit was set up long before the battle (artillery only). Before combat begins, it may add a the \Aspect\ of \aspect{Locked in} to any two zones on the map. This \Aspect\ can be free-tagged by any allied artillery unit, and remains an \Aspect\ on the zone which may be tagged normally thereafter.

\item[Scatterable mines payload]
this unit can deliver area-denial ordnance (\iflandscape{i.e., }{}mines). Pass values placed by the unit from an interdiction strike are permanent (artillery and aircraft only).

\item[Scout]
this unit can continue movement after entering a zone containing enemy units (infantry and armour only).

\item[Skill substitution]
With an appropriate narrative, additional \Stunts\ may be designed to allow \Skill\ substitutions. Each unit may only ever have one \stunt{Skill substitution} \Stunt. The following are offered as representative examples.

\begin{description}
\item[Agile]
can use \skill{Movement} in place of \skill{Armour} (armour only).

\item[Graphite payload]
this unit can deliver payloads designed to interrupt electrical and electromagnetic function (artillery and aircraft only).  It may use its \skill{Indirect Fire} Skill to effect Jam attacks (which would normally use the \skill{Signals}). Note that this can be combined with \stunt{Zone Effects} to jam all units in a zone (regardless of owner).

\item[Shoot and scoot]
this weapon system is designed to be fired while on the move or to move very soon after firing a mission. It may use its \skill{Movement} Skill instead of \skill{Camouflage} (artillery only).

\item[Technology enhancement]
increase any \Skill\ by one. This \Stunt\ may be taken at most twice per \Skill, for a total bonus of +2.

\item[Stealth technology]
designed to hide, this unit can use \skill{Camouflage} in place of \skill{Armour} (armour only).
\end{description}

\item[VTOL]
this unit is designed to stay on target --- once on the map it may remain, moving a maximum of 1 zone (its free move) per turn (aircraft only).

\item[Zone effects]
this unit may attack all units in the target zone with one roll at -2 (armour, artillery, or aircraft only). Units do not need to be spotted to be attacked in this fashion.
\end{description}

\subsubsection{Leadership stunts}

Each platoon leader additionally chooses one of the following four Stunts.

\begin{description}
\item [Battlefield genius]
units can be one zone further from the Leader than otherwise allowed.

\item[Logistics genius]
units in platoon do not have the \aspect{Out of ammo} Aspect.

\item[Tactical genius]
units in platoon ignore one extra zone of range when attacking.

\item[Not a genius]
sink Stunt for crap commanders.
\end{description}

