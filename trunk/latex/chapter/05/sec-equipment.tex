\section{Combat Equipment}
\label{sec:personal-combat-equipment}

Most equipment needs little more than a list of their statistics, and so at the end of this chapter there will be a table of all the equipment tidily presented in a way that's easy to use during actual play. Each can benefit from a little cluster-specific story, though, so feel free to write that. Give the weapons technical sounding names and military designations. Give the armour manufacturer's names and model numbers. Make them yours.

%%% Table of weapon stunts
\iflandscape
{\begin{table*}[hp]}
{\begin{sidewaystable*}[hp]}
\centering
\iflandscape
{\begin{tabular}{l lll p{\columnwidth}}\toprule}
{\begin{tabular}{l lll p{0.6\columnwidth}}\toprule}
Stunt		& CC & ST & EW		& Description\\
\midrule
Awkward reload	&     & Yes &		&
\aspect{Out of ammo} Aspect is free-taggable (or free compel) after regular fire and not just area of effect fire. \\
Civilian	& Yes & Yes & Yes	&
weapon may be used without a \stunt{Military-grade} stunt. \\
Dispersed fire	&     &     & Yes &
weapon can fire as an Area of Effect weapon, applying its offensive roll to each target in a zone. \\
Explosive	& Yes & Yes & Yes	&
weapon can fire as an Area of Effect weapon, applying its offensive roll to each target in a zone (including the firer, if appropriate). \\
Free modal	& Yes & Yes & Yes	&
as Modal, but is set automatically rather than as an action. \\
Full auto	&     & Yes & 		&
weapon can fire as an Area of Effect weapon, applying its offensive roll to each target in a zone. AoE effect cannot be used in the same zone as the firer. After firing on full auto, the firer's \aspect{Out of ammo} Aspect is free-taggable. \\
High capacity	&     & Yes & 		&
\aspect{Out of ammo} can not be free-tagged. \\
High recoil	&     & Yes & Yes	&
weapon can only be fired every other round unless the firer is prone. \\
Integral	& Yes &     & 	&
this weapon is built in to the wielder. \\
Low recoil	&     & Yes & Yes	&
weapon can be fired without penalty in low gravity. \\
Modal		& Yes & Yes & Yes	&
this weapon has multiple modes that can be selected with a combat action. \\
Non lethal	& Yes & Yes & Yes	&
weapon can only be used for \Composure{} attacks. \\
Stealthy	& Yes &     & 	&
weapon does not appear to be a weapon outside of combat. \\
Thrown	& Yes &     & 	&
This weapon may only be thrown, using the \skill{Agility} Skill,  at range 1-2. Normal penalties for exceeding this range (-2 per band) apply. Weapon has the \aspect{Out of Ammo} Aspect, which may be compelled. \\
Undetectable	&     & Yes & Yes	&
any Skill check made to detect this weapon is made at -2. \\
Versatile	& Yes &     & 	&
This weapon may be thrown, using the \skill{Agility} Skill, at range 1-2. Normal penalties for exceeding this range (-2 per band) apply. The weapon may only be re-used if the character goes and spends an action picking it up from the target zone. \\
\bottomrule
\end{tabular}
\caption{Personal combat weapon stunts}
\label{tab:personal-weapon-stunts}
\iflandscape{\end{table*}}{\end{sidewaystable*}}


\subsection{Weapons}

The statistics of weapons are:


\begin{center}
\begin{tabular}{lp{0.6\columnwidth}}
\toprule
Harm		& modifier to offensive roll \\
Penetration	& negative modifier to armour \stat{Defense} value \\
Minimum range	& range below which a penalty is applied to offense roll \\
Maximum range	& range beyond which a penalty is applied to offense roll \\
Cost		& the target number for \skill{Wealth} rolls to acquire the weapon \\
\bottomrule
\end{tabular}
\end{center}


With the exception of \skill{Brawling} weapons, weapons may have Stunts and Aspects. These are listed in \autoref{tab:personal-weapon-stunts}. Weapons not designated as \stunt{Civilian} can only be employed by characters with the appropriate Military-grade Stunt. That is, non-civilian slug throwers require the \stunt{Military-grade Slug Throwers} Stunt, and so forth.

Some weapons have Aspects. The weapon becomes a new scope of Aspects that can be tagged in addition to the usual ones on friends, foes, and places.

Weapons break down into four categories, each represented by a Skill of the same name:
% \begin{itemize}
% \item Brawling
% \item Close Combat (CC)
% \item Slug Thrower (ST)
% \item Energy Weapon (EW)
% \end{itemize}

\subsubsection{Brawling}
\label{sec:brawling}

Brawling involves the nitty gritty fighting with fists, found weapons, clubs, and knives. Brawling weapons do not have Aspects or Stunts.

\subsubsection{Close Combat}
\label{sec:personal-combat-close-combat}

The \skill{Close Combat} Skill covers all melee weapons, all of which are civilian. If a player wishes his character to own such a weapon it should simply be granted. There is no obvious reason to differentiate or restrict. The \skill{Assets} check to acquire a blade anywhere is 1. The Broadsword includes any long two-handed blade, including battle axe. While not a common weapon, where technology and industry have fallen behind, these are the mainstay of the heavy infantry.

Any long stick with a pointy end is a spear. These stats model all pole-arms. Wherever technology has fallen, these cheap and effective weapons will be common. In some cultures spears might also be found as part of a spacecraft's defensive equipment --- a long pointy weapon could be particularly effective in the narrow confines of a ship.

Some \skill{Close Combat} weapons are designed to be thrown. These have the \stunt{Thrown} Stunt and they get to be re-used indefinitely, as with a firearm or laser. They get the \aspect{Out of ammo} Aspect to model this.

\subsubsection{Slug Throwers}
\label{sec:personal-combat-slug-throwers}

The basic cost for a slug thrower is 3, modified by the difference between the weapon technology and the technology of the system in which it is purchased. Thus a T2 combat rifle requires an \skill{Assets} check of 3 in a T2 system, but 5 in a T0 system and only 1 in a T4 system. \stunt{Civilian} weapons are one level cheaper.

Some slug throwers automatically have certain aspects:

\begin{center}
\begin{tabular}{lp{0.59\columnwidth}}
\toprule
Out of ammo	& automatic on anyone firing a slug thrower. Freely taggable after a Full Auto area of effect attack. \\
Military-grade	& automatic on any weapon with\-out the \stunt{Civilian} Stunt. \\
Concealed	& automatic on anyone with a weapon with minimum range 0\\
\bottomrule
\end{tabular}
\end{center}

\subsubsection{Energy Weapons}
\label{sec:personal-combat-energy-weapons}

% %%% This should be identical to armour-stunt-costs, minus costs.
% \begin{table}[t]
% \centering
% \begin{tabular}{p{0.2\columnwidth}p{0.65\columnwidth}}\toprule
% Stunt		& Description\\
% \midrule
% Civilian	&
% May be used without a \stunt{Military-grade} stunt. \\
% Flexible	&
% Easily shifts with the wearer, allowing greater mobility. \\
% Lightweight	&
% This armour is made of a lightweight material. \\
% Pressurized	&
% Acts as a pressure suit, carrying its own supply of oxygen and power for heat and communication. \\
% Power Suit	&
% This armour is powered. \\
% Servos	&
% Enhanced mobility. \\
% Sensors	&
% Equipped with enhanced sensor equipment. \\
% Armoured \newline ~ penetrators	&
% Armour is optimized for punching through other armour. \\
% Crushing fists	&
% Armour is optimized for punching through people. \\
% Long range	&
% Extensive power and environmental resources: does not have the \aspect{Out of juice} transfer aspect. \\
% Jump jets	&
% limited flight capability: wearer gains +2 to \skill{Agility} checks for the purpose of movement, with no maximum movement rate. \\
% \bottomrule
% \end{tabular}
% \caption{Personal combat armour stunts}
% \label{tab:personal-armour-stunts}
% \end{table}

%%% This should be identical to armour-stunts, plus costs.
\definecolor{DTLoddrow}{rgb}{0.9,0.9,0.9}
\begin{table*}[hp]
% \iflandscape
\begin{tabular}{p{0.2\columnwidth}p{0.65\columnwidth}}
% {\begin{tabular}{p{0.15\textwidth}ccp{0.61\textwidth}}}
\toprule
Stunt		& Description %
\DTLforeach{armour-stunts}{%
\DTLname=name,\DTLdesc=desc}%
{%
\DTLiffirstrow{\\\midrule}{\\}%
\DTLifoddrow{\rowcolor{DTLoddrow}}{}%
\DTLname	& \DTLdesc %
}%
\\\bottomrule
\end{tabular}
\caption{Personal armour stunts}
\label{tab:personal-armour-stunts}
\end{table*}


The basic cost for an energy weapon is 4, modified by the difference between the weapon technology and the technology of the system in which it is purchased. Thus a T3 laser pack requires an \skill{Assets} check of 4 in a T3 system, but 6 in a T1 system and 3 in a T4 system.

Civilian weapons are one level cheaper.

Energy weapons automatically get the Aspect \aspect{Out of juice}, which one might compel to restrict actions in order to conserve energy.

\subsection{Armour}\label{sec:personal-combat-armour}

Armour not designated as \stunt{Civilian} can only be employed by characters with the \emph{Military-grade EVA} Stunt or any Military-grade personal combat Stunt. A referee would be perfectly right to rule by context (say, for example, that chain mail can't be used with \stunt{Military-grade EVA}), but writing those down as rules would just create a ton of uninteresting exceptions.

For more details on the \stunt{Civilian} stunt and other customizations for armour, see \autoref{sec:personal-armour-design}.

\rulebox{When in doubt, have everyone agree to be reasonable.}

Armour has three statistics: \stat{Defense}, \stat{Stamina Mod}, and \stat{Agility Mod}.

\begin{description}
\item [Defense]
The amount by which an attacker's roll is reduced automatically. It may be modified by the attacking weapon's penetration value.
\item [Stamina Mod]
Applies to powered armour only --- the amount by which the wearing character's \skill{Stamina} rolls are modified. Note that the \skill{Stamina} Skill is not modified, only rolls, so the \Health{} stress track is not affected.
\item [Agility Mod]
Applies to all armour, positive values implying powered armour. This value modifies all \skill{Agility} rolls made by the wearer. Note that the \skill{Agility} Skill is not modified, only rolls. Thus a character with an untrained \skill{Agility} Skill and powered armour with an \stat{Agility Mod} of +2 would roll \dF{} - 1 (untrained Skill value) + 2 (\stat{Agility Mod}) for any \skill{Agility} checks. Most armour will have a negative \stat{Agility Mod}, representing the awkwardness or discomfort of wearing it.
\end{description}

Armour has cost 3, modified by the difference between the armour technology and the technology of the location at which it is purchased. Civilian armour is one cost level cheaper.

\subsubsection{Armour Aspects}
\label{sec:armour-aspects}

\begin{description}
\item[Very heavy] referees should happily compel this to ruin roads, damage bridges, and get the authorities mad where it refers to high technology armour. For lower technology armour it may merely imply awkwardness, restricted vision, and other encumbrance effects appropriate to the armour being described. It might reasonably be compelled against \skill{Stealth} checks and so forth too.
\item[Out of juice] powered armour gets the Aspect \aspect{Out of juice}, which one might compel to restrict actions in order to conserve energy.
\item[Industrial equipment] the armour is intended for civilian industrial use rather than combat.
\end{description}

