\section{Damage}\label{sec:personal-combat-damage}

\index{consequences}
When a character has been hit by an attack that generates shifts, she may take damage, called \emph{stress}. Before marking the stress, she may reduce the shifts by applying one or more \Consequence{}s: a mild \Consequence{} reduces the number of shifts by one, a moderate \Consequence{} by two, and taking a severe \Consequence{} reduces the number of shifts by four.

After mitigation by \Consequence{}s, the remaining number of shifts indicate the box to be marked on the appropriate stress track. Mark this box and all boxes below it. If the highest box to be marked has already been marked, the damage ``rolls up'': mark the next higher open box and all below it.

A player may only ever have a maximum of three \Consequence{}s and may only have a maximum of one of each type regardless of the track the \Consequence{} was scored against. This means that a character suffering economic hardship (see Chapter 4) is easier to take out.

The defender determines the precise wording of the \Consequence{} (subject to reasonableness, as determined by table authority).

\subsection{Taken out}
\label{sec:personal-combat-taken-out}

A character is out of play when he sustains a hit past the end of any stress track. This means a person can be \TakenOut{} without ever taking a \Consequence{}, and therefore without ever taking any serious damage! A person that takes eight shifts of stress past the end of his \Health{} track cannot be saved. That's a one-shot kill\ldots or maybe there's a better way to narrate it?

The attacker narrates taking out his opponent (subject to reasonableness, as determined by table authority). Anything that suits the method (gunfire, punching, whatever) and that genuinely removes the character from play is suitable.

When narrating how an opponent is \TakenOut{}, it is essential to articulate how and if the opponent can return to the game. A ship that has been \TakenOut{} is no longer able to participate in space combat, but could, in theory, be boarded. In this case, the game could shift to the personal combat minigame. Or it could be destroyed, in which case it could not re-enter the game.

This gives a lot of power to the victor, and should be an incentive to players to offer concessions when things aren't going their way. A major opponent \TakenOut{} in personal combat can no longer fight, but the long-term repercussions are determined by the narrative. Being \TakenOut{} might also change features of a character sheet, though this requires some negotiation.

\subsection{Healing}
\label{sec:personal-combat-healing}

Characters cannot begin removing \Consequence{}s until the associated stress track has been cleared (and this is not instantaneous but rather dependent on the number of boxes and the associated \Skill{}).

\subsubsection{Recovering Stress Box Hits}

Stress box hits are not real damage. They are the sweats, panic, scratches, ``only a flesh wound,'' and so on: nothing that can't be fixed with a tiny amount of downtime and nothing that actually affects performance. Consequently all \Health{} and \Composure{} stress track hits are cleared at the first instance of downtime, whether that's a fancy hotel room with no one fighting in it or just the three days' travel time to the slipknot.

\rulebox{All \Health{} and \Composure{} stress hits are erased after a few days' relaxing downtime.} The table should rule when enough time has passed or whether the downtime was sufficiently relaxing.

% ~

% ~

\subsubsection{Recovering Consequences}

Healing \Consequence{}s is governed, in the first instance, by an external time frame, which forces players to endure the effects of combat through the rest of the session.
\begin{itemize}
\item A mild \Consequence{} is cleared as soon as combat is over.
\item A moderate \Consequence{} remains until the end of the session.
\item A severe \Consequence{} must be carried through one complete session in which the associated stress track is never marked. If it is incurred during session one, it is gone no sooner than the end of session two, and if the associated stress track takes hit in a fight during that session, you'll need to hold the \Consequence{} through yet another one.
\end{itemize}

\subsubsection{Medics}

In addition to the purely mechanical process of recovery described above, there may be narrative reasons to introduce the need for actual medical help. The following guidelines are suggested, when pertinent. A mild \Consequence{} can be treated by a medic without a roll after the combat in which the wound was sustained is over. It requires a first-aid kit.

A moderate \Consequence{} remains until a medic can make a successful check against difficulty zero. Base time to heal is a week with (positive or negative) shifts modifying time to solve by one per shift. It requires a medical clinic (such as would be found on an ambulance or in a ship's sick bay), and the technology rating of the facility is applied as a modifier to the roll.

A severe \Consequence{} can be healed by a medic rolling against difficulty 4. It requires an advanced medical facility such as would be found in a hospital, and the technology rating of the facility is applied as a modifier to the roll. The referee may decide that the facility is, despite technology, better or worse equipped and apply this as a modifier to the difficulty. This takes one month, modified by the number of shifts achieved. In no case is the impact of the severe \Consequence{} removed before the end of the session following the one in which it was received. Example: getting a finger shot off

