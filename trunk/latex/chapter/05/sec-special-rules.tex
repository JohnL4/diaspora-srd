\section{Special Rules}\label{sec:personal-combat-special-rules}

\subsection{First Blood}\label{sec:personal-combat-first-blood}

Getting shot is scary. Even when you are a professional. When a player marks a \Health{} stress track hit and has not yet marked any \Health{} or \Composure{} boxes, the \Composure{} stress track is also marked at the same value (and all boxes below, as always). After this initial combat shock all attacks are against \Health{} or \Composure{} but not both.

Note that Consequences reduce shifts \emph{before} they are marked as damage, so they do not have to be applied separately for each track here.  This means that when first hit, a player must decide whether to take a Consequence that will have a doubled effect (but making the character more vulnerable in subsequent rounds) or decide to tough it out in hopes of finishing the fight quickly.

\subsection{Out of Ammo}\label{sec:out-of-ammo}

Who wants to count bullets? Not us. It's way more fun to have an Aspect, and let your opponents decide when you run out of bullets. Anyone using a slug thrower automatically gets the Aspect \aspect{Out of ammo} to be compelled liberally, but which cannot be free-tagged.

Anyone who has used a slug thrower to make an Area of Effect attack (using the \stunt{Full Auto} stunt) gets the Aspect \aspect{Out of ammo} to be compelled liberally, and it can be free-tagged each time the weapon is used for an Area of Effect attack.

\subsection{Military-Grade and Civilian}
\label{sec:military-grade-and-civilian}

The use of any weapon or armour that does not have the \stunt{Civilian} Stunt requires a Military-grade Skill. It may be the case that it is sufficient at some tables to deny access and not explain, but it might be more satisfying to have a mechanism. A character without the appropriate Military-grade stunt can use the military equipment (at her Skill level), but only by paying a fate point for each roll. Thus the player can have her character use the superior but unfamiliar equipment, but with an attendant loss in fate points.

\subsection{Hostile Environments}\label{sec:hostile-environments}

Sometimes a fight will take place in an environment where the integrity of armour is important not only to absorb combat damage but also to resist environmental effects. These environments might include low pressure, high pressure, or toxic atmospheres. In these cases a loss of suit integrity has serious ramifications. A hostile environment suit has lost integrity when the wearer takes any \Health{} track Consequence.

\iflandscape{}{\vfil}
\subsection{Zero Gravity}\label{sec:zero-gravity}

When fighting in zero or low gravity the scene has the Aspect \aspect{Zero gravity} or \aspect{Low gravity}. This can be tagged as usual by participants.

Some weapons are recoilless, and are designed for low gravity. These have the \stunt{Low Recoil} Stunt. All attacks using weapons without the \stunt{Low Recoil} Stunt use the \skill{MicroG} Skill instead of their preferred Skill (\skill{Brawling}, \skill{Close Combat}, \skill{Slug Thrower}). The \skill{MicroG} Skill does not confer knowledge of the maintenance and repair of any weapons: for that, checks need to be made against the relevant weapons Skill.

\skill{MicroG} rolls may also be called for to perform movement or other activity in zero or low gravity.

The referee may determine additional penalties that apply in MicroG environments: without a handhold, it simply may not be possible to throw a grenade effectively.

In some contexts the shifting of gravity can lead to interesting play environments. This might lead to a permanent penalty on all action in the scene. For example:
\begin{description}
\item [Sloping gravity]
the ship is rotating under thrust. All actions are done as if in gravity (i.e. without the \skill{MicroG} Skill) and are at -2
\item [Stuttering microgravity]
a drive keeps kicking in and out. All actions are as in micro-G, but at -1
\item [Low gravity]
all actions are at -2, using the better of \skill{MicroG} or the relevant combat Skill.
\end{description}

These environmental effects may be determined by the referee as the map is designed, or they may be a consequence of player actions.

