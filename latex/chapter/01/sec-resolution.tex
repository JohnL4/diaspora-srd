\newpage
\section{Resolution}
\label{sec:resolution}

First, the acting player declares his intended action and, if there is one, its target. Next, the player managing the order of events (usually the referee) asks for compels.

\index{aspects!compelling!scope}
Any player can compel the acting player's character. If someone does, he offers a story based on one of the acting player character's Aspects and presents a fate point. If the acting player accepts, he receives the fate point and complies with the compeller, forfeiting his action. If he denies he must pay the compeller a fate point. During combat, the compeller can demand only one of two things: inaction or forced movement. The mechanical specifics of these is described in each combat mini-game.

The referee, however, can compel at any time, in or out of combat, and the effects of the compel are not constrained. That is, the referee can offer pretty much anything as a compel, as long as it ties in to the Aspect well.

Once compels have been resolved, any actions that can be taken will be resolved. This usually means that dice are rolled and a Skill value added.

\index{aspects!invoking}
\index{aspects!compelling}
Players may invoke an Aspect of theirs, narrating how it relates to the situation, pay a fate point, and gain either +2 on their roll or the right to re-roll. They may tag an Aspect on their opponent, narrating how it relates to the situation, pay a fate point, and gain +2 on their roll.

Players may tag as many free-taggable Aspects as they like that exist on their opponent and gain +2 on their opponent for each.

Players may use any spin accumulated by them or an ally to gain +1 on their roll (see ``Shifts and Spin'', below). Any number of spin points may be used towards a given roll.

\rulebox[
Compels are before dice.
Invokes, tags, and spin are after dice.
]{
Compels happen before the dice hit the table.
Invokes, tags, and spin happen after the dice hit the table.
}

There are a couple of possible kinds of tasks you might run into:

% \newpage
\vfil
\subsection{Fixed Difficulty Roll}
\label{sec:fixed-difficulty-roll}

\index{skills!fixed-difficulty roll}
The referee sets a difficulty level and tells you what skill you need to use. You throw your fate dice, add them to your skill and if it's equal to or greater than the difficulty, you succeed. If you fail or want to improve your result, you can invoke an Aspect and spend a fate point for +2 or a re-roll. If you still are not satisfied with the result, you can try to bring in another Aspect. And so on.

A fixed difficulty roll is made with \dplusskill{} against a target value established by the referee based on the estimation of the task's difficulty (the Ladder may be a useful tool here).

The result is the number of shifts obtained.  Zero shifts is a success. Rolls that use shifts for effects do not generate effects with zero shifts. These may be useless successes.

% ~

\subsection{Opposed Roll}
\label{sec:opposed-roll}

\index{skills!opposed roll}
You want to beat someone else at something. The defender rolls defensively and 
you need to meet or beat that roll offensively. Use fate points as before to 
increase your result. Your opponent may well do the same.

An opposed roll is \dplusskill{} compared against an opponent's \dplusskill. 
Attacker and defender may use different skills. The result is the number of 
shifts obtained.

Zero shifts is a success. Rolls that use shifts for effects do not generate 
effects with zero shifts. These may be useless successes.

\subsection{Shifts and Spin}
\label{sec:shifts-and-spin}

\begin{wrapfigure}[14]{l}[\sidebarwidth]{\halfbarwidth}
\colorbox{sbbackground}{\begin{minipage}{\halfbarinnerwidth}%
\sideboxtitle{Summary: Rolls}
\begin{description}
\item[Fixed]~\\
referee sets a difficulty value. the player must meet or exceed it with dice + skill.

\keyword{shifts} = result - difficulty

\item[Opposed]~\\
attacker and defender roll dice + skill.

\keyword{shifts} = attacker result - defender result

If negative 3 or lower, defender gets \keyword{spin}.

\end{description}
\end{minipage}}%
\end{wrapfigure}%
% \end{sidebox}

\index{skills!shift}\index{skills!spin}
The degree to which you beat your target value is your \keyword{shift}. Shifts 
may have a mechanic effect. In combat, for example, shifts determine the amount 
of damage done. If you exceed your opponent when you make a defensive roll that 
does not have any effect other than being a successful defense, you don't get 
shifts. Instead, \emph{for every three} by which you exceed the attacking roll, 
you generate \keyword{spin}.

Spin can be used by you or an ally to gain +1 on a roll --- basically your 
defensive maneuver put your opponent at a disadvantage or one of your allies at 
an advantage somehow. You need to use up spin by the time the turn comes back 
to whoever generated it --- he's the last person that can take advantage.

It can be handy to throw some kind of token on the table to indicate spin and 
let anyone pick it up. Index cards with the generator's name on it greatly aid 
remembering when it expires.

\index{shifts}
\rulebox{Each number above the target value is called a \keyword{shift}.}
Hitting the target number exactly is a success, but generates no shifts. When an attacker fails by 3 or more (negative three shifts), the defender gets spin.

Spin can be spent by the defender or any ally, and must be used before the end of the defender's next turn. Defenses that can harm the attacker (such as defending against \skill{Electronic Warfare} in space combat) do not generate spin because they already have an effect beyond successful defense.

\subsection{Declarations}
\label{sec:declarations}

\index{declarations}
At any time a player can pay a fate point and declare a true fact about the world as pertains to the character's apex \Skill. The referee can return the fate point and modify the fact but cannot simply deny it altogether.

\subsection{Skill interactions}\label{sec:skill-interactions} % \href{sec:id67}

Sometimes it makes sense to have two skills interact to form the basis of a
check.
\begin{description}
\item[Skill A is \emph{Limited by} Skill B]~

This indicates that Skill A is used for the check but at a level no
higher than Skill B.
\item[Skill A is \emph{Amplified by} Skill B]~

This indicates that Skill A is used for the check. If Skill B exceeds Skill
A, then an additional +1 is granted.
\end{description}

