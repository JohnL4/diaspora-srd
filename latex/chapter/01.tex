%%% ``Playing With FATE'' chapter of the Diaspora SRD
%%%
%%% This file is distributed under the Artistic Licence, v.2.0
%%% You should have received a file containing the license along
%%% with this file; if not see:
%%% http://www.perlfoundation.org/artistic_license_2_0
%%%
\chapter{Playing with FATE}
\label{cha:playing-with-fate}

Diaspora is a role-playing game with a focus on hard(ish) science-fiction adventure. You build a universe, you build characters, and then you play with them in it.

Underpinning the game is the task resolution system described in this chapter.

\index{Spirit of the Century}
All conflicts in Diaspora are resolved using the FATE mechanics as elaborated in \emph{Spirit of the Century} and available from the System Reference Document for that game, available on the \href{http://bzr.mausdompteur.de/fate3/fate3.html}{Internet}.%
\footnote{ \texttt{http://bzr.mausdompteur.de/fate3/fate3.html} }

You roll your set of four fudge dice, which yields a result between -4 and +4, you add an appropriate skill, and then you compare against some difficulty level, which might be someone else's roll or might be a level imposed by the referee.

%%% Define some text blocks, and then lay them out depending on the
%%% document layout.
%
%% blurb about fudge dice
\def\FUDGEDICE{
\index{dice!fudge dice}
Almost every roll in \emph{Diaspora} is \dF: a single roll of four Fudge dice, yielding a range from -4 to +4. A Fudge die is a d6, with two faces marked -, two marked +, and two blank. You add up the +s, subtract the -s, and you have a total.
The resultant probability curve is the same as rolling \texttt{4d3-8}, so you could use ordinary \texttt{d6}s and treat 1--2 as -, 3--4 as blank, and 5--6 as +.
}%
%
%% blurb about alternate dice systems
\def\ALTDICE{
\index{dice!alternate systems}
{\medskip\large\bfseries Option: alternate dice systems}
%
% The system will work with a different probability curve by using one of the following die mechanics:
\begin{description}
\item [\texttt{d6-d6}, $\pm5$]
Roll two different-coloured six-sided dice and subtract one from the other. You can establish whatever convention you like for which die is which (such as subtracting dark from light, or red from black), as long as you are consistent.

\item [\texttt{d6-d6}, $\pm4$]
As above, but treat $\pm5$ as 0. This has the same numeric range as \dF, but with a better chance of an extreme result.
\end{description}
}%
%
%% probability comparison table
\def\DICETABLE{
\begin{tabular}{r@{}l*3{r@{/}lr@{.}l}}
\multicolumn{2}{c}{}
{}	& \multicolumn{4}{c}{\dF}
{}	& \multicolumn{4}{c}{\texttt{d6-d6}, $\pm4$}
{}	& \multicolumn{4}{c}{\texttt{d6-d6}, $\pm5$} \\
\toprule
\multicolumn{2}{c}{roll}
{}	& \multicolumn{2}{c}{odds} & \multicolumn{2}{c}{\%}
{}	& \multicolumn{2}{c}{odds} & \multicolumn{2}{c}{\%}
{}	& \multicolumn{2}{c}{odds} & \multicolumn{2}{c}{\%} \\
\midrule
$\pm$&5	& \multicolumn{8}{c}{}		&  1&36	&  2&77 \\
$\pm$&4	&  1&81	&  1&24	&  2&36	&  5&55	&  2&36	&  5&55 \\
$\pm$&3	&  4&81	&  4&95	&  3&36	&  8&3	&  3&36	&  8&3 \\
$\pm$&2	& 10&81	& 12&35	&  4&36	& 11&1	&  4&36	& 11&1 \\
$\pm$&1	& 16&81	& 19&75	&  5&36	& 13&88	&  5&36	& 13&88 \\
& 0	& 19&81	& 23&46	&  8&36	& 22&22	&  6&36	& 16&66 \\
\bottomrule
\end{tabular}
}%
%
% \begin{sidebox}*{Fudge dice}
\begin{wrapfigure}[20]{r}[\sidebarwidth]{\halfbarwidth}%*{Fudge dice}
% \begin{sidetable}
\ifthenelse{\boolean{LANDSCAPE}}%
{
% \begin{sidebox}[Dice probability distributions]{Fudge dice}[tab:fudge-dice]
\centering%
\hspace*{\fill}
\begin{tabular}{p{0.45\columnwidth}p{0.45\columnwidth}r}
\multicolumn{2}{p{0.95\columnwidth}}{\FUDGEDICE} \\
\ALTDICE &
\flushright \DICETABLE
\end{tabular}
\hspace*{\fill}
}{%
\colorbox{sbbackground}{\begin{minipage}{\halfbarinnerwidth}
\sideboxtitle{Fudge dice}%
\FUDGEDICE

\ALTDICE
\end{minipage}}%
% \centering \DICETABLE
}%
% \label{tab:4df-in-statistics}
% \end{sidetable}
\end{wrapfigure}
% \end{sidebox}


Diaspora is also a set of mini-games. Each of these use fate dice, Aspects, and other elements from the FATE system but the may have other distinctions.  These mini-games are:

\ifthenelse{\boolean{LANDSCAPE}}{\newpage}{}

\begin{itemize}
\item Cluster Creation (\autoref{cha:clusters})
\item Character Creation (\autoref{cha:characters})
\item Personal Combat (\autoref{cha:personal-combat})
\item Space Combat (\autoref{cha:space-combat})
\item Social Combat (\autoref{cha:social-combat})
\item Platoon Combat (\autoref{cha:platoon-combat})
\end{itemize}

\vfil
\section{Players get the Power}\label{sec:players-get-the-power}
% \vfil

\emph{Diaspora} is not a game in which the players drive the action without the input from the referee, who only establishes the setting and mediates the rules.  But many of the ways players can use their characters' skills give the player some power over narration.

\rulebox{Say yes or roll the dice.}

The guiding principle is \emph{say yes or roll the dice.} When a player has an idea about what he wants to happen, it can often be the case that what he wants doesn't mesh with what the referee wanted. Look at the idea, ignore your plans, and either say \emph{yes} or set a difficulty and make them roll to see what happens.

Alternatives to \emph{yes} exist that are not \emph{no.} One popular one is, \emph{yes, but...}: In this case the referee agrees but adds a complication. If everyone is grinning and nodding, the referee has succeeded. Another is \emph{yes, and...}: Here the referee agrees and escalates the player's idea even further.

Players sometimes get to say something that's true without much mediation from the referee.

We also talk frequently about \emph{the table}. The table is the sum of the players, the referee included, with all opinions weighed equally. The table is the consensus, and it is more important than any single player's authority, including the referee's.

\rulebox{The table is the consensus.}

We grant equal weight (though your table may choose otherwise) to all players throughout the first session. Cluster, world, and character creation are all egalitarian pursuits.

\ifthenelse{\boolean{LANDSCAPE}}{}{\vfil}

% \vfil
\section{Abstraction}
\label{sec:abstraction}
% \vfil

% 
\begin{wraptable}{r}[\sidebarwidth]{3\sidebarwidth}
\centering
\begin{tabular}{r@{:\quad}l}
\toprule
Score	& Adjective \\
\midrule
+8	& Legendary \\
+7	& Epic \\
+6	& Fantastic \\
+5	& Superb \\
+4	& Great \\
+3	& Good \\
+2	& Decent \\
+1	& Average \\
+0	& Mediocre \\
-1	& Poor \\
-2	& Terrible \\
\bottomrule
\end{tabular}
\caption[The Adjective Ladder]{The Adjective Ladder}
\label{tab:the-ladder}
\end{wraptable}


In place of hard rules, what Diaspora rewards is narration: narration from the players, and from the referee. Giving details of what you want to happen within the game is as important as working out what roll on the dice is needed for success. Both, of course, will happen. The talk around the table will always be a mix of in-game-character-based narration and out-of-character rules discussion. There are no necessary mechanical consequences of narration in the game, but it may still prove to be the most memorable of the session.

Player authority and character integrity are both important. Because of the fate point economy, it will often be the case that the player wants something to happen that the character would not. Two things follow from this.

First, the referee keeps very little mechanical information secret. Mechanical details, such as aspects and skills, are not hidden from the players (unless there is a game-based reason why that might be). Players are always maintaining a double awareness at all times, and the tension between player and character is something that the FATE system exploits powerfully.

Second, there is a continual back and forth between these two levels, and narration, from the players and from the referee, becomes essential. A player narrates what he wants to happen, which may lead to an out-of-character tabulation of whether a roll is needed and what the target number might be. Dice are rolled, and the result leads to more narration (from the successful player, from the referee, or from the table generally) giving an interpretation of the roll within the game.

Abstraction facilitates narration, because it allows the players to define constraints or accomplishments for themselves. Narration feeds into the rules, which then in turn provide opportunities for the interpretation of a given roll, in the form of more narration. It's all about the stories.


\begin{wraptable}{r}[\sidebarwidth]{3\sidebarwidth}
\centering
\begin{tabular}{r@{:\quad}l}
\toprule
Score	& Adjective \\
\midrule
+8	& Legendary \\
+7	& Epic \\
+6	& Fantastic \\
+5	& Superb \\
+4	& Great \\
+3	& Good \\
+2	& Decent \\
+1	& Average \\
+0	& Mediocre \\
-1	& Poor \\
-2	& Terrible \\
\bottomrule
\end{tabular}
\caption[The Adjective Ladder]{The Adjective Ladder}
\label{tab:the-ladder}
\end{wraptable}


In FATE, successes and difficulties are rated by numbers or by the terms on the ladder (\autoref{tab:the-ladder}). Our Ladder here is slightly different from the \emph{Spirit of the Century} Ladder, in that the term \emph{Fair} is replaced by \emph{Decent}.

The words only applicable directly when a single character acts. Since an apex \Skill\ is at level 5 (as we will see in character generation) and since the best result from a roll of the dice is +4, a result of +9 represents an exceptionally successful attempt at something by a dedicated professional.

While higher numbers are possible (through the invocation of Aspects, described below), most numbers in the game, when all things are considered, are single digits. If one is looking for appropriate adjectives to describe an action, it is often the difference between two rolls that might determine the quality of success. So, in an opposed roll (described below, in which a player roll is compared against a referee roll) results of 7 against 5 represent a decent success.

\vfil
% 
\begin{wraptable}{r}[\sidebarwidth]{3\sidebarwidth}
\centering
\begin{tabular}{r@{:\quad}l}
\toprule
Score	& Adjective \\
\midrule
+8	& Legendary \\
+7	& Epic \\
+6	& Fantastic \\
+5	& Superb \\
+4	& Great \\
+3	& Good \\
+2	& Decent \\
+1	& Average \\
+0	& Mediocre \\
-1	& Poor \\
-2	& Terrible \\
\bottomrule
\end{tabular}
\caption[The Adjective Ladder]{The Adjective Ladder}
\label{tab:the-ladder}
\end{wraptable}

\section{The Fate Point Economy}\label{sec:the-fate-point-economy} %\label{sec:id58}

Fueling almost all interactions in \emph{Diaspora} is the fate point economy. Characters have fate points, as do ships, and the referee has an unending supply. Even if a given interaction doesn't actually lead to an exchange of fate points, the possibility that it might do so inevitably affects player choices. Fate points are limited, and as a scarce resource, players will be looking to spend them carefully, and collect them zealously. If a player wants something to happen and the dice have said no, then fate points provide a mechanism for the player to create success.

Fate points use other qualities of a character to create in-game effects; that is why the precise wording of an aspect can be so important. The natural instinct for players is to hoard fate points, and save them for a big flourish at the end. But there are rewards to be had in keeping the flow of fate points relatively constant. Maybe not for every roll, but regularly, fate points should be spent by players, or should be offered by the referee, to create a sense of them as units of trade, as a genuine economy, that creates an ebb and flow throughout the session.


\begin{sidebox}{Summary: Aspects}
\begin{description}
\item[Invoke your aspect after dice roll:]~

+2 or re-roll, costs you a fate point

\item[Tag another aspect after dice roll:]~

+2 or re-roll, costs you a fate point

\item[Compel an aspect before dice roll:]~

negotiated result, costs someone a fate point (compeller if accepted, compellee if denied)
\end{description}
\end{sidebox}

\section{Aspects}\label{sec:aspects}  %\href{sec:id59}

\begin{sidebox}{Summary: Aspects}
\begin{description}
\item[Invoke your aspect after dice roll:]~

+2 or re-roll, costs you a fate point

\item[Tag another aspect after dice roll:]~

+2 or re-roll, costs you a fate point

\item[Compel an aspect before dice roll:]~

negotiated result, costs someone a fate point (compeller if accepted, compellee if denied)
\end{description}
\end{sidebox}


All characters and some things will have \Aspects. These short phrases indicate 
what is important about the character. Scenes might have \Aspects, maps might 
have \Aspects, systems, worlds, and cities might all have \Aspects. Give a 
thing an \Aspect\ when you want it to have a feature but don't need a specific 
rule mechanism to govern how that feature operates. Instead you are declaring 
that something is important and leave it to players to determine how to make 
it important.

% \vfil
\subsection{What do Aspects do?}\label{sec:what-do-aspects-do} %\href{sec:id60}

There are a number of ways that \Aspects\ come into being, and a number of 
ways they can be used during conflict (whether that's just a skill check 
during regular role-play or a specific roll in a combat sequence).

\rulebox{
Any time you roll the dice, you could bring \Aspects\ into play.
}

\subsubsection{invoking aspects}

You can \emph{invoke}\index{aspects!invoking} one of your own \Aspects\ 
\emph{after} you roll the dice. You narrate how the \Aspect\ affects the roll 
and, assuming everyone at the table nods assent and says \emph{that's cool,} 
you can add +2 to your roll or else re-roll. You pay a fate point immediately.

\subsubsection{tagging aspects}

\begin{sidebox}{Aspects}
The selection of a character's aspects an essential part of character generation.

Aspects are the catalysts for the economies of fate points. They need to be worded in a way that you can invoke them on yourself (for when a bonus to a roll is needed), but --- more importantly --- they need to invite compels from the referee. Otherwise you lose your fate points too quickly and there is no obvious source for replenishment.

A well-worded Aspect can be both revealing of the character's nature, and be obviously invokable for both benefit and detriment.

Not all aspects can work that way, and it may emerge in play that some aspects do not enter into the fate point economy at all. They are the ones which can be traded out through the experience process.

Aspects reveal something about the character that the character may not even be aware of. Similarly, an Aspect might be a physical object (an heirloom weapon, or a spaceship). In making that choice, the player is telling the referee that this object is part of the character identity. It won't be taken away, but it will also confer obligations and responsibilities, so that it too is an active part of the economy.
\end{sidebox}

You can \emph{tag}\index{aspects!tagging} an \Aspect\ on something else that's 
relevant to the roll \emph{after} you roll the dice. That could be an \Aspect\ 
on your opponent, an ally, on a map  or any other \Aspect\ that's relevant and 
not on your character. Take +2 on your roll or else re-roll. You pay a fate 
point immediately.

\subsubsection{compelling opponent aspects}

You can \emph{compel}\index{aspects!compelling} an \Aspect\ on your opponent 
\emph{before} you roll the dice. In this case you offer your opponent a deal 
related to his \Aspect: he can take the deal and one of your fate points or 
deny the deal and give you a fate point. Outside of a combat sequence the deal 
can be quite free-form and it is a negotiation between \emph{players} and not 
between characters. You might offer the Referee a deal relating to an NPC, a 
deal relating to an ally or most commonly a deal offered by the Referee to a 
character's player. During a combat sequence the effects of a compel are far 
more constrained (and dealt with in detail in the appropriate section).

\subsubsection{compelling other aspects}

You can \emph{compel} an \Aspect\ on a scene or zone (or anything for that 
matter). You offer the Referee a deal related to the \Aspect: he can take the 
deal and one of your fate points or deny the deal and give you a fate point. 
In or out of combat, the deal is free-form and it is a negotiation between 
\emph{players} and not characters. You might offer the Referee a deal relating 
to any scope as mentioned in the ``Aspect scope'' section.

\subsection{Aspect scope}

As mentioned above, any character, vehicle, location, object, story arc or 
anything else may have \Aspects. For any given \Aspect, the ``thing'' the 
\Aspect\ belongs to is called the \Aspect's \emph{scope}. You may only tag one 
\Aspect\ on each related \emph{scope}\index{aspects!scope} per roll. In some 
cases additional scopes may be added during play, but some sample \Aspect\ 
scopes are:

\begin{itemize}
\item yourself
\item Opponent
\item System
\item Scene
\item Zone
\item Ship
\item Campaign
\item Ally
\end{itemize}

In addition, any number of free-taggable\index{aspects!free tags} Aspects from any scope may be tagged and don't count against your tagging limit (that is, you can tag two free-taggables at zone scope and still tag a third if there is one for the usual fate point cost).

\subsection{Where do Aspects come from?}

Aspects come into being in several ways:
\begin{itemize}
\item player characters start with 10 Aspects derived from the character generation stories. They get a \emph{fate refresh} of 5 fate points at the beginning of a session.

\item spaceships start with 5 Aspects\index{spaceships!aspects!starting amount} created by the designer (some forced by the design process). They start each session with five fate points.\index{aspects!on spaceships}\index{spaceships!fate points!starting amount}

\item scenes, maps, campaigns, and things get Aspects at the discretion of the referee.\index{aspects!on scenes}\index{aspects!on maps}\index{aspects!on campaigns} The referee has an unlimited supply of fate points.

\item players can put an Aspect on a character or scene with a \textbf{Maneuver}.
\end{itemize}

\subsection{Maneuvers}\label{sec:maneuvers} % \href{sec:id61}

A maneuver\index{maneuver} is an action your character takes that will change the status of something and this status change will be represented by the addition of an Aspect. The referee will decide what to roll (either a \nameref{sec:fixed-difficulty-roll} or an \nameref{sec:opposed-roll} --- see the \nameref{sec:resolution} section) and on success the target acquires an appropriate Aspect.\index{aspects!from maneuvers}

Having an Aspect of your choosing placed on an enemy is pretty powerful all by itself, but there is an additional power: an Aspect placed as a result of a maneuver can be tagged \emph{without paying a fate point} once by the maneuverer or an ally (it is \emph{free-taggable}).\index{maneuver!and free tags} It can be tagged normally subsequently as long as the Aspect lasts, but the first time (and only the first time) is free.

\rulebox{Place an Aspect on an opponent or a scene with a Skill check}
Place an Aspect on an opponent or a scene with a Skill check (static or opposed, as determined by the referee). If successful, the target now has the Aspect.\index{aspects!from skill checks} This Aspect can be tagged once for free and thereafter for a fate point.


\newpage
\section{Resolution}
\label{sec:resolution}

First, the acting player declares his intended action and, if there is one, its target. Next, the player managing the order of events (usually the referee) asks for compels.

\index{aspects!compelling!scope}
Any player can compel the acting player's character. If someone does, he offers a story based on one of the acting player character's Aspects and presents a fate point. If the acting player accepts, he receives the fate point and complies with the compeller, forfeiting his action. If he denies he must pay the compeller a fate point. During combat, the compeller can demand only one of two things: inaction or forced movement. The mechanical specifics of these is described in each combat mini-game.

The referee, however, can compel at any time, in or out of combat, and the effects of the compel are not constrained. That is, the referee can offer pretty much anything as a compel, as long as it ties in to the Aspect well.

Once compels have been resolved, any actions that can be taken will be resolved. This usually means that dice are rolled and a Skill value added.

\index{aspects!invoking}
\index{aspects!compelling}
Players may invoke an Aspect of theirs, narrating how it relates to the situation, pay a fate point, and gain either +2 on their roll or the right to re-roll. They may tag an Aspect on their opponent, narrating how it relates to the situation, pay a fate point, and gain +2 on their roll.

Players may tag as many free-taggable Aspects as they like that exist on their opponent and gain +2 on their opponent for each.

Players may use any spin accumulated by them or an ally to gain +1 on their roll (see ``Shifts and Spin'', below). Any number of spin points may be used towards a given roll.

\rulebox[
Compels are before dice.
Invokes, tags, and spin are after dice.
]{
Compels happen before the dice hit the table.
Invokes, tags, and spin happen after the dice hit the table.
}

There are a couple of possible kinds of tasks you might run into:

% \newpage
\vfil
\subsection{Fixed Difficulty Roll}
\label{sec:fixed-difficulty-roll}

\index{skills!fixed-difficulty roll}
The referee sets a difficulty level and tells you what skill you need to use. You throw your fate dice, add them to your skill and if it's equal to or greater than the difficulty, you succeed. If you fail or want to improve your result, you can invoke an Aspect and spend a fate point for +2 or a re-roll. If you still are not satisfied with the result, you can try to bring in another Aspect. And so on.

A fixed difficulty roll is made with \dplusskill{} against a target value established by the referee based on the estimation of the task's difficulty (the Ladder may be a useful tool here).

The result is the number of shifts obtained.  Zero shifts is a success. Rolls that use shifts for effects do not generate effects with zero shifts. These may be useless successes.

% ~

\subsection{Opposed Roll}
\label{sec:opposed-roll}

\index{skills!opposed roll}
You want to beat someone else at something. The defender rolls defensively and 
you need to meet or beat that roll offensively. Use fate points as before to 
increase your result. Your opponent may well do the same.

An opposed roll is \dplusskill{} compared against an opponent's \dplusskill. 
Attacker and defender may use different skills. The result is the number of 
shifts obtained.

Zero shifts is a success. Rolls that use shifts for effects do not generate 
effects with zero shifts. These may be useless successes.

\subsection{Shifts and Spin}
\label{sec:shifts-and-spin}

\begin{wrapfigure}[14]{l}[\sidebarwidth]{\halfbarwidth}
\colorbox{sbbackground}{\begin{minipage}{\halfbarinnerwidth}%
\sideboxtitle{Summary: Rolls}
\begin{description}
\item[Fixed]~\\
referee sets a difficulty value. the player must meet or exceed it with dice + skill.

\keyword{shifts} = result - difficulty

\item[Opposed]~\\
attacker and defender roll dice + skill.

\keyword{shifts} = attacker result - defender result

If negative 3 or lower, defender gets \keyword{spin}.

\end{description}
\end{minipage}}%
\end{wrapfigure}%
% \end{sidebox}

\index{skills!shift}\index{skills!spin}
The degree to which you beat your target value is your \keyword{shift}. Shifts 
may have a mechanic effect. In combat, for example, shifts determine the amount 
of damage done. If you exceed your opponent when you make a defensive roll that 
does not have any effect other than being a successful defense, you don't get 
shifts. Instead, \emph{for every three} by which you exceed the attacking roll, 
you generate \keyword{spin}.

Spin can be used by you or an ally to gain +1 on a roll --- basically your 
defensive maneuver put your opponent at a disadvantage or one of your allies at 
an advantage somehow. You need to use up spin by the time the turn comes back 
to whoever generated it --- he's the last person that can take advantage.

It can be handy to throw some kind of token on the table to indicate spin and 
let anyone pick it up. Index cards with the generator's name on it greatly aid 
remembering when it expires.

\index{shifts}
\rulebox{Each number above the target value is called a \keyword{shift}.}
Hitting the target number exactly is a success, but generates no shifts. When an attacker fails by 3 or more (negative three shifts), the defender gets spin.

Spin can be spent by the defender or any ally, and must be used before the end of the defender's next turn. Defenses that can harm the attacker (such as defending against \skill{Electronic Warfare} in space combat) do not generate spin because they already have an effect beyond successful defense.

\subsection{Declarations}
\label{sec:declarations}

\index{declarations}
At any time a player can pay a fate point and declare a true fact about the world as pertains to the character's apex \Skill. The referee can return the fate point and modify the fact but cannot simply deny it altogether.

\subsection{Skill interactions}\label{sec:skill-interactions} % \href{sec:id67}

Sometimes it makes sense to have two skills interact to form the basis of a
check.
\begin{description}
\item[Skill A is \emph{Limited by} Skill B]~

This indicates that Skill A is used for the check but at a level no
higher than Skill B.
\item[Skill A is \emph{Amplified by} Skill B]~

This indicates that Skill A is used for the check. If Skill B exceeds Skill
A, then an additional +1 is granted.
\end{description}


\subsection{Skill interactions}\label{sec:skill-interactions} % \href{sec:id67}

Sometimes it makes sense to have two skills interact to form the basis of a
check.
\begin{description}
\item[Skill A is \emph{Limited by} Skill B]~

This indicates that Skill A is used for the check but at a level no
higher than Skill B.
\item[Skill A is \emph{Amplified by} Skill B]~

This indicates that Skill A is used for the check. If Skill B exceeds Skill
A, then an additional +1 is granted.
\end{description}



\begin{wraptable}{r}[\sidebarwidth]{3\sidebarwidth}
% \begin{table}[ht]
\centering
% \includegraphics[clip,trim=0.8cm 5.9cm 9.3cm 7.8cm,scale=1]{tables/ladder_time_system}
\begin{tabular}{l}
\toprule
Instant \\
A few moments \\
Half a minute \\
A minute \\
A few minutes \\
15 minutes \\
Half an hour \\
An hour \\
A few hours \\
An afternoon \\
A day \\
A few days \\
A week \\
A few weeks \\
A month \\
A few months \\
A season \\
Half a year \\
A year \\
A few years \\
A decade \\
A lifetime \\
\bottomrule
\end{tabular}
\caption[The Time Track]{The Time Track}
\label{tab:the-time-track}
% \end{table}
\end{wraptable}

\subsection{Dealing with Time}
\label{sec:dealing-with-time}
\index{skills!time}

When you want an action to succeed but have the degree of success (or failure) determine how long it took, the referee should set the difficulty and the base time needed to resolve (picked from the Time Track in \autoref{tab:the-time-track}). Each shift generated moves up the track one line. Negative shifts move down the track one line for each shift.

Some tasks might fail altogether with negative shifts, but potentially go faster with greater success.

The referee can decide that a task cannot fail if the character takes all the time in the world to finish it. In that case, negative shift does not signify failure but additional time needed, by moving down from the base time set according to the rolled result.



