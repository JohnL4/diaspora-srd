\chapter{Clusters}
\label{cha:clusters}

The first session of a Diaspora campaign is used to create the setting and the characters. At the time of this first session it is not necessary to have a referee (what some games call the Game Master or GM).  Everyone can have complete narrative authority over the pieces they will create.

Designate someone as caller. This person will guide the group through the application of the rules and perhaps take notes on the results, even though the caller's creative input need not be any greater than that of any other player.

\index{cluster}\index{slipstream}
You will create a handful of systems and find out what they are like, filling in details with your own stories as you make sense of the system statistics.
%
You will then link these systems into a structure called the clus\-ter, which will show which systems are connected to each other by slipstreams. Faster-than-light travel between stars only occurs along these paths. Once this geometry is established, it can be useful to return to the systems and write a little more: how do various surpluses and deficiencies affect slipstream traffic? Who supplies slip ships? Who com\-petes?

% \vspace{5.85cm}

% .

% Table of system attribute ratings

\begin{table*}[ht]
\centering
% \includegraphics[clip,trim=6.25cm 2cm 1.8cm 1.4cm,scale=0.65]{tables/ladder_time_system}
\begin{tabular}{clll}
\toprule
Rating & Technology & Environment & Resources \\
\midrule
4 & On the verge of collapse
& Many garden worlds
& All you could want \\
3 & Slipstream mastery
& Some garden worlds
& Multiple exports \\
2 & Slipstream use
& One garden and several survivable worlds
& One significant export \\
1 & Exploiting the system
& One garden and several hostile environments
& Rich \\
0 & Exploring the system
& One garden world (perhaps additional barren worlds)
& Sustainable \\
-1 & Atomic power
& Survivable world
& Almost viable \\
-2 & Industrialization
& Hostile environment (gravity but dangerous atmosphere)
& Needs imports \\
-3 & Metallurgy
& Barren world (gravity, no atmosphere)
& Multiple dependencies \\
-4 & Stone Age
& No habitable gravity or atmosphere
& No resources \\
\bottomrule
\end{tabular}
\caption{System attribute ratings}
\label{tab:system-attribute-ratings}
\end{table*}

\section{Systems}\label{sec:Systems} % \href{sec:id70}

The first step in creating a cluster is to create the set of systems that will belong in it. The caller shall assign each player, including himself, one or two systems. The total number of systems usually is between six and ten.

Systems will be described by their statistics, and their Aspects. Details can be elaborated through narrative, but will have no mechanical effect in the game unless accompanied by Aspects.

\index{slipknot}
Each system represents some place in space where humans might reside. A place where two slipknots exist --- those mysterious points in space that allow limited faster-than-light travel. Nothing else is written in stone --- a system can be completely empty but for the slipknots, and that's got to have a story.

Typically a system consists of a star and some attendant planetary bodies. It could be a as familiar as a yellow star with eight worlds, one of which is habitable, or as exotic as an artificial quintet of neutron stars and a vast field of rubble a thousand million miles away. These things are for you to determine. They are what you invent to make sense of the statistics.

\iflandscape{}{\vfil}
\section{Generation}\label{sec:Generation} % \href{sec:id71}

\begin{enumerate}
\item Each player (including the referee) is assigned the worlds they will develop. Everyone will use slightly different notation, but numbering the worlds to be generated on a piece of paper, allowing three or four lines per system, is sufficient. We find it helpful if everyone records the information for themselves as the process is underway.

\item In turn, each player determines the attributes for their system. Systems have three attributes:
\stat{Technology}, \stat{Environment} and \stat{Resources}.

% \begin{itemize}
% \item Technology
% \item Environment
% \item Resources
% \end{itemize}
%
% % \vfil
% \newpage

Attributes are generated by a roll of the dice (\dF). The typical world will therefore be T0 E0 R0, or a system with a sustainable garden world which is actively exploring space. See \autoref{tab:system-attribute-ratings} for descriptions of each attribute rating.

\item The Slipstream Guarantee. It is suggested (though perhaps not strictly necessary) that at least one system be able to create and maintain slip-capable ships. If no system in the cluster has T2 or higher, give the system with the lowest sum of attributes and the system with the highest sum of attributes each T2.

\item Each attribute value corresponds to a short phrase (see the table below), which may become (and will certainly influence) one of the system's Aspects. These may be noted, but as they derive automatically from the attribute value, they may just be borne in mind as the procedure continues.

\item Each player shall give each of their systems a name.

\item Players give each of their systems two Aspects that reflect their unique identities, extrapolated from the attributes. The final result is under the control of the table authority. Give it two Aspects that describe things that are not represented by the numbers or that are implied by the numbers. These things might relate to politics,philosophy, geography, hydrography, local astrography or history. Or something else...

\item The relationship between the systems within the cluster is then developed. The procedure is described below in the section \nameref{sec:linking-systems}.

\item Players should now examine their systems and their place in the cluster and add a final Aspect to each system to reflect their place in this implied web of trade and politics. Discuss the ramifications of these worlds and their placement --- who is the hub? Who controls technology? Can the resource-heavy worlds defend them? Do they need to?

\item Finally, the player generating the system should write a brief paragraph describing life in it.  This will get fleshed out further through character generation and further through play, so it's not necessary that it be comprehensive. A few questions that one might want to answer include:

\begin{itemize}
\item What does the sky look like here?
\item How does the average person live?
\item Why was this system colonized?
\item What has changed since then?
\end{itemize}

\end{enumerate}


\section{Linking Systems}\label{sec:linking-systems} % \href{sec:id72}

In Diaspora, systems are separated by unknown volumes of space --- their positions in the universe are so diverse as to be likely unknown and possibly unknowable. And yet they are connected into a tight cluster of only a few stars by some currently inexplicable laws of physics. These connections, the slipstreams, take only an instant to traverse, but in that instant vast quantities of heat accumulate and must be dissipated upon arrival.

\index{slipknot}
Slipstream points (slipknots) are located at a distance roughly 5 AU (astronomical units) above and below the barycenter, which is the point around which all bodies in the system revolve. How close you need to be to this point is determined by the technology level of your slip system --- a small device, but capable of translating the ship across unknown distances pre-determined by hidden geometries of our universe.

You can re-use your clusters as often as you like --- there is room in any one of them for more than one campaign --- but they are small enough that it's simple to build a new one every time you make characters if you prefer.


\section{Construction Sequence}
\label{sec:construction-sequence}

You have already created some number of Worlds, so now we're going to determine how they are connected. The Caller will draw a line of Systems, using the initial letter of each system name to identify it.

For each system, the owning player will roll four fudge dice.

\begin{itemize}
\item On a \emph{negative} result, connect the system to the next neighbour in the line.
\item On a \emph{zero} result, connect the system to the next neighbour in the line as above, but also, if a system further down the list has no connections, connect to that neighbour.
\item On a \emph{positive} result, do all that you do for a zero result and if another system further down the list has no connections, also connect to that neighbour.
\end{itemize}

Continue for each system until all systems are connected. The second to last system never needs a roll --- it will always connect only to its next neighbour.

That's your cluster! It will have natural hubs and relationships between systems with positive and negative resources. Each world is connected by one to five links to other systems. Next we will discover which worlds can exploit these links and which will have to pay to engage in interstellar trade.

\section{System Attributes}
\label{sec:System Attributes}

\begin{description}
\item[Technology]
Because of the nature of the game, the technology scale is privileging space exploration technology over other technological advances. Associated with each tech level is a whole host of other technological advances, and these may be extrapolated in the world design, and may be clarified with Aspects.

\item[Environment]
Generally a high-environment system is going to see vast immigration. How the local system inhabitants feel about that will drive regional politics and adventure.

\item[Resources]
The resource value of a system is what drives the economy. It tells you if the system is economically dependent on other systems, or if it is supporting them. In order to cultivate a system, invent the flow of trade in this way: every system with a R-2 or less is getting something from somewhere, and every system with an R2 or more may very well be the source. Knowing what these economic factors are should create plenty of room for competing interests and establish
some conflicts between systems.
\end{description}


