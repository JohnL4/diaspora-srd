\section[Wargaming]{Space Combat Wargaming}
\label{sec:space-combat-wargaming}

This combat Sequence is presented in a form sufficient to play independently as a wargame. The core design philosophy that makes this a wargame is that the combat is chiefly between spacecraft. That is, rolls are based on ship statistics (which function as analogues of \Skills) and ship \Aspects\ are invoked, tagged, and compelled. The role of individuals, even player characters, is for the most part ancillary to actual combat.

When playing this mini-game as part of a role-playing game, however, the range of action for players extends to their characters and the spaceships in play. Specifically, at any point in the combat Sequence, players should feel free to have their characters do things to influence events. Chiefly this will be a maneuver --- a \Skill\ check, possibly opposed, intended to place a free-taggable \Aspect\ on an enemy vessel. This is not part of the combat Sequence because it depends on the creativity and judgment of the players rather than on a strict application of rules, and consequently sits firmly in the space of the role-playing game, with final authority residing with the table, the caller, or perhaps the referee as appropriate.

As a stand-alone game, ships can be pitted against one another, without the need for player characters. Ships may be drawn from the lists below, or may be designed from scratch. Assuming ships all have standard (default) crews (i.e. they do not possess the Stunt, \stunt{Skeleton Crew}), any of the following basic scenarios should be playable:
\begin{description}
\item[Duel]
two ships at identical technology levels attempt to take each other out.
\item[Border Patrol]
T3 civilian ship seeks to escape two T2 military ships.
\item[Pirate Attack]
T2 ship attempts to take out another T2 ship which may not fire until the pirate has fired or initiated \skill{EW}.
\item[Smuggling]
T2 civilian ship seeks to escape T2 military ship.
\end{description}

\subsection{Aspects and Fate Points}
\label{sec:Aspects and Fate Points}

Each ship should have five \Aspects, revealed to all at the table, which need not be the \Aspects\ given in the ship list. Each ship also begins with five fate points.

\subsection{Crew}
\label{sec:Crew}

All crew positions are assumed to start at Skill level 2. \skill{Pilot} is not automatically raised to match the \Vshift. Players may spend additional points to raise the \Skill\ level of a given crew position. Ships receive between three and six points to spend: the base is 3. Add 1 for T3 vessels. Add two if the ship is military (all ships are military unless they have the \stunt{Civilian} Stunt). Points may be spent to raise the value of the crew position by one or to make a \Skill\ Military-grade (most often with \Pilot\ or \Communications\ on military ships).

