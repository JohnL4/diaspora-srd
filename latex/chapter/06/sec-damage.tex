\section{Damage}\label{sec:Damage}
\vfil

Shifts for a given attack are calculated by the difference between an adjusted attack roll and an adjusted defense roll.
%
\Consequences\ can be used to buy down these shifts before they are applied as damage. A mild \Consequence\ reduces the number of shifts by one, a moderate \Consequence\ reduces the shifts by two, and a severe \Consequence\ reduces the shifts by four. A ship may only have a maximum of one \Consequence\ of each severity. A ship can have no more than three \Consequences\ total regardless of the type of track the \Consequence\ was on: \Frame, \Data, or \Heat. The actual \Consequence\ is named by the defender.

A \Consequence\ is also an \Aspect\ and can be free-tagged (once) by any opponent at any time after it is applied, and tagged or compelled as usual thereafter.

Once the number of shifts has been modified by \Consequences\, the corresponding box on the specified stress track is marked and so are all open boxes below it. If the corresponding box has already been marked, the stress ``rolls up'': mark the next higher open box. If shifts or roll-up mean you must mark a box off the high end of the stress track, the ship is \TakenOut.

Unlike \Consequences, the attacker narrates the final state of the defeated ship (exploding in a blaze of glory; empty derelict; captured), subject of course to final approval by the referee. This may be a reason for the crew to voluntarily take themselves out before it comes to this, either by surrendering, agreeing to a tether, or jumping into the lifeboats!

This also means a ship can be \TakenOut\ without ever taking a \Consequence\ and therefore without ever taking any serious damage! A ship that takes eight shifts more than its \Frame\ stress track cannot be saved. This is the canonical piracy success: the winner chooses to narrate the \TakenOut\ result as surrender and an undamaged ship is captured.

If relevant conditions are met, \TakenOut\ may also include an enforced Tether as a step towards coupling (see Special Maneuvers). A ship that is \TakenOut\ cannot be used for positioning advantage in subsequent turns; it's usually best simply to remove its counter from the map.

\Consequences\ can be compelled, tagged, or invoked just like any other \Aspect. Their description is up to the controlling player but must obviously appear to be negative and meet with the table's approval as a suitable description of the effects on the vessel. And remember, if you forced the \Consequence\ you can tag it once for free! Remember that at any time during the fight but before damage is marked, any spacecraft owner may negotiate a \Concession\ rather than play out.

\subsection{Recovering Stress Box Hits}
\label{sec:Recovering Stress Box Hits}

Stress box hits are not real damage. They are alarms going off, rattled crew, shrapnel dinging off the hull, shutting down non-essential systems, blowing air to avoid explosive decompression, and so on. Nothing that can't be fixed with a tiny amount of downtime and nothing that actually affects performance.

Remember that \Heat\ track hits may be cleared by not using drives for a turn. Each turn that the \Vshift\ is not engaged allows the highest filled box in the \Heat\ track to be cleared.

All stress box hits are cleared at the first instance of downtime, whether that's time in dock or just the time in transit to the slipstream.

\subsection{Recovering Consequences}
\label{sec:Recovering Consequences}

In calculating time to repair, there are two scales that must be kept in mind. First, is in-game time. Repairs take time, and this has to be modeled somehow. More important, however, is living with the repercussions of a space combat in real time, from the player's perspectives. While clearing a mild \Consequence\ is possible right after combat, and a moderate \Consequence\ can be repaired with any respectable facility, severe \Consequences\ should be felt by the players; they should realize that they have seriously hurt their ship.

Consequently, the soonest that a serious \Consequence\ can be removed is at the end of the session following the one when it was received.  If the ship is being used, players must carry the effects of severe damage for at least one full session in addition to the session it is received, during which time the \Consequence\ should be continually compelled by the referee.

A mild \Consequence\ can be repaired by an engineer or computer expert (depending on the type of \Consequence) without a roll after the combat scene is over.

A moderate \Consequence\ remains until the engineer or computer expert can make a successful check against difficulty zero. Base time for repairs is a week with (positive or negative) shifts modifying the time to repair by one per shift. It requires docking at a repair facility within one technology rating of the ship. If the only repair facility available is of an inappropriate technology rating, treat the \Consequence\ as Severe for repair purposes.

A severe \Consequence\ can be repaired by an engineer or computer expert against a difficulty of 4. It requires docking at a repair facility within one technology rating of the ship (though the referee may decide that the facility is, despite technology, better or worse equipped to repair the vessel and apply this as a modifier to the difficulty). Repairing a severe \Consequence\ also forces an extra maintenance check. Regardless of when the maintenance work is done, the \Consequence\ is only removed at the end of the session after the one in which it was received. If the referee needs a guideline for the time of repairs, he may say it takes one month, modified by the number of shifts achieved; such time pressures, however, do not outweigh the need for the severe \Consequence\ to be borne for a full session.

