\section{The Sequence}\label{sec:social-combat-sequence}

Combat occurs according to a strict sequence of events, just as with the other combat systems. In order to run the Sequence, one player should be named the caller (usually the referee, but this is not essential). The duty of the caller is to run the Sequence: he ensures that each phase is given sufficient time and that there is a smooth pace as phases proceed. The caller should have the Sequence sheet in front of him during the game, and should keep track of the most recent skill used by each character.


\begin{sidebox}{The Sequence}
The Caller establishes the order characters will act in, then iterates over the following sequence, one character at a time:

\begin{enumerate}
\item Declare an action (and potentially a target).
\item \Compels{} from the table.
\item Free move (optional).
\item Roll \dplusskill{}.
\item Apply \Aspects{} and spin (player and table).
\item Action resolution and narration.
\end{enumerate}

When all players have had a turn, check timer box and determine whether victory conditions are met.
\end{sidebox}


To begin with, the caller will establish the order in which players will be polled for their actions. The best possible way to do this is the simplest way the table can all agree on: a random order proceeding clockwise, starting with the highest \skill{Charm} and then clockwise, or descending order \skill{Charm} (or whatever social Skill is most relevant). 

\begin{enumerate}
 \item Declare

For each player, the caller will ask for an action. Actions can be one of the following:

\begin{itemize}
\item Move
\item Composure attack
\item Obstruct
\item Maneuver
\item Move another
\end{itemize}

If the player is running multiple characters (as might well be the case if he is the referee), each of these characters should declare and resolve their actions separately as though run by different players.

\item Compel

Once the player declares his character's action and target, the caller will ask the table for compels. A compel can involve any of the acting character's \Aspect{}s, any \Aspect{} on his equipment, any \Aspect{} on the zone he is in, or any \Aspect{} on the scene. Anyone wanting to compel should hold up a fate point token and name the \Aspect{} being compelled. The caller will verify that it is a legitimate \Aspect{} for a compel and the acting player can either accept the fate point (and thus the compel) or pay the compelling player's character a fate point and deny the compel.

If a compel is accepted by the player, go to the next character (possibly one run by the same player).

\item Free move

Next the caller will ask the player to make his free move. The player may then move his character a single zone if he wants to.

\item Skill check

\rulebox{Characters in social combat may not use the same Skill twice in a row.}

The caller will then ask the player what Skill will be used for his action. This may not be the same skill used in the previous round.
Each action requires a 4dF + Skill roll to resolve. 

\item Aspects and Spin

Once the dice are on the table, Aspects may be invoked or tagged by all participating players as appropriate. The usual rules for tagging Aspects apply: you may tag only one of each category of Aspect except for free-taggable Aspects, of which you may tag as many as are available. A tagged or invoked Aspect adds 2 to the roll or allows a re-roll.

During the Aspect tagging, the caller will offer all players any spin that's on the table in order to improve their rolls. It can be spent to add one to a roll.

\item Resolution

Once all negotiable dice modifications are complete, the caller announces the resolution of the roll (who won) and directs the appropriate player to narrate the result. The authority to narrate depends upon the action declared --- see below for details.

\end{enumerate}

When all players have had a turn, the caller then checks a box on the timer and determines whether the victory conditions have been met. If there is a victory, he announces it and hands control to the referee. If there is no victory, he begins the next turn.

\subsection{Move}\label{sec:Move}

For a move action, the player rolls 4dF + Skill, then modify by any Aspects tagged or invoked. He may then move his character this many zones, expending movement points as needed to erode any pass values that might be on borders between his character and his goal.

The move action represents the character aligning himself with his interests (moving towards a target zone) or feigning alignment with another in order to be more effective (moving closer to another in order to reduce range modifiers).

\subsection{Composure attack}\label{sec:Composure attack}

A Composure attack is an effort to remove a character from play altogether by attacking his Composure stress track until he is Taken Out. To begin, the acting player names the target of the attack. The attacker names his attacking Skill and the target names the Skill he will use to defend. Both will narrate their efforts or otherwise justify the Skill selection.

Both players then roll 4dF + Skill and modify through Aspect tags, invokes, and spin. Count the attacker's shifts and then reduce the shifts by the range between characters. The defender may reduce these shifts using Consequences. He may reduce the shifts by one by taking a mild Consequence, reduce by two by taking a moderate Consequence, or reduce by four by taking a severe Consequence. He may apply more than one Consequence if necessary. Each Consequence becomes a free-taggable Aspect on the character.

The remaining shifts are then used to mark the defender's Composure stress track: one box on the track is marked at the rank according to the shifts and all open boxes below it (one shift marks the first box, three shifts marks the first, second, and third box, and so on). If the highest box to be marked has already been filled, then the next highest available box is filled. If the box to be filled is past the end of the character's Composure stress track, then the character is Taken Out. The two players should negotiate what this means, mediated by the referee.

If the attacker fails his roll by three or more (gets three or more negative shifts), the defender gets spin.

The Composure attack represents an attempt to remove a character from play by making her ineffective. It might be an embarrassing anecdote designed to shame the character into removing herself from the scene, or a stinging insult that makes her too angry to act with the social subtlety necessary to participate in this kind of combat. Or something else.

\subsection{Obstruct}\label{sec:Obstruct}

The player obstructing begins by identifying the zone that will be obstructed. He then rolls 4dF + Skill - Range, modified by any Aspects tagged or invoked. If he obtains a positive result, he may apply the number of shifts as pass values on any borders in the zone. The total of all pass values added cannot exceed the number of shifts. So, if a player generated four shifts he could create a single pass value of four on one border, or a pass value of three on one border and one on another, or any other combination of pass values adding up to no more than four.

The obstruct action represents efforts to pin a character into his current mind-set or deflect him from ideas that would be contrary to the acting character's interests. This might be attempts at levity in order to block off a more sober zone, awkward geek behaviour in order to make it harder to get into an intimate zone, or similar.

% \vfil

\subsection{Maneuver}\label{sec:Maneuver}

The player maneuvering begins by identifying the target of the maneuver. This target is typically a zone, but may be a character or the entire scene. He then announces the Aspect he intends to put on  the target and narrates the effort. He then rolls 4dF + Skill, modified by any Aspects tagged or invoked. If he obtains a positive result, the target acquires an Aspect described by the acting player. This Aspect is free-taggable once by any ally. Putting an Aspect of \aspect{Long-winded anecdote} on a zone will give other players a reason to avoid that zone, lest they be mired in a boring conversation, and unable to escape.

Permanent Aspects are Aspects that affect the person or zone directly. This includes things like \aspect{Liar}, \aspect{Out of crudit\'ees}, and so on. Transient Aspects are Aspects that derive from the continuous action of an individual. \aspect{Making socially unacceptable small talk}, \aspect{Spitting}, and so on. Transient Aspects last only until the placing character acts again, though he may use the Aspect in this last turn of its existence.

The caller determines whether a given \Aspect{} is permanent or transient.

% badbox here
% \newpage

\subsection{Move Another}\label{sec:Move Another}

The move another action is an attempt to force another character to move along the social map in a direction desired by the attacker. To begin, the acting player names the target of the attack. The attacker names his attacking Skill and the target names the Skill he will use to defend. Both will narrate their efforts to justify the Skill selection.

Both players then roll 4dF + Skill and modify through Aspect tags, invokes, and spin. Count the attacker's shifts and then reduce the shifts by the range between characters. These shifts are then used to move the defending player: one zone or pass value per shift, exactly as a move action.

If the attacker fails by three or more shifts, the defender is awarded a spin token.

The move another action is a careful effort to persuade. It represents effective rhetoric, brilliant argument, seduction, and like forms of persuasion. The acting character is trying to manipulate the target character directly.

