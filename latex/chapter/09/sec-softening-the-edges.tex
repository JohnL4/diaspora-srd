\iflandscape{}{\vfil}
\section{Softening the Edges}\label{sec:Softening the Edges}

Now, as Diaspora is trying to build a ``hard science-fiction'' atmosphere (even if we have incorporated faster-than-light travel), it may be seen as out-of-subgenre to include aliens, psionics, etc. Far be it, however, for us to declare them off limits! The abstraction provided by the underlying FATE system is such that, mechanically, these things are not interestingly different.

% ~
% \newpage
% ~

\subsection{Non-Human Races}\label{sec:Non-Human Races}

\index{races, non-human}\index{aliens|see{races, non-human}}
If you want to create a non-human race, start by deciding what's really different, in game mechanical terms, between them and the total range of abilities to be found among humans. Designing a balanced race really means not going beyond these parameters, which, in the largely abstract system presented here, allows for a seven-point range between -1 (untrained) and 5. Wording of Aspects can increase this effective range (both for humans and for the aliens) by and additional +/- 2 (creating a modified eleven-point range, -3 to 7).

Non-human races are best modeled by augmenting the existing rules as subtly as possible --- that is, the best aliens for the purposes of Diaspora are going to be those with few game mechanical differences no matter how bizarre the story behind them might be: you might introduce a new Skill that represents some feature specific to these aliens and maybe even require that all members of the species have it, but we recommend resisting structural changes to how a character is created.

Formulating a non-human race (or converting one from another source) may then be framed in terms of general statements that are applicable when measured against a normal human. They may suggest some Skills should be higher than others, or that some should be lower, and Stunts that would be more common in the race than in humans. These differences form the mechanical description of the race. These differences do not change the number of Stunts or Skills or the shape of the Skill pyramid; they only describe certain features of the choices one would make within those constraints in order to create a believable alien of that type.

The parameters for creating members of other races lie completely with the discretion of the player, subject to referee and table approval, of course. All choice, of course, remains with the player, as there are always exceptions to any general rule.

Optional: All aliens may be thought to have an automatic Aspect, \emph{not human}, to be compelled liberally whenever something built for or designed by humans is encountered.

% \newpage

% ~

\subsection{Psionics}\label{sec:Psionics} % \href{sec:id215}

\index{psionics}
With the referee's permission, Psionics can be introduced. The precise mechanics would require table consensus, but what is important is that the balance between characters not be thrown off. Anything should be able to be modeled with an appropriate balance of Skills, Aspects, and Stunts.  If something is happening that will be seen as exceeding human norms, it should require investment in both a Skill and a Stunt. Typically, there will also be an appropriate Aspect.

What follows are ten ideas for how some psionic abilities could be modeled within the game system. In each of them we are assuming a common Skill powering all the abilities, Psionics; it may be desirable to have different Skills: one for Telepathy, one for Teleportation, etc., in a Psionics-heavy campaign. Some are just maneuvers, some are stunts, and some are alternate ways to use skills.

\begin{description}
\item[Awareness] ~

add a Stunt ``Swap a skill'' to use Psionics for Alertness, with +2 to the roll with the restriction that it may only detect the existence of sentient minds. (Perhaps making it military-grade would allow an clairvoyant image, a flash of what a detected mind sees?)

\item[Mass Suggestion] ~

add a Stunt ``Swap a Skill'' to use Psionics for Oratory.

\item[Psychic assault] ~

select \emph{Have a thing} as a Stunt to get Integral Equipment: Knife, but let it be a psychic attack (perhaps Composure damage only, in exchange for +2 penetration).

Rather than tied to brawling Skill, it could be powered by a new Skill ``Psionics.'' Some range could be modeled perhaps by allowing attacks at range 0-1, but doing only damage 1.

\item[Regeneration] ~

add a Stunt ``Swap a skill'' to use Psionics for Medical.  Because age is not modeled in Diaspora (all characters have the same number of Skills, and players determine the shape of their own Skill pyramids), it costs nothing to add descriptive features to this such as, ``eternally young,'' because there are no associated mechanical effects. If this is the character the player wants, let them have it, as long as there is some character investment (perhaps through an Aspect).

\item[Shield] ~

want to resist a psychic assault? Take an Aspect ``Psionic shield'' which will give you +2 to any defense, as long as you are willing to spend a fate point; fortunately, the investment need only come after the dice have hit the table, and you might win without any investment!

\item[Telekenesis] ~

add a Stunt ``Swap a Skill'' to use Psionics for Agility.  (Perhaps making it ``Military-grade'' would allow an initial range of 1 or 2 zones, instead of zero; that would then represent a player investment of two Stunts and a Skill).

\item[Telepathic suggestion] ~

As a maneuver, use Psionics, to put an Aspect on nearby individuals to give a free-taggable Aspect to a future action by yourself or an ally.

\item[Telepathy] ~

with the Psionics Skill, take the  Stunt ``Swap a Skill'' to use Psionics for Intimidation (the victims are always aware you are reading their mind, and they may attempt to resist with Resolve).

\item[Technopathy] ~

do you want to communicate with machines? Add a Stunt ``Swap a Skill'' to use Psionics for Computer, or for Repair (or, with two Stunts, both!).

\item[Teleportation] ~

short-range, line-of-sight teleportation can be modeled with Psionics, which replicates the movement bonuses possible from Agility.
\end{description}

% ~

Some campaigns may wish to model the psychic toll that using such abilities represents. If so, each use of Psionics that achieves a Superb (+5) or better result costs a fate point, in addition to any points spent ensuring the success. Since the restriction for players will be the same as it is for any characters run by the referee, such a restriction, while onerous, should ultimately favour the players who have greater resources of fate points.

% ~

% ~

\subsection{Landing a Spaceship}\label{sec:Landing a Spaceship} % \href{sec:id216}

Spacecraft design in Diaspora precludes the possibility of landing a
spacecraft easily: they are simply not built to balance on their tails
in an atmosphere. If transferring between orbit and the surface is a
plot your table doesn't want to tell, however, there are options.

Given that an interface vehicle costs one build point in the spaceship
design process (see next section), a table might decide that any ship
that invests two build points can land in an atmosphere. The choice to
be able to land a ship (avoiding highports, orbital stations, any
surface-to-orbit transfer, etc.) is offset by the additional
investment in frame construction. The most capable ships will continue
to be those specialized vessels without the ability to land.

