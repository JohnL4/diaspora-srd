\section{Getting Tactical}
\label{sec:getting-tactical}

The tactical mini-games have a particular ability to help the referee when things aren't going well at the table. Sometimes you will find that the players are not making progress against a puzzle or a secret and instead are circling the problem with planning and shopping. Sometimes this is fun, but when you detect that it no longer is --- that players are getting bored with the lack of effect --- the mini-games can solve this instantly.

In particular, social combat can be applied to almost any issue (a heist, a kidnapping and ransom, removing an aristocrat from power, or even influencing the trade policies of a whole system) and so it's a handy way to push a table from planning to action.

And the joy of the combat system is exactly there: you don't need to plan and prepare. You just need to get the antagonists and protagonists on a credible map and then turn the rules like a crank.

Part of the reason this works is because it spurs action, but it also works because it forces you, the referee, to start partitioning the issue: you need to decide who the protagonists are in the context of the conflict, who the antagonists are, who is relevant to the outcome but not an agent, what the actual goals of each party, and how to represent that graphically as a combat map. All that tightly-regulated thinking will make the situation crystal clear.

\ifthenelse{\boolean{LANDSCAPE}}
{\newpage}{}
The mini-games with actual injury at stake (space, personal, and platoon) are even more focused and are more obvious to deploy. This is in line with advice from Spirit of the Century:

\rulebox{When in doubt, ninjas!}

The reason combat works is because even if the players lose, a result is forced out of the situation. Even in the worst possible situation it doesn't have to be death and an ending, and yet it can be, which is powerful.

Concessions play a big part in this. When things are going badly for the players, suggest that they concede. Have them offer something to get away (or, cooler, to save their friends) and suddenly failure is interesting, too.

% \newpage

And that chaining also helps: conceding a space combat might lead to a boarding action with personal combat. Conceding personal combat might lead to a trial using social combat. Losing a platoon combat might lead to a chase scene in orbit using space combat. You could be here all night turning this crank and creating stories.

