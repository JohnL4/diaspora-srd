\iflandscape{}{\newpage}\section[Wargaming]{Wargaming}
\label{sec:personal-combat-wargaming}

Sometimes it's fun just to make one-off characters and have them shoot at each other. To play independently as a tactical war game, you need three things: a map, a story, and characters.

\subsection{The Map}
\label{sec:personal-combat-wargaming-map}

Someone is chosen as caller. Either the caller or the table draws a map. Is it a shoot out in an airport? A race to secure a bunker at the top of a hill? A boarding action in a submarine or a spaceship? Whatever the case, you need a map to play on.

You can start with a blank piece of paper, and take turns drawing features, until it looks good enough. Feel free to write words on the map too --- these can become Aspects and help clarify what's what.

Once that is done, divide the map into zones. You don't want too many, but enough to allow opportunities for getting outside of range, and to allow movement. When drawing zones, it is often helpful to go from corner to corner: that means it is always clear when a character enters an area (from a door, or otherwise along a side) what zone he is in.

\subsection{The Story}
\label{sec:personal-combat-wargaming-story}

The process of drawing a map has already begun to determine what the story is: is this a fight to the death? Are there teams? Is most of the table maneuvering against a small cadre controlled by the caller (or by someone else)? Is there a difference in tech level between two sides? Whatever the case, articulating the story that is being told might mean that you go back and change the map slightly, add an Aspect to a zone or two, or whatever.

Most important is that the story articulates victory conditions, which need not be the same for all players. Is this a fight to the death? An attempt to capture someone alive? Someone working to escape detection and get out of a building, or sabotage a spacecraft's drives? Whatever the case, the victory condition might be defined in terms of time: get off the ship in eight turns; spend two turns alone in the engine room setting explosives.

\subsection{Characters in Wargaming}
\label{sec:characters-in-wargaming}

\iflandscape{\iflandscape
{\begin{wraptable}[14]{r}[0.9\sidebarwidth]{5cm}}
{\begin{wraptable}[14]{r}[0.9\sidebarwidth]{5.5cm}}
% \begin{table}[ht]
\centering
\begin{tabular}{ll}\toprule
Skill & type \\
\midrule
Agility		\\
Alertness	\\
Brawling	& combat \\
Close Combat	& combat \\
Energy Weapons	& combat \\
EVA		\\
MicroG		\\
Resolve		& track \\
Slug Throwers	& combat \\
Stamina		& track \\
Stealth		\\
Tactics		\\
\bottomrule
\end{tabular}
\caption{Useful skills for personal combat wargaming}
\label{tab:personal-wargaming-skills}
% \end{table}
\end{wraptable}
}{}

Once the map and the story are determined, everyone should spend five minutes (no more) making one or two characters to push around the map.

\subsubsection{Skills}

Given the limited focus of this tactical game, 3-cap characters should be sufficient: pick one Skill at level 3, two at level 2, and three at level 1. Everything else is considered untrained. While any Skill might be taken, \autoref{tab:personal-wargaming-skills} presents Skills particularly relevant to this mini-game.

% \newpage

\subsubsection{Stress Tracks}

\iflandscape{}{\iflandscape
{\begin{wraptable}[14]{r}[0.9\sidebarwidth]{5cm}}
{\begin{wraptable}[14]{r}[0.9\sidebarwidth]{5.5cm}}
% \begin{table}[ht]
\centering
\begin{tabular}{ll}\toprule
Skill & type \\
\midrule
Agility		\\
Alertness	\\
Brawling	& combat \\
Close Combat	& combat \\
Energy Weapons	& combat \\
EVA		\\
MicroG		\\
Resolve		& track \\
Slug Throwers	& combat \\
Stamina		& track \\
Stealth		\\
Tactics		\\
\bottomrule
\end{tabular}
\caption{Useful skills for personal combat wargaming}
\label{tab:personal-wargaming-skills}
% \end{table}
\end{wraptable}
}

Characters should only concern themselves with the \Health{} and \Composure{} stress tracks. Each is three boxes long. If the character has \skill{Resolve} at level 1 or 2, the \Composure{} track has four boxes; if he has \skill{Resolve} 3, the \Composure{} track has five boxes. If the character has \skill{Stamina} at level 1 or 2, the \Health{} track has four boxes; if he has \skill{Stamina} 3, the \Health{} track has five boxes.

\subsubsection{Stunts}

Every character selects a Stunt. Making something Military-grade or altering how a stress track works are both obvious choices. (For some stories, it may be desirable to allow two Stunts per character; that's fine, as long as it's the same across the board).

\newpage

\subsubsection{Aspects}

Each character should have three Aspects, revealed to all at the table. Each character also begins with three fate points.

Making a note card for each character, placed in front of the player with all the relevant information and a small pile of fate points stacked on top keeps all the information clear at all times. This is obviously scaled back from the RPG, and introduces a slightly different calculus for what constitutes a success. With reduced characters, teamwork, particularly in laying down maneuvers to be free-tagged, is rewarded.
