\subsection{Artillery}
\label{sec:Artillery}

Artillery units are equipment capable of Indirect Fire which are kept off map. They move only in a notional sense insofar as they can roll Movement as a defensive roll against counter-battery detection and Indirect Fire. Infantry-based artillery (such as mortar or grenade launcher crews) should be represented by including an Indirect Fire Skill on an infantry unit. Artillery batteries that need to be represented on the map for purposes of the scenario should be represented by their attending personnel as lightly armed infantry units.

It can be handy to create an off-map artillery card for artillery platoons, especially if they have a command range greater than one. This will greatly simplify aircraft attacks on the artillery platoon. Artillery has a 3-cap Skill column. Artillery units have one Skill at rank 3, one at rank 2, and one at rank 1. Their Morale track is two boxes long. If the unit has Veteran Skill 1 or 2, the Morale track is three boxes long. If the unit has veteran Skill 3, then the Morale track is four boxes long.

Artillery may make Movement rolls to change position on their battery
card if there is more than one zone on the card. Moving artillery
units do not remove \SPOTTED\ markers, however.

Artillery can only fire on targets that are in line-of-sight to a
friendly unit that is currently attached to a platoon (or does not
need to be) and has no Out Of Communications (\OOC) counters.

All artillery units in the same platoon are considered to be in the
same zone as their leader for purposes of command and communication,
and for purposes of any attacks that affect all targets in a single
zone when attacked by aircraft, unless they have a Stunt that allows a
greater command range. All members of an artillery platoon not
situated on the map must be on the same artillery card (they cannot be
spread over multiple cards). Not all units in an artillery platoon
need to actually be artillery units (there might be an infantry leader
unit supplying comms and other coverage and an armour unit supplying
AA for example).
Non-artillery units in an artillery platoon must be in the off-map
artillery card in order to be associated with the platoon. This means
that, although the leader unit might be represented by something other
than artillery (an armour or infantry unit might be more advantageous)
it will gain no advantages from its mobility on the map.

