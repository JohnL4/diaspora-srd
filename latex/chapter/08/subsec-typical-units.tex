\subsection{Typical Units}
\label{sec:typical-units}

%%% macros for unit stat blocks
% parameters:
% - * indicates we need to output a tikz block
% - leader (+)
% - unit name
% - platoon & unit number
% - aspect
% - skill list
% - stunt list
% - movement
% - morale
%% We assume this is already inside a tikz block
\DeclareDocumentCommand \infantryblock {S{+}mmmmmmm} {
  \def\unitskills{#5}
  \def\unitstunts{#6}
  \unitblock{Infantry}{#2}{#3}{#4}
  \unitskillsblock{(-1.25,3.4)}\unitskills
  \unitstuntsblock{(0.75,3.3)}\unitstunts
  \unitmovementblock{(1.75,1)}{2}{#7}
  \unitmoraleblock{}{2}{#8}
  \unitleaderblock
}
\DeclareDocumentCommand \artilleryblock {S{+}mmmmmmm} {
  \def\unitskills{#5}
  \def\unitstunts{#6}
  \unitblock{Artillery}{#2}{#3}{#4}
  \unitskillsblock{(-1.25,3.4)}{\unitskills}
  \unitstuntsblock{(0.75,3.3)}{\unitstunts}
%   \unitmovementblock{(1.75,1)}#7
  \unitmoraleblock{}{2}{#8}
  \unitleaderblock
}
\DeclareDocumentCommand \armourblock {S{+}mmmmmmm} {
  \def\unitskills{#5}
  \def\unitstunts{#6}
  \unitblock{Armour}{#2}{#3}{#4}
  \unitskillsblock{(-1.25,3.4)}{\unitskills}
  \unitstuntsblock{(0.75,3.3)}{\unitstunts}
  \unitmovementblock{(-0.4,1)}{4}#7
  \unitmoraleblock{}{1}{#8}
  \unitleaderblock
}
\DeclareDocumentCommand \aircraftblock {S{+}mmmmmmm} {
  \def\unitskills{#5}
  \def\unitstunts{#6}
  \unitblock{Aircraft}{#2}{#3}{#4}
  \unitskillsblock{(-1.25,3.4)}{\unitskills}
  \unitstuntsblock{(0.75,3.3)}{\unitstunts}
%   \unitmovementblock{(-0.4,1)}{4}#7
  \unitmoraleblock{}{1}{#8}
}
\DeclareDocumentCommand \unitblock {mmmm} {
  \def\unittype{#1}
  \def\unitname{#2}
  \def\unitnum{#3}
  \def\unitaspect{#4}
  \unitblockborder
  \unitheaderblock{\unittype}{\unitname}{\unitnum}{\unitaspect}
}
\def\unitblockborder{
  \draw [rounded corners] (-3,0) rectangle +(6,5.5);
}
% identifying information
\def\unitheaderblock#1#2#3#4{
  \draw [rounded corners=3pt]
    (0,5.1) [anchor=base east] node {\bfseries #1 Unit:}
      [anchor=base west] node{\sffamily #2} ++(0,-0.1)--++(2.2,0)
    (-0.5,4.55) [anchor=base east] node {Platoon/ID:}
      [anchor=base west] node{\sffamily #3} ++(0,-0.1)--++(1.25,0)
    (-1.25,4.0) [anchor=base east] node {Aspect:}
      [anchor=base west] node{\sffamily #4} ++(0,-0.1)--++(4,0)
    ;
}
% Leader checkbox section
\DeclareDocumentCommand \unitleaderblock {S{+}}{
  \draw [rounded corners=3pt]
    (2.95,4.7) [left] node {Leader?}
      [below left] node {\scriptsize (+2 morale)}
      ++(-0.1,0.2) rectangle +(-0.5,0.5)
    ;
  \IfBooleanT{#1}{
  \draw [line width=1]
    (2.4,5.15) -- +(0.1,-0.25) -- +(0.4,0.15)
    ;
  }
}
% morale boxes
\def\unitmoraleblock#1#2#3{
%   \typeout{Morale:~#1~#2~#3}
  \draw [rounded corners=3pt]
    (-2.85,0) [above right] node {morale}
    \foreach \x in {1,...,#2}
      { ++(0.2,0.5*\x) rectangle +(0.5,0.5) }
    ;
  \filldraw [rounded corners=6pt,fill=gray]
    (-2.65,0) ++(0.0,0.5*#2)
    \foreach \x in {1,...,#3}
      { ++(0,0.5*\x) rectangle +(0.5,0.5) }
    ;
}
% skills
\def\unitskillsblock#1#2{
  \draw #1
    [anchor=base west] node {Skills}
    \foreach \snum/\smod/\snam in #2 {
      \skillline{\snum}{\smod}{\snam}
    }
    ;
}
\def\skillline#1#2#3{
    ++(0,-0.5*#1)
      [above left] node {#2}
      [above right] node {\scriptsize #3}
      +(0,0.1) -- +(1.75,0.1)
}
% stunts
\def\unitstuntsblock#1#2{
  \draw #1
    [above right] node {Stunts} ++(0,0.1)
    \foreach \snum/\smod/\snam in #2 {
      \stuntline{\snum}{\smod}{\snam}
    }
    ;
}
\def\stuntline#1#2#3{
    ++(0,-0.5*#1)
      [above right] node {\scriptsize #2}
      +(0,0.1) -- +(2.0,0.1) node [above] {\scriptsize #3}
}
% movement
\def\unitmovementblock#1#2#3{
  \draw #1
    [above] node {Movement}
    +(0,-0.85) node {\scriptsize base #2}
    ++(-0.75,0)rectangle +(1.5,-0.75)
    ;
}

%%% list processing
%%% from http://maraist.org/comma-separated-lists-in-latex_08-2009/

% kill a single token: match it (removes it from the stream) and
% immediately do nothing with it.
% \def\@swallow#1{}
% %
% \def\forlist#1#2#3{%
% %   \typeout{DLM:~ forlist~ args~ `#1'~ `#2'~ `#3'}
%   \@NullTest@forlist{#1}{#2}#3\@end@forlist}
% %
% \def\@NullTest@forlist#1#2{%
%   \@ifnextchar\@end@forlist%
%     {#1\@swallow}%
%     {\@prime@forlist{#1}{#2}}}
% %
% \def\@prime@forlist#1#2#3\@end@forlist{%
%   \@test@forlist{#1}{#2}#3,\@end@forlist}
% %
% \def\@test@forlist#1#2{%
%   \@ifnextchar\@end@forlist%
%       {#1\@swallow}%
%       {\@more@forlist{#1}{#2}}}
% %
% \def\@more@forlist#1#2#3,#4\@end@forlist{%
%   #2{#3}{\@test@forlist{#1}{#2}#4\@end@forlist}}

%%% Another approach
\def\TERMINATOR{LISTTERMINATORSTRING}
\def\SkillList(#1)(#2){\xSkillList#1,LISTTERMINATORSTRING, }
\def\xSkillList#1,{\def\TMPlist{#1}%
%   \typeout{item: \TMPlist}
  \ifx\TMPlist\TERMINATOR\SkillListItem{#1}
  \else\SkillListItem{#1}%
    \expandafter\xSkillList
  \fi}
\def\SkillListItem#1{ %skill: #1 ~}
  \draw ++(0,-0.5) [above right] node {\scriptsize #1} ;
  }
%%%

%%% HACK approach
\DeclareDocumentCommand \HACKSkillList {mmmmmmm} {
  \typeout{HACKSkillList: #1 . #2 . #3 . #4 . #5 . #6 . #7 }
  \draw #1
%     ++(0,-0.5) [above right] node {\scriptsize #1}
    ++(0,-0.5) [above right] node {\scriptsize #2}
    ++(0,-0.5) [above right] node {\scriptsize #3}
    ++(0,-0.5) [above right] node {\scriptsize #4}
    ++(0,-0.5) [above right] node {\scriptsize #5}
    ++(0,-0.5) [above right] node {\scriptsize #6}
    ++(0,-0.5) [above right] node {\scriptsize #7}
  ;
}
\begin{sidebox}{Typical T-1 units}
\centering
\begin{tikzpicture}[scale=0.75]
\tikzstyle{every node}=[transform shape]
\begin{scope}
\infantryblock%
  {Marines}% name
  {B--1}% number
  {they'll never see us comin'}% aspect
  {1/+3/Camouflage,2/+2/Direct Fire,3/+2/Observation,4/+1/Armour,5/+1/Hand-to-hand,6/+1/Command}% skills
  {1/Special Forces/T-1,2/{}/T-2,3/{}/T-3,4/{}/T-4}% stunts
  {}% move
  {4}% morale
\end{scope}
\begin{scope}[yshift=-6cm]
\armourblock%
  {Light Tanks}% name
  {B--2}% number
  {Outfitted for rough terrain}% aspect
  {1/+4/Armour,2/+3/Direct Fire,3/+2/Movement,4/+1/Anti-Air}% skills
  {1/Light/free,2/{}/T-1,3/{}/T-2,4/{}/T-3,5/{}/T-4}% stunts
  {2}% move
  {4}% morale
\end{scope}
\begin{scope}[xshift=6cm,yshift=0cm]
\artilleryblock%
  {Mortar Team}% name
  {B--3}% number
  {Pin-point accuracy}% aspect
  {1/+3/Indirect Fire,2/+2/Camouflage,3/+1/Movement}% skills
  {1/Zone Effects/free,2/{}/T-1,3/{}/T-2,4/{}/T-3,5/{}/T-4}% stunts
  {2}% move
  {4}% morale
\end{scope}
\begin{scope}[xshift=6cm,yshift=-6cm]
\aircraftblock%
  {Bomber Squad}% name
  {B--4}% number
  {Smart-bomb payload}% aspect
  {1/+4/Direct Fire,2/+3/Observation,3/+2/Anti-air,4/+1/Movement}% skills
  {1/Long Range/free,2/{}/T-1,3/{}/T-2,4/{}/T-3,5/{}/T-4}% stunts
  {2}% move
  {4}% morale
\end{scope}
\end{tikzpicture}

% \forlist{\EmptySkill}{\SkillLineNode}{one,two,three}
%   \SkillList(one,two,three)

\end{sidebox}


\vfil

\subsubsection{Typical Infantry}

Skill tree: Camouflage 3, Direct Fire 2, Observation 2, Armour 1, Hand to-Hand 1, Command 1.

Infantry are used to capture and hold territory as well as to provide spotting for heavier units.  They have NCOs capable of regrouping broken units and are adept at close combat as well as ranged. They excel at not being seen.

\subsubsection{Typical Armour}

Skill tree: Armour 4, Direct Fire 3, Movement 2, Anti-air 1.

There are two core types of armour: assault tanks designed to move into heavy fire and attack units spotted by associated infantry, and tank hunters, which would swap Armour and Direct Fire.

\vfil

\subsubsection{Typical Artillery}

Skill tree: Indirect Fire 3, Camouflage 2, Movement 1.

Artillery's immediate objective is to destroy spotted enemy equipment. It does so by projecting a huge volume of fire, which makes it suddenly very vulnerable. It offsets this vulnerability by immediately moving and re-hiding.

\subsubsection{Typical Aircraft}

Skill tree: Direct Fire 4, Observation 3, Anti-air 2, Movement 1.

Aircraft Skill trees are capped by their primary design goal --- Direct Fire for ground attack vehicles, Observation for reconnaissance craft, and Anti-air for interceptors. Most aircraft will be capable in all of these. Movement for aircraft indicates their re-arm time --- high Movement rates indicate rapid re-arming cycles, trading off for specialty effectiveness such as ground attack or anti-air capability.
