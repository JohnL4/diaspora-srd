\section{The Sequence}\label{sec:the-sequence}

The Sequence proceeds in a fixed order around the table, with each player acting for each of her units in whatever order she prefers. The Sequence order is best left simple and static --- clockwise from the person on the left of the caller, say.

Identify a player to act as caller. The caller manages the Sequence. The player currently acting is the Actor.

Each player in turn selects a platoon that has not acted and checks its units for platoon membership. For each unit that is part of the platoon, perform an action. Finally, mark the platoon Acted. Each unit that acts performs some bookeeping tasks (reducing jam counters by one and clearing old interdiction markers). Units that are not in platoon membership can move one free move into any zone that does not contain enemy units, or fire on an enemy unit that fired on them, or unjam, or camouflage. If an unmembered unit is in command range of another platoon's leader, it may be adopted by that platoon. No fate points are added, no Consequences are transferred: all that happens is the unit is added to the adopting platoon.

When every player has had a chance to act with every platoon under his command, the turn is complete. So basically players are alternating platoons until the turn is done. The order of platoon selection is up to the players. The first player to act is the defending player. If the scenario has no logical defender, flip a coin.

A turn is a set of cycles through the sequence such that every player has had an opportunity to move every platoon under his control.

\vfil
% \newpage

\subsection{Move}\label{sec:platoon-combat-move}

\begin{sidebox}{Compelling Movement}

The objective of the rule that makes you pause for a compel after every zone moved, is to more closely model terrain effects and tactical \emph{funneling} by increasing vulnerability in long moves. This also makes ambushes work really well if planned effectively in appropriate terrain --- even though the enemy knows your \emph{hidden} units are there, he has to pay fate points to flee past a choke point safely or get fate points to stop there and take the ambush the \emph{player} knows is coming. This careful use of the player/character distinction is tricky but fun.
\end{sidebox}


Roll \skill{Movement} and move the number of shifts, up to the maximum movement permitted. If your movement places you in (or passes through) the same zone as an enemy unit, both you and it gain a \SPOTTED{} marker of value 3 and the moving unit ceases movement.

Some special movement rules for specific unit types:

\begin{description}
\item [Artillery]
roll \skill{Movement} against a \SPOTTED{} marker value and reduce its value by the number of shifts.

\item [Aircraft]
roll \skill{Movement} and move the number of shifts along the Re-arm track. Aircraft on their LAUNCH! box may be placed (without a roll) on any zone on the map or any battery zone off the map. Aircraft in a battery zone may be attacked exactly as spotted artillery, though only with \skill{Anti-air} Skills from a unit on the same artillery card as the spotted artillery unit.

\item[March in formation]
A platoon leader may choose to march his platoon in formation. He makes one \skill{Movement} roll and may move all of the units in his zone up to the maximum movement rate of the slowest unit in the zone. All of these units have now acted.
\end{description}

If movement takes a unit out of line-of-sight from all enemy units, reduce any \SPOTTED{} marker by one.

For each zone that a unit enters during its movement, an enemy may compel it to halt in that zone.

Moving is not affected by range or line of sight.


\subsection{Attack}\label{sec:platoon-combat-attack}

Add a \SPOTTED{} marker to the attacking unit, to a maximum of four. Roll your appropriate attack Skill against enemy \skill{Armour} Skill on any unit with a \SPOTTED{} marker, counting shifts as damage. Subtract range to target. Add the minimum range of the attack type (zero for \skill{Hand-to-hand}, one for \skill{Direct fire}, two for \skill{Indirect fire}). $-3$ shifts gains spin for the defender.

This extra arithmetic is basically saying that the range count, for purposes of modifying the attack roll, starts (as zero) at the minimum range of the weapon.

\skill{Indirect Fire} may not act at range zero or 1. \skill{Direct Fire} may not act at range zero. \skill{Hand-to-Hand} may ONLY attack at range zero. \skill{Anti-air} may attack at any range but recall that all ranges from ground to aircraft are increased by one. Range from aircraft to aircraft are counted normally. Artillery attacks targets without range modification. Attacks on artillery in an off-map ``battery zone'' are at effective range 10 zones when attacked by on-map units.

Before marking shifts as damage, the shifts may be reduced by applying Consequences. A platoon can have a maximum of three Consequences: one mild, which can reduce incoming shifts by one; one medium, which can reduce incoming shifts by two; and one severe, which can reduce incoming shifts by four. A Conesquence is a free-taggable Aspect on the platoon. A platoon may have no more than three Consequences --- one of each type.

A hit that would mark a box past the end of the unit's Morale track takes out that unit.

Attacking is affected by range and line of sight.

Remember that the defending roll, after modifications by invokes, stays on the table for the turn!

\subsection{Interdict}\label{sec:platoon-combat-interdict}

Select a target zone. Roll your appropriate attack Skill against target zero. Subtract range to target zone. Distribute shifts as pass values on any border for that zone (thus 4 shifts could place a pass value 2 on two borders, 4 on one border, 1 on 4 borders, or any other combination). Interdiction lasts until the beginning of the attacker's next turn. Interdiction attacks grant \SPOTTED{} markers exactly as attacks.

Interdiction is affected by range and by line of sight.

\subsection{Rally}\label{sec:platoon-combat-rally}

Roll \skill{Command} against highest Morale track hit to repair track hits by shifts (as repair on spacecraft). Can roll against any unit in the platoon.

Rally is not affected by range or line of sight.

\subsection{Jam}\label{sec:platoon-combat-jam}

Roll \skill{Signals} against another unit's \skill{Signals}. For each shift generated, place an \OOC{} counter on the attacked unit. Failure by three or more generates spin for the defender. Note that this attack is especially effective against a leader unit.

Jamming is not affected by range or line of sight. Targets need not be spotted. Units can have no more than four \OOC{} markers.

\vfil
\subsection{Unjam}\label{sec:platoon-combat-unjam} % \href{sec:id198}

Roll \skill{Signals} against zero and remove the shifts in \OOC{} counters from yourself. If this unit is a leader unit, it may remove \OOC{} counters from members of its platoon.

Unjamming is not affected by range or line of sight. It may be performed by units not currently in platoon membership.

% \vfil
\subsection{Maneuver}\label{sec:platoon-combat-maneuver}

Roll any Skill and supply some narration to describe the effect. Subtract range to target zone. Success places an Aspect on a zone. The Aspect is free-taggable once by an allied unit. Use maneuvers to model artillery cratering (\aspect{Cratered}), forward observation (\aspect{Laser designation}), covering fire (\aspect{Keeping our heads down}), and so on. Maneuvers that use \skill{Direct Fire}, \skill{Indirect Fire}, or \skill{Hand-to-Hand} grant \SPOTTED{} markers exactly as attacks of those type.

Maneuvers cannot be used to place an Aspect on a unit. They can only be used to place Aspects on a zone.

Maneuvers are affected by range but not line of sight. The table has to agree that the maneuver works despite the range and line of sight facts, however.

Permanent Aspects are Aspects that affect a zone directly. This includes things like \aspect{Cratered roadway}, \aspect{Forest fire}, and so on. Transient Aspects are Aspects that derive from the continuous action of an individual. \aspect{Targeting laser}, \aspect{Covering fire}, and so on.

Transient Aspects last only until the placing unit acts again, though it may use the Aspect in this last turn of its existence.

The caller determines whether an Aspect is permanent or transient.

% \clearpage
% \newpage
\vfil
\subsection{Spot}\label{sec:platoon-combat-spot}

Roll \skill{Observation} against \skill{Camouflage} for any artillery that has fired or any other unit in Line of sight. Place a \SPOTTED\ marker on the unit including the number of shifts (or a \SPOTTED\ token for each shift if you're using glass beads or coins or similar). Failure by three or more generates spin for the defender. If the unit already has a \SPOTTED\ marker, increase it by the number of shifts.

Spotting is not affected by range but is affected by line of sight. Units can have no more than four \SPOTTED\ markers.

\begin{sidebox}{SPOTTED}
You can't kill it if you don't know where it is. We assume that everyone starts out hidden, so rather than model stealth we model being spotted and stealth takes care of itself.

Hence the \SPOTTED\ marker. You can't shoot anything that doesn't have a \SPOTTED\ marker, though you can attack the zone (with the \stunt{Zone Effect} stunt) or attempt to spot.

Attack actions are restricted to units that have been spotted. All other actions can proceed without having a spotted target.

\SPOTTED\ markers are eroded by successful attempts to hide (\skill{Camouflage}) or by moving while out of the line of sight.

\SPOTTED\ markers are gained by enemy activity (\skill{Spot} observation) or by firing your weapons.

\SPOTTED\ markers might be a stack of chips under the unit (one per spotted value) or glass beads touching the model or a paper chit with the number written on it. As the value will go up and down, an easy way to record this change is the only driving requirement beyond identifying the value with a particular unit.
\end{sidebox}


\newpage
\subsection{Cover}\label{sec:platoon-combat-cover}

Roll \skill{Camouflage} against a target value of zero and reduce any \SPOTTED\ markers on the unit by the number of hits achieved. Covering is not affected by range or line of sight. It may be performed by units not currently in platoon membership.

\subsection{Limitations}\label{sec:platoon-combat-limitations}

You may not use the same Skill for your action that was used in defense (unless allowed by a Stunt).

