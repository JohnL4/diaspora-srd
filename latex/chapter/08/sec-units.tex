\section{Units}\label{sec:Units}
\iflandscape{\vfil}{}


A unit is the minimum element represented on the map for each type: a single miniature or counter. For infantry, that's a team of a few soldiers. For armour that's one vehicle. For artillery that's a battery.

There are four types of team units: infantry, artillery, armour, and aircraft. For each platoon, one unit (which may not be aircraft) is also designated the leader. There is no maximum number of units in a platoon.

Each unit in a platoon grants the platoon one fate point. All fate points are kept on the platoon and spent from the platoon. All Consequences are on the platoon.

Similarly, spin counters are associated with platoons and not with units. They may be spent by any unit in the platoon. Spin expires after having had one complete turn in which to use it (thus if spin is acquired during a defensive roll, it lasts until the end of the platoon's next opportunity to act whereas if it is acquired during movement, say, it lasts until the end of the platoons next opportunity to act and not the opportunity in which it moved).

Units that are not normally part of a platoon (typically aircraft) are associated with a particular platoon and donate their fate point to that platoon. They draw fate points from that platoon when invoking, tagging, or compelling.

All units have Skills, Aspects, Stunts, and a \Morale\ stress track. Skills are an n-cap pyramid (i.e. one Skill at level three, two Skills at level two, and three Skills at level one) or a column (i.e. one Skill at level four, one Skill at level three, one Skill at level two, and one Skill at level one).

All units have one Aspect and contribute one fate point to their platoon. Infantry units have a baseline of zero Stunts, plus one Stunt for each technology level. All other units have one Stunt, plus one additional Stunt for each technology level; consequently, units at T-1 or lower do not have Stunts. As described below, the platoon leader always has one additional Stunt, regardless of technology level. No unit may have negative Stunts.

Units have only one stress track: \Morale. When a unit takes a hit past the end of its \Morale\ track that cannot (or will not) be mitigated by a platoon \Consequence, it is eliminated. The narrative associated with this elimination can be determined by the table: it might represent panic and dispersal or surrender; a complete lack of morale is also adequately explained by everyone being killed. Some combination of the three is most likely.  The mechanical effect at this scale is the same. As with other stress tracks, a hit on a marked box rolls up to the next unmarked box.

\subsection{Skills}
\label{sec:platoon-combat-skills}

All Skills are eligible to be chosen by any unit type. See \autoref{tab:platoon-unit-skills} for a list of the unit skills available.

\begin{table}[ht]\centering
\begin{tabular}{lp{0.6\columnwidth}}
\toprule
Skill		& Roll to: \\
\midrule
Anti-air	& inflict harm on aircraft units \\
Armour		& defend against fire \\
Camouflage	& avoid detection \\
Command		& improve (repair) morale \\
Signals		& jam or unjam a unit's communications \\
Direct Fire	& inflict harm in line of sight \\
Hand-to-Hand	& inflict harm in the same zone \\
Indirect Fire	& inflict harm beyond line of sight (including off map) \\
Movement	& move. Without \skill{Movement}, a unit is only capable of advancing a single zone (the free move) per turn. \\
Observation	& detect and locate enemy artillery fire \\
\midrule
Skill		& Description \\
\midrule
Specialist	& sink \Skill\ that has no mechanical effect. The apex \Skill\ of a unit not designed for the represented form of combat (eg, artillery crews as infantry or a staff convoy as armour). \\
Veteran		& modifier to \Morale\ track \\
\bottomrule
\end{tabular}
\caption{Platoon unit skills}
\label{tab:platoon-unit-skills}
\end{table}


Units may only roll Skills for which they are trained. The only
exception is when defending against an opposed roll, in which case the
untrained Skill is presumed to be zero. The only case presented here
is Armour, though with an appropriate Stunt  Camouflage may also
qualify.


% \subsection{Unit Stunts} % should be?
\subsection{Stunts}\label{sec:unit-stunts}
\begin{description}
\item[Cavalry]
this unit is undersized and overpowered, so its maximum move is increased by one (infantry, armour, or artillery only). Infantry units may take this \Stunt\ multiple times: the unit may be thought to have an intrinsic vehicle for mobility. When taken by an armour unit, the \Stunt\ is designated ``\stunt{Light}.''

\item[Special forces]
this unit is not automatically spotted when it shares a zone with an enemy unit (infantry only).

\item[Wireless]
unit is attached to an integrated communications net, increasing command range by 1. May be taken multiple times.

\item[Engineer]
may use a successful maneuver roll to use up the free-tag on an enemy-applied \Aspect\ or make two maneuver rolls on the zone it is in instead of the usual one (infantry and armour only).

\item[Guerrilla tactics]
attacks from this unit never generate spin for the defender (infantry only).

\item[Highly trained]
this unit has one additional morale box.

\item[Infantry carrier]
this unit can carry infantry (armour or aircraft only). One infantry unit in the zone can move with this carrying unit (including traversing the Re-arm track for aircraft). The infantry unit cannot act this turn (before or after the move). The unit can begin the game carrying its infantry load. For aircraft, when the aircraft re-enters the map, the infantry is deployed and may act normally; the aircraft may not otherwise act while deploying infantry. Carried infantry do not have to be in the same platoon as the carrier.

\item[Interceptor]
if this unit is on the LAUNCH! box, it may enter the map any time an enemy aircraft enters the map and act immediately before the target aircraft can act. It may act only against this target aircraft (aircraft only).

\item[Irregulars]
this unit is an irregular non-professional unit (a sink \Stunt, chosen only to model a unit that is less effective than other units of the same technology level). Other sink \Stunts\ can be invented to fit the scenario: \stunt{Slow}, to represent a low rate of fire, etc.

\item[Long range]
ignores one zone for attack roll range modification. May be taken multiple times.

\item[Orbital]
this unit can only be attacked by fire from other orbital units (artillery only). Orbital units that are attacked with the jam action, however, take damage as though attacked with weapons (in addition to the effects of jamming).

\item[Prepared positions]
this unit was set up long before the battle (artillery only). Before combat begins, it may add a the \Aspect\ of \aspect{Locked in} to any two zones on the map. This \Aspect\ can be free-tagged by any allied artillery unit, and remains an \Aspect\ on the zone which may be tagged normally thereafter.

\item[Scatterable mines payload]
this unit can deliver area-denial ordnance (ie, mines). Pass values placed by the unit from an interdiction strike are permanent (artillery and aircraft only).

\item[Scout]
this unit can continue movement after entering a zone containing enemy units (infantry and armour only).

\item[Skill substitution]
With an appropriate narrative, additional \Stunts\ may be designed to allow \Skill\ substitutions. Each unit may only ever have one \Skill\ substitution \Stunt. The following are offered as representative examples.

\begin{description}
\item[Agile]
can use \skill{Movement} in place of \skill{Armour} (armour only).

\item[Graphite payload]
this unit can deliver payloads designed to interrupt electrical and electromagnetic function (artillery and aircraft only).  It may use its \skill{Indirect Fire} Skill to effect Jam attacks (which would normally use the \skill{Signals}). Note that this can be combined with \stunt{Zone Effects} to jam all units in a zone (regardless of owner).

\item[Shoot and scoot]
this weapon system is designed to be fired while on the move or to move very soon after firing a mission. It may use its \skill{Movement} Skill instead of \skill{Camouflage} (artillery only).

\item[Technology enhancement]
increase any \Skill\ by one. This \Stunt\ may be taken at most twice per \Skill, for a total bonus of +2.

\item[Stealth technology]
designed to hide, this unit can use \skill{Camouflage} in place of \skill{Armour} (armour only).
\end{description}

\item[VTOL]
this unit is designed to stay on target --- once on the map it may remain, moving a maximum of 1 zone (its free move) per turn (aircraft only).

\item[Zone effects]
this unit may attack all units in the target zone with one roll at -2 (armour, artillery, or aircraft only). Units do not need to be spotted to be attacked in this fashion.
\end{description}

\subsubsection{Leadership stunts}

Each platoon leader additionally chooses one of the following four Stunts.

\begin{description}
\item [Battlefield genius]
units can be one zone further from the Leader than otherwise allowed.

\item[Logistics genius]
units in platoon do not have the \aspect{Out of ammo} Aspect.

\item[Tactical genius]
units in platoon ignore one extra zone of range when attacking.

\item[Not a genius]
sink Stunt for crap commanders.
\end{description}


\subsection{Stress Tracks}\label{sec:platoon-unit-stress-tracks}

All units have a single stress track, \Morale. A platoon may expend \Consequences to mitigate hits past the end of the track on a unit. A platoon has three \Consequences\ to allocate and each can mitigate two shifts. As is standard in FATE, a platoon \Consequence\ becomes an Aspect and may be free-tagged once or compelled or tagged normally to affect any unit in the platoon.

\begin{itemize}
\item Infantry units have a base \Morale\ stress track of two boxes.
\item Armour units have a base \Morale\ stress track of one box.
\item Artillery units have a base \Morale\ stress track of two boxes.
\item Aircraft units have a base \Morale\ stress track of one box.
\end{itemize}

Unit \Morale\ tracks are modified by the unit's \Veteran\ skill. Leader units also gain a bonus \Morale.

\begin{itemize}
\item Leader units increase the base \Morale\ stress track by two.
\item Units with Veteran 1 or Veteran 2 increase their \Morale\ stress track by one.
\item Units with Veteran 3 or Veteran 4 increase their Morale stress track by two.
\item Units with Veteran 5 or Veteran 6 increase their Morale stress track by three.
\item Some stunts may further alter the length of the Morale stress track.
\end{itemize}


\subsection{Aspects}
\label{sec:platoon-unit-aspects}

All units have one descriptive Aspect chosen by the owner and add one fate point to their platoon. All units also have the Aspect \aspect{Out of ammo}. A unit, when spending fate points, expends platoon fate points. When a unit gains a fate point through a compel, that fate point belongs to the platoon.


\subsection{Infantry}
\label{sec:Infantry}

% \input{C08/TB-infantry}

Infantry units represented a small number of individuals of similar or concerted equipment: a unit typically represents 2-5 individuals, though it could be as many as 12. Specific weapons and armour per individual are not modelled except as they are represented in the Skill and Stunt list.

Infantry have a 3-cap Skill pyramid. Infantry units choose one Skill at rank 3, two Skills at rank 2, and three Skills at rank 1. They have a Morale track two boxes long. If the unit has Veteran at rank 1 or 2, the Morale track is three boxes long. If the unit has Veteran at rank 3, the Morale track is four boxes long.

The maximum movement for infantry is two zones. Infantry units may move a maximum of two zones regardless of their movement roll.


\subsection{Armour}
\label{sec:armour}

Armour units are individual tanks, cars, or other mobile armoured platform. They represent all of the equipment present on precisely that model of vehicle.

Armour has a 4-cap Skill column. Armour units have one Skill at rank 4, one at rank 3, one at rank 2, and one at rank 1. They have a Morale track one box long. If the unit has Veteran Skill 1 or 2, the Morale track is two boxes long. If the unit has Veteran Skill of 3 or 4, the Morale track is three boxes long.

The maximum movement for armour is four zones. Armour units may move a maximum of four zones regardless of their movement roll result.


\subsection{Artillery}
\label{sec:Artillery}

Artillery units are equipment capable of Indirect Fire which are kept off map. They move only in a notional sense insofar as they can roll Movement as a defensive roll against counter-battery detection and Indirect Fire. Infantry-based artillery (such as mortar or grenade launcher crews) should be represented by including an Indirect Fire Skill on an infantry unit. Artillery batteries that need to be represented on the map for purposes of the scenario should be represented by their attending personnel as lightly armed infantry units.

It can be handy to create an off-map artillery card for artillery platoons, especially if they have a command range greater than one. This will greatly simplify aircraft attacks on the artillery platoon. Artillery has a 3-cap Skill column. Artillery units have one Skill at rank 3, one at rank 2, and one at rank 1. Their Morale track is two boxes long. If the unit has Veteran Skill 1 or 2, the Morale track is three boxes long. If the unit has veteran Skill 3, then the Morale track is four boxes long.

Artillery may make Movement rolls to change position on their battery
card if there is more than one zone on the card. Moving artillery
units do not remove \SPOTTED\ markers, however.

Artillery can only fire on targets that are in line-of-sight to a
friendly unit that is currently attached to a platoon (or does not
need to be) and has no Out Of Communications (\OOC) counters.

All artillery units in the same platoon are considered to be in the
same zone as their leader for purposes of command and communication,
and for purposes of any attacks that affect all targets in a single
zone when attacked by aircraft, unless they have a Stunt that allows a
greater command range. All members of an artillery platoon not
situated on the map must be on the same artillery card (they cannot be
spread over multiple cards). Not all units in an artillery platoon
need to actually be artillery units (there might be an infantry leader
unit supplying comms and other coverage and an armour unit supplying
AA for example).
Non-artillery units in an artillery platoon must be in the off-map
artillery card in order to be associated with the platoon. This means
that, although the leader unit might be represented by something other
than artillery (an armour or infantry unit might be more advantageous)
it will gain no advantages from its mobility on the map.


\iflandscape{}{\vfil}
\subsection{Aircraft}
\label{sec:Aircraft}

Aircraft are independent units and therefore require no leader unit. They also move differently from other units: an aircraft unit may place itself on any zone on the map when its turn to act comes up.

Aircraft are automatically spotted when they are on the map.

Aircraft movement is different from other units:

Aircraft begin on the \LAUNCH\ box of the Re-arm track.

While on the Re-arm track, an aircraft unit may make Movement rolls to pro\-gress along the track.

An aircraft on the \LAUNCH\ box at the beginning of its turn may be placed on any zone on the map. Its turn is now over.

Once an aircraft acts while on the map (usually in the turn after it has moved there), it is returned to the RE-ARM box of the Re-arm track.

An aircraft unit on the map may act as any other unit except that it may not make a Movement roll.

Aircraft have a 4-cap Skill column. Aircraft units have one Skill at rank 4, one at rank 3, one at rank 2, and one at rank 1. They have a \Morale\ track one box long. If the unit has \Veteran\ Skill 1 or 2, the \Morale\ track is two boxes long. If the unit has \Veteran\ Skill 3 or 4, then the Morale track is three boxes long.

The maximum movement for aircraft is zero zones as they are not represented on the map. Aircraft units do not move on the map in the same fashion as ground units and so do not make Movement rolls except to decrease their re-arm time.

Aircraft increase range by 1 for all distance calculations (both against them and against other targets).

Aircraft can only be attacked by the Anti-air Skill.


\subsection{Leadership}
\label{sec:Leadership}

Each platoon has one, and only one, leader unit. An infantry, armour, or artillery unit can be designated a leader.

A leader unit may perform a  action in addition to its normal action. It may, therefore, make two  actions in a turn.

Leaders have one Stunt chosen from the leadership Stunts list in addition to the Stunts for their base unit type.

Leaders add two morale boxes to the unit to which they are attached. The maximum movement for leader units is their base unit's maximum movement. Leader units contribute one extra fate point to the platoon (one for the leader and one for the base unit).

Leader units have two Aspects --- one for the leader and one for the base unit --- in addition to \aspect{Out of ammo}.


\subsection{Typical Units}
\label{sec:typical-units}

%%% macros for unit stat blocks
% parameters:
% - * indicates we need to output a tikz block
% - leader (+)
% - unit name
% - platoon & unit number
% - aspect
% - skill list
% - stunt list
% - movement
% - morale
%% We assume this is already inside a tikz block
\DeclareDocumentCommand \infantryblock {S{+}mmmmmmm} {
  \def\unitskills{#5}
  \def\unitstunts{#6}
  \unitblock{Infantry}{#2}{#3}{#4}
  \unitskillsblock{(-1.25,3.4)}\unitskills
  \unitstuntsblock{(0.75,3.3)}\unitstunts
  \unitmovementblock{(1.75,1)}{2}{#7}
  \unitmoraleblock{}{2}{#8}
  \unitleaderblock
}
\DeclareDocumentCommand \artilleryblock {S{+}mmmmmmm} {
  \def\unitskills{#5}
  \def\unitstunts{#6}
  \unitblock{Artillery}{#2}{#3}{#4}
  \unitskillsblock{(-1.25,3.4)}{\unitskills}
  \unitstuntsblock{(0.75,3.3)}{\unitstunts}
%   \unitmovementblock{(1.75,1)}#7
  \unitmoraleblock{}{2}{#8}
  \unitleaderblock
}
\DeclareDocumentCommand \armourblock {S{+}mmmmmmm} {
  \def\unitskills{#5}
  \def\unitstunts{#6}
  \unitblock{Armour}{#2}{#3}{#4}
  \unitskillsblock{(-1.25,3.4)}{\unitskills}
  \unitstuntsblock{(0.75,3.3)}{\unitstunts}
  \unitmovementblock{(-0.4,1)}{4}#7
  \unitmoraleblock{}{1}{#8}
  \unitleaderblock
}
\DeclareDocumentCommand \aircraftblock {S{+}mmmmmmm} {
  \def\unitskills{#5}
  \def\unitstunts{#6}
  \unitblock{Aircraft}{#2}{#3}{#4}
  \unitskillsblock{(-1.25,3.4)}{\unitskills}
  \unitstuntsblock{(0.75,3.3)}{\unitstunts}
%   \unitmovementblock{(-0.4,1)}{4}#7
  \unitmoraleblock{}{1}{#8}
}
\DeclareDocumentCommand \unitblock {mmmm} {
  \def\unittype{#1}
  \def\unitname{#2}
  \def\unitnum{#3}
  \def\unitaspect{#4}
  \unitblockborder
  \unitheaderblock{\unittype}{\unitname}{\unitnum}{\unitaspect}
}
\def\unitblockborder{
  \draw [rounded corners] (-3,0) rectangle +(6,5.5);
}
% identifying information
\def\unitheaderblock#1#2#3#4{
  \draw [rounded corners=3pt]
    (0,5.1) [anchor=base east] node {\bfseries #1 Unit:}
      [anchor=base west] node{\sffamily #2} ++(0,-0.1)--++(2.2,0)
    (-0.5,4.55) [anchor=base east] node {Platoon/ID:}
      [anchor=base west] node{\sffamily #3} ++(0,-0.1)--++(1.25,0)
    (-1.25,4.0) [anchor=base east] node {Aspect:}
      [anchor=base west] node{\sffamily #4} ++(0,-0.1)--++(4,0)
    ;
}
% Leader checkbox section
\DeclareDocumentCommand \unitleaderblock {S{+}}{
  \draw [rounded corners=3pt]
    (2.95,4.7) [left] node {Leader?}
      [below left] node {\scriptsize (+2 morale)}
      ++(-0.1,0.2) rectangle +(-0.5,0.5)
    ;
  \IfBooleanT{#1}{
  \draw [line width=1]
    (2.4,5.15) -- +(0.1,-0.25) -- +(0.4,0.15)
    ;
  }
}
% morale boxes
\def\unitmoraleblock#1#2#3{
%   \typeout{Morale:~#1~#2~#3}
  \draw [rounded corners=3pt]
    (-2.85,0) [above right] node {morale}
    \foreach \x in {1,...,#2}
      { ++(0.2,0.5*\x) rectangle +(0.5,0.5) }
    ;
  \filldraw [rounded corners=6pt,fill=gray]
    (-2.65,0) ++(0.0,0.5*#2)
    \foreach \x in {1,...,#3}
      { ++(0,0.5*\x) rectangle +(0.5,0.5) }
    ;
}
% skills
\def\unitskillsblock#1#2{
  \draw #1
    [anchor=base west] node {Skills}
    \foreach \snum/\smod/\snam in #2 {
      \skillline{\snum}{\smod}{\snam}
    }
    ;
}
\def\skillline#1#2#3{
    ++(0,-0.5*#1)
      [above left] node {#2}
      [above right] node {\scriptsize #3}
      +(0,0.1) -- +(1.75,0.1)
}
% stunts
\def\unitstuntsblock#1#2{
  \draw #1
    [above right] node {Stunts} ++(0,0.1)
    \foreach \snum/\smod/\snam in #2 {
      \stuntline{\snum}{\smod}{\snam}
    }
    ;
}
\def\stuntline#1#2#3{
    ++(0,-0.5*#1)
      [above right] node {\scriptsize #2}
      +(0,0.1) -- +(2.0,0.1) node [above] {\scriptsize #3}
}
% movement
\def\unitmovementblock#1#2#3{
  \draw #1
    [above] node {Movement}
    +(0,-0.85) node {\scriptsize base #2}
    ++(-0.75,0)rectangle +(1.5,-0.75)
    ;
}

%%% list processing
%%% from http://maraist.org/comma-separated-lists-in-latex_08-2009/

% kill a single token: match it (removes it from the stream) and
% immediately do nothing with it.
% \def\@swallow#1{}
% %
% \def\forlist#1#2#3{%
% %   \typeout{DLM:~ forlist~ args~ `#1'~ `#2'~ `#3'}
%   \@NullTest@forlist{#1}{#2}#3\@end@forlist}
% %
% \def\@NullTest@forlist#1#2{%
%   \@ifnextchar\@end@forlist%
%     {#1\@swallow}%
%     {\@prime@forlist{#1}{#2}}}
% %
% \def\@prime@forlist#1#2#3\@end@forlist{%
%   \@test@forlist{#1}{#2}#3,\@end@forlist}
% %
% \def\@test@forlist#1#2{%
%   \@ifnextchar\@end@forlist%
%       {#1\@swallow}%
%       {\@more@forlist{#1}{#2}}}
% %
% \def\@more@forlist#1#2#3,#4\@end@forlist{%
%   #2{#3}{\@test@forlist{#1}{#2}#4\@end@forlist}}

%%% Another approach
\def\TERMINATOR{LISTTERMINATORSTRING}
\def\SkillList(#1)(#2){\xSkillList#1,LISTTERMINATORSTRING, }
\def\xSkillList#1,{\def\TMPlist{#1}%
%   \typeout{item: \TMPlist}
  \ifx\TMPlist\TERMINATOR\SkillListItem{#1}
  \else\SkillListItem{#1}%
    \expandafter\xSkillList
  \fi}
\def\SkillListItem#1{ %skill: #1 ~}
  \draw ++(0,-0.5) [above right] node {\scriptsize #1} ;
  }
%%%

%%% HACK approach
\DeclareDocumentCommand \HACKSkillList {mmmmmmm} {
  \typeout{HACKSkillList: #1 . #2 . #3 . #4 . #5 . #6 . #7 }
  \draw #1
%     ++(0,-0.5) [above right] node {\scriptsize #1}
    ++(0,-0.5) [above right] node {\scriptsize #2}
    ++(0,-0.5) [above right] node {\scriptsize #3}
    ++(0,-0.5) [above right] node {\scriptsize #4}
    ++(0,-0.5) [above right] node {\scriptsize #5}
    ++(0,-0.5) [above right] node {\scriptsize #6}
    ++(0,-0.5) [above right] node {\scriptsize #7}
  ;
}
\begin{sidebox}{Typical T-1 units}
\centering
\begin{tikzpicture}[scale=0.75]
\tikzstyle{every node}=[transform shape]
\begin{scope}
\infantryblock%
  {Marines}% name
  {B--1}% number
  {they'll never see us comin'}% aspect
  {1/+3/Camouflage,2/+2/Direct Fire,3/+2/Observation,4/+1/Armour,5/+1/Hand-to-hand,6/+1/Command}% skills
  {1/Special Forces/T-1,2/{}/T-2,3/{}/T-3,4/{}/T-4}% stunts
  {}% move
  {4}% morale
\end{scope}
\begin{scope}[yshift=-6cm]
\armourblock%
  {Light Tanks}% name
  {B--2}% number
  {Outfitted for rough terrain}% aspect
  {1/+4/Armour,2/+3/Direct Fire,3/+2/Movement,4/+1/Anti-Air}% skills
  {1/Light/free,2/{}/T-1,3/{}/T-2,4/{}/T-3,5/{}/T-4}% stunts
  {2}% move
  {4}% morale
\end{scope}
\begin{scope}[xshift=6cm,yshift=0cm]
\artilleryblock%
  {Mortar Team}% name
  {B--3}% number
  {Pin-point accuracy}% aspect
  {1/+3/Indirect Fire,2/+2/Camouflage,3/+1/Movement}% skills
  {1/Zone Effects/free,2/{}/T-1,3/{}/T-2,4/{}/T-3,5/{}/T-4}% stunts
  {2}% move
  {4}% morale
\end{scope}
\begin{scope}[xshift=6cm,yshift=-6cm]
\aircraftblock%
  {Bomber Squad}% name
  {B--4}% number
  {Smart-bomb payload}% aspect
  {1/+4/Direct Fire,2/+3/Observation,3/+2/Anti-air,4/+1/Movement}% skills
  {1/Long Range/free,2/{}/T-1,3/{}/T-2,4/{}/T-3,5/{}/T-4}% stunts
  {2}% move
  {4}% morale
\end{scope}
\end{tikzpicture}

% \forlist{\EmptySkill}{\SkillLineNode}{one,two,three}
%   \SkillList(one,two,three)

\end{sidebox}


\vfil

\subsubsection{Typical Infantry}

Skill tree: Camouflage 3, Direct Fire 2, Observation 2, Armour 1, Hand to-Hand 1, Command 1.

Infantry are used to capture and hold territory as well as to provide spotting for heavier units.  They have NCOs capable of regrouping broken units and are adept at close combat as well as ranged. They excel at not being seen.

\subsubsection{Typical Armour}

Skill tree: Armour 4, Direct Fire 3, Movement 2, Anti-air 1.

There are two core types of armour: assault tanks designed to move into heavy fire and attack units spotted by associated infantry, and tank hunters, which would swap Armour and Direct Fire.

\vfil

\subsubsection{Typical Artillery}

Skill tree: Indirect Fire 3, Camouflage 2, Movement 1.

Artillery's immediate objective is to destroy spotted enemy equipment. It does so by projecting a huge volume of fire, which makes it suddenly very vulnerable. It offsets this vulnerability by immediately moving and re-hiding.

\subsubsection{Typical Aircraft}

Skill tree: Direct Fire 4, Observation 3, Anti-air 2, Movement 1.

Aircraft Skill trees are capped by their primary design goal --- Direct Fire for ground attack vehicles, Observation for reconnaissance craft, and Anti-air for interceptors. Most aircraft will be capable in all of these. Movement for aircraft indicates their re-arm time --- high Movement rates indicate rapid re-arming cycles, trading off for specialty effectiveness such as ground attack or anti-air capability.



