\section{Stress Tracks}
\label{sec:Stress Tracks}

Every character has three stress tracks: Health, Composure, and Wealth. Each has a relevant Skill that can modify the number of boxes in the track. Some Stunts can modify the number of boxes as well. The Health track is associated with the Stamina Skill. Composure is associated with Resolve. Wealth is associated with Assets.

Tracks start out with three boxes in them, which represents a character untrained in the relevant Skill. If the relevant Skill is 1 or 2, the track is 4 boxes; if it is 3 or 4, the track is 5 boxes; if it is 5, the track has 6 boxes.

\begin{description}
\item[Health]~

The Health stress track represents how close you are to sustaining an injury that will affect your performance and require time to recover from. It does not represent actual injury. The Health stress track is modified by Stamina.

The Health stress track takes hits when a character loses a combat check --- he takes a bullet, gets burned by a laser, is cut by a knife, or is punched in the eye. It's not an effective injury unless it causes a Consequence --- there is no mechanical effect to having a box filled in a track. It's when boxes you don't have get filled that you have trouble.

\item[Composure]~

The Composure stress track represents how close you are to mental breakdown. It does not represent the degree of actual breakdown. The Composure stress track is modified by Resolve.

The Composure stress track takes hits when a character loses a social combat check and sometimes when under fire in combat. As with Health, it's not an effective hit until a Consequence is applied.

\item[Wealth]~

The Wealth stress track represents how close you are to having real financial trouble. It does not represent actual debt or financial ruin but rather how close you are to feeling the ramifications of debt. The Wealth stress track is modified by Assets.

The Wealth stress track takes hits when a character fails a Wealth check when buying something or assisting with monthly ship maintenance. It follows all the same rules as the other stress tracks do, though recovery can take longer.

\end{description}

\subsection{Consequences}\label{sec:Consequences} % \href{sec:id79}

Any time you are taking hits to a stress track, you can reduce the number of hits with Consequences. A mild Consequence reduces the incoming hits (usually shifts) by one, a moderate Consequence reduces them by two, and a severe Consequence reduces by four. Normally you can have at most three Consequences and no more than one of each kind. Each Consequence is a kind of Aspect and represents real damage: ``Shattered jaw'' or ``Hopelessly depressed'' or ``Hunted by loan sharks'' are all good. Each is free-taggable by your enemies once. Consequences are discussed in more detail in each combat chapter.


\subsection{Taken Out}\label{sec:Taken Out} % \href{sec:id80}

When you take a hit that would go off your stress track, you are Taken Out. Whoever scored that fatal hit gets to decide what happens to you. You could be dead or you could just be unconscious. Or, with a financial hit, you could be slaving away in a burger joint with no prospects of happiness or promotion.


\subsection{Concessions}\label{sec:Concessions} % \href{sec:id81}

At any time in a fight of any kind, if you have not been Taken Out, you can offer a concession. Referees especially like this to keep villains alive for another day. A concession is something you offer to end the combat instead of play it through. If your opponent accepts it, it's true. Good concessions give something up but keep you in play. Things like, ``I tell him the combination to the safe but sneak out while he's not looking, escaping back to my safe house'' or ``Our ship escapes through the slipknot but our motors are not working so we're stranded on the other side'' are great.



