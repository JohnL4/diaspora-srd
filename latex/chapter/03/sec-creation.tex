\section{Creation}

\begin{sidebox}{Original Material}

You can download two different character sheets from the Diaspora home page:

\begin{itemize}
\item the \href{http://www.phreeow.net/Diaspora/Diaspora%20character%20sheet.pdf}{standard character sheet}
\item a \href{http://www.phreeow.net/Diaspora/Diaspora%20character%20sheet%20con.pdf}{folding character sheet}
\end{itemize}

\end{sidebox}


\index{character creation}
Character creation is ideally done as part of the first session: characters develop naturally out of the system development, and the process of making characters in turn elaborates crucial details about the cluster. Characters are composed of four mechanical elements: their Aspects, their Skills, their Stunts, and their stress tracks (Health, Composure, and Wealth).

\begin{description}
\item[Aspects] %\index{aspects!characters}
are short, evocative statements that describe the character in ways that can be used mechanically both for and against the character as well as being points at which the referee can suggest actions to players for their characters.

\item[Skills]
are the basic abilities of the character, chosen from a list provided later in this section, and used mechanically to add to the basic roll during any conflict in which the Skill is relevant.

\item[Stunts]
are new rules that apply to the character.

\item[Stress tracks]
are indications of how stressed the character is physically, mentally, and financially.
\end{description}

Aspects derive from the character's background. Skills and Stunts are selected after the background is constructed. Stress tracks have a basic rating modified by some Skills and Stunts.

Some phases are collaborative.

\index{character creation!phases}
For each phase, players should follow this procedure:
\begin{enumerate}
\item Write a short paragraph describing the events of this phase (think in terms of allocating no more than five minutes for this each time; less is fine)

\item In turn, read them out to each other. This is important, as it helps others learn about your character at the same time that you do.

\item Select two Aspects, derived from the written paragraph. They can literally be phrases pulled from the paragraph or new phrases relevant to the phase. This can be done individually, or as a consultative process with the table. Once selected, everyone should read out their two derived Aspects. You'll find that there is plenty of fiddling with Aspects at this point --- have fun with it and don't get too stuck on procedure: your core objective is to come up with cool Aspects, and so listening to the table and how they respond to your ideas can often yield exciting results.

\item Repeat for each of the five phases, until each character has ten Aspects.
\end{enumerate}

Going through the five phases for four characters might take 45-60 minutes, including reading aloud the gradual development of the characters after each phase.

\rulebox{Characters have ten Aspects.}

\begin{description}
\item[phase one: growing up]~

This phase should establish the character's home system and maybe some information about his family and upbringing. Information written here might reasonably feed back into the system description for the world: it's likely that the player will find new ideas percolating about the world as he wonders about his character's place in it. The two Aspects derived from this phase might include features of the home world, such as how its technology or political structure impacts the character.

\item[phase two: starting out]~

This phase describes the character picking a direction in life. It might be a career choice or an education or it might be a circumstance forced upon him, but it should be a formative choice that establishes who the character has decided he will be. Career decisions often mean the player decides whether the character has gone to space before, and, if so, in what capacity. Does he serve on a ship? Is he part of some military or government organization? An independent trader? A belter mining asteroids? Scientist, ninja, spy? Perhaps he is a barbarian, uncomfortable with all the technology that drives the cluster; perhaps he is an explorer, or a drive mechanic, or someone who never found a career, and has been wandering the stars looking for purpose...

\item[phase three: moment of crisis]~

Now players will write a brief description of an event that created change in the character --- something that the character would talk about later (maybe to his pals around drinks, maybe only to his wife, maybe to himself while in his sleep with the cold sweats and the voices and the screaming). The moment of crisis must reference the character of the player to your right --- you want to bring them in as an observer, or a participant, or even as the focus of the event. This is an opportunity to help define another character as well as your own.

\item[phase four: sidetracked]~

This phase is about events out of your control. As with life, not everything goes as planned, and it may be that your life has taken an unexpected turn. This phase revisits the player to your left character's moment-of-crisis event from your character's perspective. They wrote you into their background in phase three --- now is your chance to tell it the way your character saw it happen.

\item[phase five: on your own]~

In the final phase, write briefly about where the character is now. What are his immediate needs and goals? What is he doing to get by in a hostile universe?
\end{description}

