\section{Opposition}\label{sec:opposition} % \href{sec:id91}

One aspect of conflict in the stories you will tell is going to be combat, whether physical or social, and that will require some kind of mechanically represented opposition.

This process acknowledges that the player characters are exceptional. They aren't superheroes, but they are exceptionally competent.

\subsection{Non-Player Characters}\label{sec:Non-Player Characters} % \href{sec:id92}

It's not necessary to create statistics for all characters that the referee will bring into play. In many cases the referee need only establish their single Skill rank in much the same manner as he would estimate a difficulty level for a static check.

Non-Player Characters (NPCs) have Skills in a pyramid just as the regular player characters have, but the peak value of the pyramid might be lower than 5. A moderate-threat NPC, for example, might be capped at Skill rank 3 (a ``3-cap'' character, for short): one Skill at 3, two at 2, and three at 1. NPCs have one Aspect for each rank in their apex Skill, one fate point per Aspect, and Stunts as appropriate (no more than 3).

This is a sufficient representation for thugs, policemen, goons, and villains. While six Skill slots does not seem like many (compared to the fifteen of the player characters), in practice it works because these characters have context-appropriate Skills. NPCs, in most cases, need only be functional in a given environment, for a short time (often only one scene), and are designed for that circumstance.

There remains, however, enough variability that a given opponent may still have a rank in a Skill that the player characters lack, and can then be co-opted and introduced into the larger story.

NPCs have Health and Composure stress tracks derived from their Skill pyramid as player characters do.

When non-player characters have not been made in advance, it is possible for the players to define Aspects for these characters through maneuvers.  This might encourage the referee to fill out the card in other ways.  The creation of NPCs becomes a collaborative process.

The referee can decide not to allow the suggested Aspect, but should offer the player something else, another Aspect to help fill the character out.  This means that a player's actions do not need to determine the Skills and abilities of the NPCs they encounter, but that through the process of interaction the players will come to know who it is that they are dealing with.

In most cases, NPCs will not take Consequences: any hit over their stress tracks takes them out of the scene. Only in cases where the referee needs the character kept alive (for plot, or because a player character has an associated Aspect) should NPCs be given Consequences.

\subsection{Animals}\label{sec:Animals} % \href{sec:id93}

Animals can be modeled precisely as non-player characters, but animal Skill diversity is probably not as high. Instead give them a \emph{Skill column}: one Skill at each rank starting at some maximum. Add a Stunt to round them out. No Skill should exceed level 6. Some Skills (such as Stamina and Resolve) will not affect tracks, but can still be used to achieve maneuvers and defenses. Any appropriate integral equipment can be modeled based on the nearest human equivalent; in most cases, it should be powered by a new Skill, Natural Weapons. All Natural Weapons or armour would also require a Stunt analogous to Integral Equipment.

Common animal Skills are: Agility, Alertness, Brawling, Charm, Intimidation, Resolve, Stamina, Stealth, Strength, Survival, and Tactics. Some animals also have a Skill in Natural Weapons, which is not available to PCs (humans with built-in weapons from a Stunt use Brawling).

Animals only have Health and Composure tracks, of whatever length the referee deems appropriate, based on the size and mass of the creature in question. A small animal might have one or two boxes, anything about human sized is three to five boxes, larger animals are 6 to 8 boxes, and giant animals may be 10 boxes or more.

Hunting larger animals therefore requires attrition, wearing down the tracks with multiple hits. Such animals can do extensive damage, and then retreat or flee. When they first achieve a hit, the players decide whether or not they want an opposing animal to take Consequences. If they do, then victory means the players can narrate the conditions of the victory (trophy!); if they do not, then the referee does (which may be death, but may also be flight).

\subsection{Mooks}
\label{sec:Mooks}

Sometimes even non-player characters are too much to represent a certain kind of threat. A pack of dogs, say, or a gang of teenagers, doesn't need full representation in the system. In such cases,  establish a \stat{threat level} to represent how much trouble these \stat{mooks} are. They will be represented without Skills or Aspects. Instead they have a single stress track that is used to mark all hits. The number of unchecked boxes on this track is also their attack and defense value for all cases. Any hit that goes past the stress track defeats all of the mooks represented by it. With mooks you do not apply the First Blood rules.

