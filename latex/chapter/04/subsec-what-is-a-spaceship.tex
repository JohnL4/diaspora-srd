% \subsection{What is a Spaceship?}\label{sec:what-is-a-spaceship}
\vfil

All science fiction spacecraft have a distinctive feel derived from their setting, and this is no different in Diaspora. Spacecraft in Diaspora are big. Space, of course, is bigger, and in the end size didn't make a great deal of difference, no matter how we chose to simulate spacecraft design.

Spacecraft are built around a symmetrical Frame, attached to which are the motors, which take reaction material and convert it into something pushed out the back end; that's how ships travel through space. We largely abstract the fuel from the conversion process since the bulk of the ship's total mass is reaction material, which is consumed and needs constant replenishment.  Without reaction material, there's no way to go anywhere.

Ships cannot enter atmosphere, as the gravity would crush the frame. All travel between planet surfaces and orbiting stations or spacecraft is done through interface vehicles.

When traveling, ships accelerate to the midpoint of their journey, turn around, and decelerate. No dogfights, no Immelmann or Crazy Ivan maneuvers: safe travel means accelerating to a midpoint at 1.0-1.5G, turn around, decelerate at 1.0-1.5G, over a period of several days. Ships are built like office towers, with small decks stacked on top of each other, which experience gravity only when the ship is under thrust.

Heat is always a problem, and an inability to dissipate heat can get one into trouble.  Burn your engines too much, or fire too many lasers, and you start to have problems in combat yourself, because of an inability to radiate heat into the darkness of space.

% ~

Both \Heat\ and \Frame\ tracks can be attacked in combat (see \autoref{cha:space-combat}), from which a ship may receive \Consequences. A third track, the \Data\ track, represents the ship's computer system: data hacking and electronic warfare (EW) generally seemed a fun and powerful dimension to add to space combat.

These concerns combine to suggest that a ships payload section is relatively small (10-30\% of the ship's mass). Given the limits on payload, space for crew, weapons, cargo, and extras is limited. Slipdrives are small to allow FTL travel within the design constraints (i.e. we wanted ships both with slipdrives and with guns), and so the limit on FTL travel comes from the point of departure, well above the ecliptic of the system.

A ship's \Vshift\ rating models its ability to change vectors: that will represent a blend of acceleration, maneuverability, and structural strength.

