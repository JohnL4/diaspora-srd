\section{Generation}\label{sec:Generation} % \href{sec:id71}

\begin{enumerate}
\item Each player (including the referee) is assigned the worlds they will develop. Everyone will use slightly different notation, but numbering the worlds to be generated on a piece of paper, allowing three or four lines per system, is sufficient. We find it helpful if everyone records the information for themselves as the process is underway.

\item In turn, each player determines the attributes for their system. Systems have three attributes:
\stat{Technology}, \stat{Environment} and \stat{Resources}.

% \begin{itemize}
% \item Technology
% \item Environment
% \item Resources
% \end{itemize}
%
% % \vfil
% \newpage

Attributes are generated by a roll of the dice (\dF). The typical world will therefore be T0 E0 R0, or a system with a sustainable garden world which is actively exploring space. See \autoref{tab:system-attribute-ratings} for descriptions of each attribute rating.

\item The Slipstream Guarantee. It is suggested (though perhaps not strictly necessary) that at least one system be able to create and maintain slip-capable ships. If no system in the cluster has T2 or higher, give the system with the lowest sum of attributes and the system with the highest sum of attributes each T2.

\item Each attribute value corresponds to a short phrase (see the table below), which may become (and will certainly influence) one of the system's Aspects. These may be noted, but as they derive automatically from the attribute value, they may just be borne in mind as the procedure continues.

\item Each player shall give each of their systems a name.

\item Players give each of their systems two Aspects that reflect their unique identities, extrapolated from the attributes. The final result is under the control of the table authority. Give it two Aspects that describe things that are not represented by the numbers or that are implied by the numbers. These things might relate to politics,philosophy, geography, hydrography, local astrography or history. Or something else...

\item The relationship between the systems within the cluster is then developed. The procedure is described below in the section \nameref{sec:linking-systems}.

\item Players should now examine their systems and their place in the cluster and add a final Aspect to each system to reflect their place in this implied web of trade and politics. Discuss the ramifications of these worlds and their placement --- who is the hub? Who controls technology? Can the resource-heavy worlds defend them? Do they need to?

\item Finally, the player generating the system should write a brief paragraph describing life in it.  This will get fleshed out further through character generation and further through play, so it's not necessary that it be comprehensive. A few questions that one might want to answer include:

\begin{itemize}
\item What does the sky look like here?
\item How does the average person live?
\item Why was this system colonized?
\item What has changed since then?
\end{itemize}

\end{enumerate}

