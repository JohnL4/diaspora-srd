\begin{wrapfigure}[14]{l}[\sidebarwidth]{\halfbarwidth}
\begin{shadebox}[Compelling Movement]{\halfbarinnerwidth}
% \begin{sidebox}{Compelling Movement}

The objective of the rule that makes you pause for a compel after every zone moved, is to more closely model terrain effects and tactical \emph{funneling} by increasing vulnerability in long moves. This also makes ambushes work really well if planned effectively in appropriate terrain --- even though the enemy knows your \emph{hidden} units are there, he has to pay fate points to flee past a choke point safely or get fate points to stop there and take the ambush the \emph{player} knows is coming. This careful use of the player/character distinction is tricky but fun.
% \end{sidebox}
\end{shadebox}
\end{wrapfigure}
