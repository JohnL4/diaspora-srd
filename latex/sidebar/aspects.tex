\begin{wrapfigure}[28]{r}[\sidebarwidth]{\halfbarwidth}%
% \begin{sidebox}{Summary: Aspects}
\colorbox{sbbackground}{\begin{minipage}{\halfbarinnerwidth}%
\sideboxtitle{Aspects}%
% \begin{sidebox}{Aspects}
The selection of a character's \Aspects\ is an essential part of character 
generation.

\Aspects\ are the catalysts for the economies of fate points. They need to be 
worded in a way that you can invoke them on yourself (for when a bonus to a 
roll is needed), but --- more importantly --- they need to invite compels from 
the referee. Otherwise you lose your fate points too quickly and there is no 
obvious source for replenishment.

A well-worded \Aspect\ can be both revealing of the character's nature, and be 
obviously invokable for both benefit and detriment.

Not all \Aspects\ can work that way, and it may emerge in play that some 
\Aspects\ do not enter into the fate point economy at all. They are the ones 
which can be traded out through the experience process.

\Aspects\ reveal something about the character that the character may not even 
be aware of. Similarly, an \Aspect\ might be a physical object (an heirloom 
weapon, or a spaceship). In making that choice, the player is telling the 
referee that this object is part of the character identity. It won't be taken 
away, but it will also confer obligations and responsibilities, so that it too 
is an active part of the economy.
\end{minipage}}
\end{wrapfigure}
% \end{sidebox}