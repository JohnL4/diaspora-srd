
\begin{sidebox}{Things}
Platoon-scale combat requires a bit more mechanical represantation than the other mini-games. You will need the following:

\newcommand{\ttilde}{$\sim$}
\begin{center}
\begin{tabular}{lr}
Miniatures or unit counters. 	& 1 per unit \\
Paper or whiteboard 		& for the map \\
String or a long ruler	 	& for line-of-sight examination \\
\marker{fate point} counters 	& \ttilde6 per platoon \\
\marker{spin} counters 		& \ttilde5 total \\
\SPOTTED{} counters 		& \ttilde10 per platoon \\
\OOC{} counters 		& \ttilde10 per platoon \\
\marker{platoon acted} markers 	& 1 per platoon \\
\end{tabular}
\end{center}

Artillery battery zones can be just a sheet of paper kept to the side of the main map.

Miniatures represent individual units, which may be any of a number of types. Since not everyone will have small-scale combat minis, 3cm squares of cardboard with a little picture drawn on them (a guy for an infantry unit, a tank for an armour unit, a big gun for an artillery unit, and a plane for an aircraft unit) can serve just as well.

Units within the same platoon can be labeled A1, A2, B1, B2, B3, etc. These codes can be tied to the unit stat sheet. An L can be written on the unit which contains the platoon leader (or the leader can always be the first number: A1, B1...).
% \vfill
\end{sidebox}
